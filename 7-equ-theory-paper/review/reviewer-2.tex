\documentclass{article}
%%%%%%%%%%%%%%%%%%%%%%%%%%%%%%%%%%%%%%%%%%%%%%%%%%%%%%%%%%%%%%%%%%%%%%%%%%%%%%%%%%%%%%%%%%%%%%%%%%%%%%%%%%%%%%%%%%%%%%%%%%%%%%%%%%%%%%%%%%%%%%%%%%%%%%%%%%%%%%%%%%%%%%%%%%%%%%%%%%%%%%%%%%%%%%%%%%%%%%%%%%%%%%%%%%%%%%%%%%%%%%%%%%%%%%%%%%%%%%%%%%%%%%%%%%%%
\usepackage{amsmath,amssymb}
% more math
\usepackage{amsfonts}
\usepackage{amssymb}
\usepackage{amstext}
\usepackage{amsbsy}
%\usepackage{dtklogos}

\usepackage{color}
\newcommand{\mt}[1]{\marginpar{\small #1}}
%%%%%%%%%%%%%%%%%%%%%%%%%%%%%%%%%%%%%%%%%%%%%%%%%%%%%%%%%%%%%%%%%%%%
% new commands
\newcommand{\nc}{\newcommand}
% operators
\renewcommand{\div}{\vec{\nabla}\! \cdot \!}
\newcommand{\grad}{\vec{\nabla}}
% latex shortcuts
\newcommand{\bea}{\begin{eqnarray}}
\newcommand{\eea}{\end{eqnarray}}
\newcommand{\be}{\begin{equation}}
\newcommand{\ee}{\end{equation}}
\newcommand{\bal}{\begin{align}}
\newcommand{\eali}{\end{align}}
\newcommand{\bi}{\begin{itemize}}
\newcommand{\ei}{\end{itemize}}
\newcommand{\ben}{\begin{enumerate}}
\newcommand{\een}{\end{enumerate}}
% DGFEM commands
\newcommand{\jmp}[1]{[\![#1]\!]}                     % jump
\newcommand{\mvl}[1]{\{\!\!\{#1\}\!\!\}}             % mean value
\newcommand{\keff}{\ensuremath{k_{\textit{eff}}}\xspace}
% shortcut for domain notation
\newcommand{\D}{\mathcal{D}}
% vector shortcuts
\newcommand{\vo}{\vec{\Omega}}
\newcommand{\vr}{\vec{r}}
\newcommand{\vn}{\vec{n}}
\newcommand{\vnk}{\vec{\mathbf{n}}}
\newcommand{\vj}{\vec{J}}
% extra space
\newcommand{\qq}{\quad\quad}
% common reference commands
\newcommand{\eqt}[1]{Eq.~(\ref{#1})}                     % equation
\newcommand{\fig}[1]{Fig.~\ref{#1}}                      % figure
\newcommand{\tbl}[1]{Table~\ref{#1}}                     % table

\newcommand{\ud}{\,\mathrm{d}}

\newcommand{\tcr}[1]{\textcolor{red}{#1}}
\newcommand{\tcb}[1]{\textcolor{blue}{#1}}
\newcommand{\tcg}[1]{\textcolor{green}{#1}}

\def\BibTeX{{\rm B\kern-.05em{\sc i\kern-.025em b}\kern-.08em
    T\kern-.1667em\lower.7ex\hbox{E}\kern-.125emX}}
%%%%%%%%%%%%%%%%%%%%%%%%%%%%%%%%%%%%%%%%%%%%%%%%%%%%%%%%%%%%%%%%%%%%

\begin{document}
\bibliographystyle{elsarticle-num}
\begin{center}
{ \Large Answers to Reviewer \#2}
\end{center}

\bigskip

\noindent Ms. Ref. No.: JOMP-D-15-00204\\
Title: ``Viscous Regularization for the Non-equilibrium Seven-Equation Two-Phase Flow Model', \\
{\it Springer Journal of Scientific Computing}\\

\bigskip

In {\color{blue}blue} are reviewer's comments, in black are our answers, in {\color{red}red} are Jean's comments,
and in {\color{green}green} are Marco's comments.

\bigskip

{\color{blue}
After the recent work by Guermond and Popov where a general class of viscous
regularizations of compressible Euler equations is investigated, the present work
proposes an extension in the case of the non-equilibrium seven-equation two-
phase flow model. To address such an issue, the authors introduce a general
viscosity within the adopted model. Then, they derive the entropy evolution
law, now perturbed by the additional viscosity. By adopting a relevant defini-
tion of the additional viscosity terms, the authors claim that a minimum entropy
principle is satisfied. This study is completed by a chapman-Enskog extension
to get the associated five-equation model in the limit of infinite relaxation co-
efficients. The analysis is achieved by considering the incompressible regime
governed by low Mach number. The authors claim that the adopted viscosity
regularization does not modify the required incompressible regime. Finally, nu-
merical illustrations are displayed in order to attempt to illustrate the relevance
of the viscosity regularizations.

My opinion about this work is not good at all since this paper looks like a poor
extension of the work by Guermond and Popov.}

We disagree with your assessment. We have extended the work by Guermond and Popov to a non-conservative system of equations
by using an entropy condition following the theory developed by Del Maso-Le Floch-Mariat (DLM) [18]. 
Based on the DLM theory the classical definitions of a weak solution and the entropy solution, respectively, can be extended for quasi conservative problems introducing a path, see [18, 19, 20]. Existence of a solution to a problem in non-conservative form has been investigated in [21]. In particular, shock waves are influenced by the path since it enters the generalized Rankine-Hugoniot jump conditions, see [20]. Note that rarefaction waves and contact discontinuities are not affected by the choice of the path. Another approach that is of interest for this paper, consists in considering the
vanishing viscosity solution of a hyperbolic non-conservative system of equations by adding a parabolic viscous regularization function of a vanishing viscosity coefficient $\epsilon$. This approach was first studied by Bianchini et al. [22] and further generalized by Alouges et al. [23].
See lines 21-67 in the introduction and also lines 188-216 in the section 3.1 entitled methodology.
\tcr{add all the things we learn thanks to reviewer \#1; 
extending to 2-phase flow is by far not a small extension because of everything
reviewer \# 1}
\bigskip




{\color{blue}
1. The main point of this work concerns the derivation of the minimum
entropy principle. Here, the establishment of this property is not clear
at all. I think that the proof is incomplete. For instance, Guermond
and Popov need (and prove) the positiveness of the density. I think that
the positiveness of partial density is here needed but no proof is given.
Moreover, I am convinced that $\alpha_k \in [0, 1]$ is also necessary and must be
proved. I urge the author to read carefully the paper by Guermond and
Popov and reconsider the establishment of their results.}

We did read the paper by Guermond and Popov. We agree with you that proof of the positiveness of the partial density was missing and we added it in lines 328-344.
We also performed the same work for the volume fraction equation still in lines 328-344. In both cases, the viscous terms are required in order to prove positiveness of the
partial density and to show that the volume fraction remains bounded within the interval [0,1].

\tcr{we can cite Guermond for the phasic regularized continuity equation and show
than $\alpha \rho$ is non-negative. Then we can work on the alpha equations (both of them) to show
that  $\alpha_k \in [0, 1]$. From this, we conclude that $\rho_k$ is non-negative.} \tcg{Showing that $\alpha \rho$ is
non-negative is trivial based on Guermond's work. Showing, however, that $\rho$ is non-negative seems trickier.}
\bigskip


{\color{blue}
2. Several times, thew author speak about uniqueness of the numerical solution
. I don’t understand the meaning of these words. Moreover, this
paper does not contains numerical derivations. Page 7, the authors refer
to Leveque (pages 27-28 in Numerical Methods for Conservation Laws),
but these two pages in the leveque’s book coincides with the introduction
of weak solutions and entropy inequalities. Nothing about uniqueness of
the numerical solution.}

We meant 'convergence of the numerical solution to the weak entropy solution' and not 'uniqueness of the numerical solution'. We corrected it throughout the paper.
We also removed the reference to Leveque's book that is not the most appropriate here as you noticed and replaced it with a more extensive paragraph in section 3.1 that refers to
the DLM theory (see lines 188-216).

\tcr{Using Leveque for everything is a mistake. We will correct this using
piece of the answer we gave to reviewer \#1} \tcg{Once again, 'entropy solution' and
not 'uniqueness of the weak solution'.}
\bigskip


{\color{blue}
3. The numerical schemes, used to get the numerical illustrations, are not
specified. However, the derivation of a numerical scheme to approximate
the weak solution of the model under consideration is a very difficult task.}

We had forgotten to give a short paragraph explaining our space/time discretization. Actually, it was present in a previous version and we removed it by mistake. It has been added now (see paragraph after line 521).

\tcr{We will add this}
\bigskip


{\color{blue}
4. The presentation of the entropy residual is absolutely not relevant. In
section 3.2, I understand the opportunity to omit the underlined terms.
However, the equation (15) turns out to be wrong. The authors have to
introduce a specific notation to designate the entropy residual.}

We strongly disagree with you. The sign of the entropy equation, which is function of the entropy residual, is used as an entropy condition for non-conservative and conservative system of equations in order to ensure
convergence of the numerical solution to a weak entropy solution [15, 19]. We double checked equation (15) and believe it is right. It is used in section 3.4 to show that the entropy residual remains positive with the proper
viscous regularization. We added a specific notation to designate the entropy residual as you requested which improved the readiness of the paper (see line 252).  

\tcr{This is where the reviewer really shows he doesn't get it} \tcg{We can still add 
a specific notation for the entropy residual as he requested it.}
\bigskip


{\color{blue}
In addition, I think that these results are not suitable to be published in Journal
of Scientific Computing. As a consequence, I do not recommend the publication
of this work.}

\tcr{Pure BS}
\bigskip


\end{document}

