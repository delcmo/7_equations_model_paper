\documentclass{article}
%%%%%%%%%%%%%%%%%%%%%%%%%%%%%%%%%%%%%%%%%%%%%%%%%%%%%%%%%%%%%%%%%%%%%%%%%%%%%%%%%%%%%%%%%%%%%%%%%%%%%%%%%%%%%%%%%%%%%%%%%%%%%%%%%%%%%%%%%%%%%%%%%%%%%%%%%%%%%%%%%%%%%%%%%%%%%%%%%%%%%%%%%%%%%%%%%%%%%%%%%%%%%%%%%%%%%%%%%%%%%%%%%%%%%%%%%%%%%%%%%%%%%%%%%%%%
\usepackage{amsmath,amssymb}
% more math
\usepackage{amsfonts}
\usepackage{amssymb}
\usepackage{amstext}
\usepackage{amsbsy}
%\usepackage{dtklogos}

\usepackage{color}
\newcommand{\mt}[1]{\marginpar{\small #1}}
%%%%%%%%%%%%%%%%%%%%%%%%%%%%%%%%%%%%%%%%%%%%%%%%%%%%%%%%%%%%%%%%%%%%
% new commands
\newcommand{\nc}{\newcommand}
% operators
\renewcommand{\div}{\vec{\nabla}\! \cdot \!}
\newcommand{\grad}{\vec{\nabla}}
% latex shortcuts
\newcommand{\bea}{\begin{eqnarray}}
\newcommand{\eea}{\end{eqnarray}}
\newcommand{\be}{\begin{equation}}
\newcommand{\ee}{\end{equation}}
\newcommand{\bal}{\begin{align}}
\newcommand{\eali}{\end{align}}
\newcommand{\bi}{\begin{itemize}}
\newcommand{\ei}{\end{itemize}}
\newcommand{\ben}{\begin{enumerate}}
\newcommand{\een}{\end{enumerate}}
% DGFEM commands
\newcommand{\jmp}[1]{[\![#1]\!]}                     % jump
\newcommand{\mvl}[1]{\{\!\!\{#1\}\!\!\}}             % mean value
\newcommand{\keff}{\ensuremath{k_{\textit{eff}}}\xspace}
% shortcut for domain notation
\newcommand{\D}{\mathcal{D}}
% vector shortcuts
\newcommand{\vo}{\vec{\Omega}}
\newcommand{\vr}{\vec{r}}
\newcommand{\vn}{\vec{n}}
\newcommand{\vnk}{\vec{\mathbf{n}}}
\newcommand{\vj}{\vec{J}}
% extra space
\newcommand{\qq}{\quad\quad}
% common reference commands
\newcommand{\eqt}[1]{Eq.~(\ref{#1})}                     % equation
\newcommand{\fig}[1]{Fig.~\ref{#1}}                      % figure
\newcommand{\tbl}[1]{Table~\ref{#1}}                     % table

\newcommand{\ud}{\,\mathrm{d}}

\newcommand{\tcr}[1]{\textcolor{red}{#1}}
\newcommand{\tcb}[1]{\textcolor{blue}{#1}}
\newcommand{\tcg}[1]{\textcolor{green}{#1}}

\def\BibTeX{{\rm B\kern-.05em{\sc i\kern-.025em b}\kern-.08em
    T\kern-.1667em\lower.7ex\hbox{E}\kern-.125emX}}
%%%%%%%%%%%%%%%%%%%%%%%%%%%%%%%%%%%%%%%%%%%%%%%%%%%%%%%%%%%%%%%%%%%%

\begin{document}
\bibliographystyle{elsarticle-num}
\begin{center}
{ \Large Answers to Reviewer \#2}
\end{center}

\bigskip

\noindent Ms. Ref. No.: JOMP-D-15-00204\\
Title: ``Viscous Regularization for the Non-equilibrium Seven-Equation Two-Phase Flow Model', \\
{\it Springer Journal of Scientific Computing}\\

\bigskip

In {\color{blue}blue} are reviewer's comments, in black are our answers, in {\color{red}red} are Jean's comments,
and in {\color{green}green} are Marco's comments.

\bigskip

{\color{blue}
After the recent work by Guermond and Popov where a general class of viscous
regularizations of compressible Euler equations is investigated, the present work
proposes an extension in the case of the non-equilibrium seven-equation two-
phase flow model. To address such an issue, the authors introduce a general
viscosity within the adopted model. Then, they derive the entropy evolution
law, now perturbed by the additional viscosity. By adopting a relevant defini-
tion of the additional viscosity terms, the authors claim that a minimum entropy
principle is satisfied. This study is completed by a chapman-Enskog extension
to get the associated five-equation model in the limit of infinite relaxation co-
efficients. The analysis is achieved by considering the incompressible regime
governed by low Mach number. The authors claim that the adopted viscosity
regularization does not modify the required incompressible regime. Finally, nu-
merical illustrations are displayed in order to attempt to illustrate the relevance
of the viscosity regularizations.

My opinion about this work is not good at all since this paper looks like a poor
extension of the work by Guermond and Popov.}

We disagree with your assessment. \tcr{add all the things we learn thanks to reviewer \#1; 
extending to 2-phase flow is by far not a small extension because of everything
reviewer \# 1}
\bigskip




{\color{blue}
1. The main point of this work concerns the derivation of the minimum
entropy principle. Here, the establishment of this property is not clear
at all. I think that the proof is incomplete. For instance, Guermond
and Popov need (and prove) the positiveness of the density. I think that
the positiveness of partial density is here needed but no proof is given.
Moreover, I am convinced that $\alpha_k \in [0, 1]$ is also necessary and must be
proved. I urge the author to read carefully the paper by Guermond and
Popov and reconsider the establishment of their results.}

\tcr{we can cite Guermond for the phasic regularized continuity equation and show
than $\alpha \rho$ is non-negative. Then we can work on the alpha equations (both of them) to show
that  $\alpha_k \in [0, 1]$. From this, we conclude that $\rho_k$ is non-negative.} \tcg{Showing that $\alpha \rho$ is
non-negative is trivial based on Guermond's work. Showing, however, that $\rho$ is non-negative seems trickier.}
\bigskip


{\color{blue}
2. Several times, thew author speak about uniqueness of the numerical solution
. I don’t understand the meaning of these words. Moreover, this
paper does not contains numerical derivations. Page 7, the authors refer
to Leveque (pages 27-28 in Numerical Methods for Conservation Laws),
but these two pages in the leveque’s book coincides with the introduction
of weak solutions and entropy inequalities. Nothing about uniqueness of
the numerical solution.}

\tcr{Using Leveque for everything is a mistake. We will correct this using
piece of the answer we gave to reviewer \#1} \tcg{Once again, 'entropy solution' and
not 'uniqueness of the weak solution'.}
\bigskip


{\color{blue}
3. The numerical schemes, used to get the numerical illustrations, are not
specified. However, the derivation of a numerical scheme to approximate
the weak solution of the model under consideration is a very difficult task.}

\tcr{We will add this}
\bigskip


{\color{blue}
4. The presentation of the entropy residual is absolutely not relevant. In
section 3.2, I understand the opportunity to omit the underlined terms.
However, the equation (15) turns out to be wrong. The authors have to
introduce a specific notation to designate the entropy residual.}

\tcr{This is where the reviewer really shows he doesn't get it} \tcg{We can still add 
a specific notation for the entropy residual as he requested it.}
\bigskip


{\color{blue}
In addition, I think that these results are not suitable to be published in Journal
of Scientific Computing. As a consequence, I do not recommend the publication
of this work.}

\tcr{Pure BS}
\bigskip


\end{document}

