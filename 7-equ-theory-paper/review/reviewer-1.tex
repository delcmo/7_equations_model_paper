\documentclass{article}
%%%%%%%%%%%%%%%%%%%%%%%%%%%%%%%%%%%%%%%%%%%%%%%%%%%%%%%%%%%%%%%%%%%%%%%%%%%%%%%%%%%%%%%%%%%%%%%%%%%%%%%%%%%%%%%%%%%%%%%%%%%%%%%%%%%%%%%%%%%%%%%%%%%%%%%%%%%%%%%%%%%%%%%%%%%%%%%%%%%%%%%%%%%%%%%%%%%%%%%%%%%%%%%%%%%%%%%%%%%%%%%%%%%%%%%%%%%%%%%%%%%%%%%%%%%%
\usepackage{amsmath,amssymb}
% more math
\usepackage{amsfonts}
\usepackage{amssymb}
\usepackage{amstext}
\usepackage{amsbsy}
%\usepackage{dtklogos}

\usepackage{color}
\newcommand{\mt}[1]{\marginpar{\small #1}}
%%%%%%%%%%%%%%%%%%%%%%%%%%%%%%%%%%%%%%%%%%%%%%%%%%%%%%%%%%%%%%%%%%%%
% new commands
\newcommand{\nc}{\newcommand}
% operators
\renewcommand{\div}{\vec{\nabla}\! \cdot \!}
\newcommand{\grad}{\vec{\nabla}}
% latex shortcuts
\newcommand{\bea}{\begin{eqnarray}}
\newcommand{\eea}{\end{eqnarray}}
\newcommand{\be}{\begin{equation}}
\newcommand{\ee}{\end{equation}}
\newcommand{\bal}{\begin{align}}
\newcommand{\eali}{\end{align}}
\newcommand{\bi}{\begin{itemize}}
\newcommand{\ei}{\end{itemize}}
\newcommand{\ben}{\begin{enumerate}}
\newcommand{\een}{\end{enumerate}}
% DGFEM commands
\newcommand{\jmp}[1]{[\![#1]\!]}                     % jump
\newcommand{\mvl}[1]{\{\!\!\{#1\}\!\!\}}             % mean value
\newcommand{\keff}{\ensuremath{k_{\textit{eff}}}\xspace}
% shortcut for domain notation
\newcommand{\D}{\mathcal{D}}
% vector shortcuts
\newcommand{\vo}{\vec{\Omega}}
\newcommand{\vr}{\vec{r}}
\newcommand{\vn}{\vec{n}}
\newcommand{\vnk}{\vec{\mathbf{n}}}
\newcommand{\vj}{\vec{J}}
% extra space
\newcommand{\qq}{\quad\quad}
% common reference commands
\newcommand{\eqt}[1]{Eq.~(\ref{#1})}                     % equation
\newcommand{\fig}[1]{Fig.~\ref{#1}}                      % figure
\newcommand{\tbl}[1]{Table~\ref{#1}}                     % table

\newcommand{\ud}{\,\mathrm{d}}

\newcommand{\tcr}[1]{\textcolor{red}{#1}}
\newcommand{\tcb}[1]{\textcolor{blue}{#1}}
\newcommand{\tcg}[1]{\textcolor{green}{#1}}

\def\BibTeX{{\rm B\kern-.05em{\sc i\kern-.025em b}\kern-.08em
    T\kern-.1667em\lower.7ex\hbox{E}\kern-.125emX}}
%%%%%%%%%%%%%%%%%%%%%%%%%%%%%%%%%%%%%%%%%%%%%%%%%%%%%%%%%%%%%%%%%%%%
\bibliographystyle{../elsarticle-num}
\begin{document}
\begin{center}
{ \Large Answers to Reviewer \#1}
\end{center}

\bigskip

\noindent Ms. Ref. No.: JOMP-D-15-00204\\
Title: ``Viscous Regularization for the Non-equilibrium Seven-Equation Two-Phase Flow Model', \\
{\it Springer Journal of Scientific Computing}\\

\bigskip

%In {\color{blue}blue} are reviewer's comments, in black are our answers, in {\color{red}red} are Jean's comments,
%and in {\color{green}green} are Marco's comments.

%\fbox{\fbox{\parbox{0.9\textwidth}{\tcr{Line numbers: we will need to convert them the the Springer format before we resubmit. The current numbers are obtained using another format!}}}}
%
\bigskip

{\color{blue}
Viscous regularization of hyperbolic conservation laws is a meaningful approach to stabilize
the numerical discretization of such problems. In the present manuscript the authors derive
a viscous regularization for a two-phase model following previous work by Guermond and
Popov for a single-phase fluid. The basic idea is to design regularization terms such that
the entropy residuals (corresponding to the physical entropy and Harten's generalized
entropies) are positive. For scalar problems in conservative form it can be proven that this
ensures uniqueness of the weak solution. Therefore, positivity of the entropy residuals is
considered a useful property also for systems.

The subject can be considered very interesting and relevant. The presentation is clear
and the analysis is sound. However, some issues are not properly addressed and the
presentation of the numerical results needs to be improved before I can recommend this
paper for publication.}

Thank you for your very thorough review. 
\bigskip


{\color{blue}
1. The hyperbolic system (1) corresponding to the seven-equation two-phase ow model
is not in conservative form because of the gradients $\nabla \alpha_k$ in (1a), (1c) and (1d).
Therefore classical results from the theory of conservation laws cannot be directly
applied to the present system. Based on the DLM theory [2] the classical definitions
of a weak solution and the entropy solution, respectively, can be extended for quasi-
conservative problems introducing a path, see [3, 4, 2]. Existence of a solution to
a problem in non-conservative form has been investigated in [5]. In [6] existence of
a solution to the Riemann problem has been verified. In particular, shock waves
are influenced by the path since it enters the generalized Rankine-Hugoniot jump
conditions, see [4]. Note that rarefaction waves and contact discontinuities are not
affected by the choice of the path. It would be interesting for the reader to know
what notion of weak solution and entropy solution is applied here.}

The problem under consideration is indeed a nonlinear hyperbolic system in non-conservative form. 
We have added a paragraph reviewing previous work dealing with weak solution and entropy condition for 
non-conservative system of equations in the introduction (see the second and third paragraphs of the introduction). We also modified
the section 3.1 entitled {\it Methodology} accordingly.

We thank you for your insightful review. The following references have been added: \cite{alouges_merlet_2004}, \cite{lefloch_1989},
\cite{lefloch_1988}, \cite{dlm} and \cite{bianchini_bressan_2005}.
(numbers refer to the list of references given below, in our reply letter).

%\tcr{Marco and I think we have a pretty good handle on this. It will include
%citing the proper references and the relevant theorems. We must remember to thank the reviewer
%for pointing this out to us. This was invaluable and improved the quality of the manuscript.
%We believe that the reviewer is actually giving us the answer to his question. In a very
%polite way, he is guiding us towards the answer.}
%\tcg{References [3] and the paper by S. Bianchini et al. entitled "Vanishing viscosity Solutions of Nonlinear Hyperbolic Systems" contain the theorems related to weak and entropy solutions when stabilizing non-conservative hyperbolic system of equations with a vanishing viscosity numerical method.}
\bigskip


{\color{blue}
2. In the abstract, see page 1, line 20, the authors claim that regularization ensures
uniqueness of the weak solution. No evidence is given in the manuscript, i.e., neither
a reference nor a proof, that justifies this statement. For scalar problems in 
conservative form it is well-known that uniqueness is ensured if an entropy inequality
holds for all entropy-entropy flux pairs. However, the problem at hand is not in
conservative form. See also the first remark.}

Our wording was incorrect and we have corrected it: we meant to state that the viscous regularization 
ensures convergence of the numerical solution to a weak (minimum-entropy satisfying) solution 
and not uniqueness of the weak solution. See also our answer to your first remark.

%\tcr{Here too we will need to cite the proper theorems from the list of references giving to us.} \tcg{Also we should be more careful about the 
%wording and write 'entropy solution' instead of 'uniqueness of the weak solution'.}
\bigskip


{\color{blue}
3. In (3) the authors give a particular choice for the interfacial pressure and velocity.
It should be mentioned that other choices are considered in the literature for which
the second law of thermodynamics can be verified, see e.g. [7].}

The definitions of the interfacial pressure and velocity are not unique, as you rightfully mentioned. 
We added References 8 and 30 with further examples of the interfacial variables. We now emphasize 
that the theoretical results devised hold as long as the choice of the interfacial
variables preserve the entropy inequality (see paragraph between Equations (3d) and (4a)). 
%
%\tcg{We should also emphasize that the
%results of the paper are independent of the choice of the definition for the interfacial variables as long as the entropy inequality is preserved.}
\bigskip


{\color{blue}
4. On page 5, l. 34-37, it should be mentioned that hyperbolicity also requires a full
set of linearly independent (left/right) eigenvectors. For the considered system this
only holds true if the non-resonance condition is satisfied, see [1].
}

We fully agree with you and modified the text accordingly (see text below Eq. (7)).

%\tcr{Another great remark. We will add this}
\bigskip


{\color{blue}
5. In view of my first remark it would be interesting to know whether the field associated
to the eigenvalue $\lambda_{3+dim}$ in (6) is linearly degenerated. In this case the jump in $\alpha$
corresponds to a contact discontinuity.
}

We have looked into this. In older publications dealing with the 7-equation model, reasonable estimates for the interfacial velocity (e.g., $u_{int} = \frac{\alpha_g \rho_g u_g + \alpha_l \rho_l u_l}{\alpha_g \rho_g + \alpha_l \rho_l}$) have led to a proof that the field associated with $\lambda_{3+dim}=u_{int}$ is indeed linear degenerate ($\nabla \lambda_{3+dim} \cdot r__{3+dim} = 0$); 
see T. Gallouet and J-M Herard, ``Numerical modeling of two-phase flows under the two-fluid two-pressure approach'', {\it Math. Models and Meth. in Appl. Sci.}, Vol.14, No.5 (2004). 
With the more recent definitions of the interfacial velocity (i.e., $u_{int} = \frac{Z_g u_g + Z_l u_l}{Z_g + Z_l}$, with $Z_k = \rho_k c_k$), our derivations were inconclusive 
(although we think it is possible that this property still holds). 
%\framebox{\framebox{still noting here ...}}
%
%\tcr{We should verify this. We need to ask Ray for the eigenvectors of the SEM if he has them and them check that
%we have a contact discontinuity} \tcb{I have not added anything about that yet, but I believe it is done in a reference I have to find.}
\bigskip


{\color{blue}
6. For the numerical computations in Section 5.2 the stiffened gas law (32) has to be
modified to account for the heat of formation $q_k$.}

Thank you for pointing out this. It was a typo. We fully agree with you and have corrected it (the mistake also occurred at another location in the text).
\bigskip


{\color{blue}
7. For both computations presented in Section 5 the authors should provide more 
information on the discretization, e.g., the type of scheme (FVM/DG, explicit/implicit)
and its main ingredients. In particular, the discretization of the non-conservative
terms in (1) is of interest. Furthermore, the spatial/temporal dicretization and the
CFL number should be mentioned.}

We had forgotten to give a short paragraph explaining our space/time discretization.
Actually, it was present in a previous version of the manuscript and we removed it by mistake! 
It has  been added now (see paragraph at the beginning of Section 5).
All simulations are carried out using continuous finite elements.

\bigskip


{\color{blue}
8. For both test cases it would be interesting to investigate grid convergence because the
chosen regularization terms are grid-dependent and vanish with increasing resolution,
i.e., stabilization by viscous regularization is successively reduced. In particular, the
grid convergence study is recommended for the shock tube problem because the
numerical results show a severe smearing for both the contact wave and the shock
wave.}

%\framebox{\framebox{I hesitate a bit now that I re-read the comments. Are they asking for
%convergence order? We may not have run enough grid refinements yet}}

We performed a grid convergence for the first test and plotted the numerical solutions of the phasic density (Figure 2a) and
the phasic volume fraction (Figure 2b) for comparisons. It is then observed that the contact and shock waves are better resolved as 
the mesh is refined. 

%\tcr{Marco will run finer grid simulations for both problems} \tcg{ and plot all of the numerical solutions on a same
%figure in order to eyeball the increasing resolution effects.}
\bigskip


{\color{blue}
9. Since the authors only give the initial data for velocity, pressure and temperature,
the authors should also provide a caloric equation of state to compute the density,
i.e., $T_k$.}

Thank you for pointing out this. We have added it.
\bigskip


{\color{blue}
10. The authors verified positivity of the entropy residual. Therefore it would be 
interesting to see whether the numerical scheme maintains this property. Therefore I
would be interested to see plots of the numerical entropy residual to check whether
it remains positive.}

Positivity of the entropy residual is indeed ensured for a theoretical point of view. 
However, when numerically evaluating the entropy residual, negative values can be obtained in 
the vicinity of the shock. This phenomenon has been previously discussed in [23]. 
We added a remark explaining such a behavior in Remark-2, at the end of Section 3.5.

\bigskip


{\color{blue}
11. Besides the two computations presented here I would like to recommend further
numerical tests exhibiting stronger variations in the volume fractions and the density, 
e.g., a shock tube problem for a water-gas configuration with 
$(\rho, u P, \alpha)_{water} = (1000kg.m^3, 0m/s, 10^5Pa, 0.99)$ and 
$(\rho, u P, \alpha)_{gas}   = (0.2 kg.m^3, 0m/s, 2000Pa, 0.99)$,
i.e., the initial states are almost ``pure'' states of either water or gas, with material
parameters as given in [8].}

We have run such a case, with nearly pure phases. We used an even smaller value for the volume fraction; instead of 0.01, as you suggested, we used 0.0005. 
The results are provided using both 200 and 1000 mesh cells.
\bigskip


{\color{blue}
12. In the conclusion, see page 23, line 41/42 the authors claim that the regularization
ensures uniqueness of the numerical solution. Since the authors do not give a 
uniqueness proof this statement seems to be too strong and should be rephrased. See also
my first and second remark.}

We corrected this statement that is incorrect and rephrased it.

%\tcr{This will be corrected once we answer the reviewer's remarks \#1+2.}
\bigskip


\bibliography{../mybibfile}
\end{document}