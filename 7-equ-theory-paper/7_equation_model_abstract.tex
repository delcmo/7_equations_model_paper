In this paper, a viscous regularization is derived for the non-equilibrium Seven-Equation two-phase flow Model (SEM). 
This regularization, based on an entropy condition, is an artificial viscosity stabilization technique
that selects a weak solution satisfying an entropy-minimum principle.
 %based on an entropy condition and and , i.e. an entropy minimum principle, that selects a weak solution.
%%that selects the physical or entropy weak solution.
%\tcr{NEW PROPOSAL: that selects a weak solution that satisfies an entropy-minimum principle}
The viscous regularization ensures nonnegativity of the entropy residual,
% \tcr{check} \tcb{done} \tcr{but we cannot prove this numerically because it is not smooth ...} \tcb{ but we are not saying that the entropy residual is numerically positive. Guermond proves it in his later paper because he has a maximum principle. I do not think we should modify it.}
is consistent with the viscous regularization for Euler equations when one phase disappears, 
does not depend on the spatial discretization scheme chosen, 
and is compatible with the generalized Harten entropies. 
%
We investigate the behavior of the proposed viscous regularization for two important limit-cases. 
First, a Chapman-Enskog expansion is performed for the regularized seven-equation two-phase flow model and 
we show that the five-equation flow  model of Kapila is recovered with a well-scaled viscous regularization. 
Second, a low-Mach asymptotic limit of the regularized seven-equation flow model is carried out whereby the 
scaling of the non-dimensional numbers associated with the viscous terms is determined such that an 
incompressible two-phase flow model, with a properly scaled regularization, is recovered. 
%
Both limit-cases are illustrated with one-dimensional numerical results, including two-phase flow shock tube tests 
and steady-state two-phase flows in converging-diverging nozzles.
A continuous finite element discretization is employed for all numerical simulations.
