In this paper, a viscous regularization is derived for the non-equilibrium Seven-Equation two-phase flow Model (SEM). 
This regularization is derived by the mean of an entropy condition, i.e. an entropy minimum principle, that selects a weak solution.
%that selects the physical or entropy weak solution.
\tcr{what is 'the' physical solution? is it unique? is there only one entropy solution? I think we should be careful about the wording here} \tcb{we discussed it} 
\tcr{NEW PROPOSAL: that selects a weak solution that satisfies an entropy-minimum principle}\tcb{ I agree with it and modified it} 
The viscous regularization also ensures nonnegativity of the entropy residual, \tcr{check} \tcb{done} \tcr{but we cannot prove this numerically because it is not smooth ...} \tcb{ but we are not saying that the entropy residual is numerically positive. Guermond proves it in his later paper because he has a maximum principle. I do not think we should modify it.}
is consistent with the viscous regularization for Euler equations when one phase disappears, does not depend on the spatial discretization scheme chosen, 
and is compatible with the generalized Harten entropies. 
%
We also investigate the behavior of the proposed viscous regularization for two theoretical 
\tcr{couldn't we say important instead? Maybe ask Ray to pick.} \tcb{We can ask Ray.} limit-cases. 
First, a Chapman-Enskog expansion is performed for the regularized seven-equation two-phase flow model and 
we show that the five-equation flow  model of Kapila is recovered with a well-scaled viscous regularization. 
Second, a low-Mach asymptotic limit of the regularized seven-equation flow model is carried out whereby the 
scaling of the non-dimensional numbers associated with the viscous terms is determined such that an 
incompressible two-phase flow model, with a properly scaled regularization, is recovered. 
Both limit-cases are illustrated with one-dimensional numerical results, including two-phase flow shock tube tests 
and two-phase flow steady-state solutions in converging-diverging nozzles.