\documentclass[11pt]{letter}

\usepackage{color}
\newcommand{\tcr}[1]{\textcolor{red}{#1}}
\newcommand{\tcb}[1]{\textcolor{blue}{#1}}
\newcommand{\tcg}[1]{\textcolor{green}{#1}}

%%%%%% Letter Size Setup %%%%%%%%%%%%%%%%%%%%%%%%%%%%%%%%%%%%%%%%%%%%%%%%%%
       \addtolength{\textwidth}{2.cm}     %% For longer or shorter text width
       \addtolength{\topmargin}{-3.0cm}    %% For more or less top margin
       \addtolength{\textheight}{4cm}    %% For longer or shorter textheight
       \addtolength{\oddsidemargin}{-1.cm} %% For odd side margin (twoside)
                                            %% or margin (oneside)

\address{Dr. Jean Ragusa \\
Department of Nuclear Engineering,\\
Texas A\&M University, College Station, TX}
% \vspace{0.5cm}}

%%%%%% The Signature  and Date %%%%%%%%%%%%%%%%%%%%%%%%%%%%%%%%%%%%%%%%%%%%

\signature{\vspace{-1cm} Marc Delchini, Jean Ragusa, and Ray Berry}


\begin{document}

\begin{letter}{Professor Chi-Wang Shu, \\  Editor-in-Chief, Springer Journal of Scientific Computing,\\
\textbf{Re: JOMP-D-15-00204-R1}}


\date{\today}
%%%%%% More vertical space can be added here %%%%%%%%%%%%%%%%%%%%%%%%%%%%%%
%         \vspace{3.0cm}

\opening{Dear Professor Shu,}
         \vspace{0.25cm}
%%%%%% More vertical space can be added here %%%%%%%%%%%%%%%%%%%%%%%%%%%%%%

Thank you for the second round of reviews for our manuscript titled, 
``Viscous Regularization for the Non-equilibrium Seven-Equation Two-Phase Flow Model'',  
by M. Delchini, J. Ragusa, and R. Berry, submitted for publication in the {\it  Springer Journal of Scientific Computing}. 

Our main interest lies
in enabling two-phase flow simulations for the seven-equation (two-pressure, two-velocity) model using \underline{{\bf continuous}}
finite elements by employing an artificial viscosity technique based on entropy principles. To our knowledge, this has 
not been done to-date.

Comments from Reviewer \#1 (received as files \tcr{{\tt jsc\_2015\_2.pdf}} and \tcr{{\tt jsc\_2016\_1.pdf}}) 
were extremely insightful and allowed
us to fill in some gaps in the original version of the manuscript. 
For instance, this led us to add, in the first revision, several sections
dealing with the theory of hyperbolic non-conservative systems of equations (HNCSE) as they apply to the 
Seven-Equation two-phase flow Model (SEM), notably (i) one full additional page in the Introduction and 
(ii) a new section, Section 3.1, on how the HNCSE theory applies to the proposed viscous regularization of the SEM.
The additional bibliography included several articles (from Del Maso, Le Floch, Murat, Bianchini, Bressan, ...) 
Reviewer \#1 also noted some incorrect wording of some statements, which we appropriately revised. A difficult shock 
tube test case, with almost pure phases on each side of the membrane, was requested with a volume fraction of $0.01$. We
performed and added such a simulation, with an even lower volume fraction of $5\times 10^{-4}$. 
After our revision, Reviewer \#1 found that our revised manuscript was {\it significantly improved in quality}, appreciated our inclusion of
theoretical background on HNCSE, among others the proof of non-negative phase density and phasic volume.  On their
conclusions, they see our contribution as going beyond a simple extension of the work by Popov/Guermond for single-phase Euler equations and they {\bf strongly} recommend publication.

Comments from Reviewer \#2 were received as file \tcr{{\tt delchini\_ragusa\_berry.pdf}}
and as a email note from the Journal following our first revision. Reviews from the second Reviewer are short and terse,
and emphasize different topics between revisions.
\begin{itemize}
\item 
In Reviewer \#2's first round of comments, 
they stated that our work was ``not good and a poor extension of the work by Guermond and Popov'', which is in stark
contrast with Reviewer \#1. In our reply, we explained why we disagreed with this assessment (please see our first reply for details). 
\item We would like to point out that in their first review, they made an unsubstantiated  claim that our entropy equation for the 
regularized seven-equation model was incorrect. In fact, it is not and we have re-affirmed the correctness of our derivation in our reply to them. 
\item In their second review, they thank us for our {\it substantial revision which highlights their motivations} but are now generally ``septic'' (we believe they mean skeptical) about the use of an artificial viscosity method. 
\begin{enumerate}
\item We do not understand the concern or skepticism of this reviewer regarding a viscous regularization technique.
Most modern solvers for hyperbolic equation systems now use Godunov methods employing
Riemann or approximate Riemann solvers to capture the discontinuities, or shocks. However,
both the methods of artificial viscosity (either explicitly included by the addition of
dissipation terms or implicitly included through the inherent truncation error of the numerical
scheme used) and Godunov methods are general shock capturing methods. The effect
of {\bf either method is the introduction of an appropriate amount of entropy into the flow} 
(e.g., J. K. Dukowicz, ``A general, non-iterative Riemann solver for Godunov’s method,'' JCP, {\bf 61}, 119--137, 1985).
With the artificial viscosity methods, the entropy is added by the dissipation produced by
the incorporated artificial viscosity. On the other hand, with Godunov methods, the entropy
is primarily added implicitly by the presence of shock waves resulting from the Riemann
problem. Actually, at least in those cases when it can be found explicitly, the shock Hugoniot
curve (i.e., the shock pressure jump as a function of the shock velocity jump) closely
resembles commonly used, early forms of explicitly added artificial shock viscosity 
(e.g., M. L. Wilkins, ``Use of artificial viscosity in multi-dimensional fluid dynamic calculations,'' JCP, {\bf 36}, 281--303, 1980).\\
\item Regarding the reviewer questioning the real interest of this regularization: as explained 
in the manuscript (and aptly highlighted by Reviewer \#2),  the goal is to define a regularization 
that satisfies a minimum entropy principle. As such, one needs to verify that the chosen form of the viscous fluxes satisfies an entropy inequality. This goes back to their incorrect remark that our entropy inequality was wrong. 
\item Regarding the reviewer's disappointment that shock solutions are simulated but not studied: our goal is to stabilize a numerical solution to the seven-equation model when discretized with continuous finite elements. A study of physical admissibility, uniqueness of Riemann solutions under phase transition, etc. is outside of the scope of our paper and may not be a topic for a journal in scientific computing. 
% We have had numerous discussions with our colleagues Profs. Popov and Guermond regarding the status of theoretical knowledge pertaining to hyperbolic systems of conservations laws.  
\item Regarding the reviewer's comments dealing with the interaction of viscosity and numerical schemes: the suggested reference (Abgrall, Karni) ``de-conservatize'' the Euler equations to solve them in non-conservative forms with finite volume methods. The seven-equation model contains non-conser-vative products but is inherently closer to 2 sets of Euler equations (in conservative form) coupled to an advection equation for the volume fraction. We do not think it is a relevant comparison with ``de-conservatized'' Euler equations. 
\end{enumerate} 
\end{itemize}
%
%The second review from Reviewer \#2 also contained some objections not raised in their initial review. 
%We find this to be quite an objectionable process. 
Our proposed technique 
is based on a vanishing viscosity approach for the seven-equation two-phase flow model. 
There certainly remain open theoretical questions, which will hopefully be addressed by mathematicians with time. However, we believe to have grounded our approach in the best available techniques (e.g., the theoretical 
work of Bianchini/Bressan we cite in our paper regarding vanishing viscosity limits; the work by Popov/Guermond
regarding the definition of an adequate viscous regularization). Obtaining solutions of two-phase flows models using {\bf \underline{continuous} } finite elements has the potential for some impact since many open-source finite element libraries
are now available and have reached a certain level of maturity. Our goal was to show the feasibility of the approach with 
\begin{enumerate}
\item
a derivation of stabilization terms for two-phase flow in the spirit of Popov \& Guermond for Euler equation (these stabilization terms can be easily added in the FEM setting);
\item
a study of the effect of the viscosity terms in each equation;
\item 
a selection of numerical results (some of them taken from previous articles on the seven-equation model).
\end{enumerate}
 

%%%
\bigskip
Even though Reviewer \#1 was
satisfied with the revised manuscript and strongly recommended publication, our manuscript's status has been 
determined to still require a major revision. We are fervently
convinced that our manuscript proposes a novel approach for solving two-phase flow problem using {\bf continuous}
finite elements and strongly believe in the value of disseminating our findings/results. 
%Based on both rounds of comments from Reviewer \#2, we do not think we (and maybe any other researchers for that matter) 
%will ever be able to satisfy them: their comments are deeply rooted in the theoretical state of knowledge for HNCSE. 
%%Are we facing an uphill struggle with a community at large solving conservative law equations using finite volume 
%%techniques and approximate Riemann solvers? 

We kindly ask that you intercede regarding Review \#2.
We hope that the thoroughness of Reviewer \#1 and their
satisfaction with the current manuscript will outweigh the vague and generally negative opinion of Reviewer \#2.
%\vspace{0.25cm}


%%%%%% More vertical space can be added here %%%%%%%%%%%%%%%%%%%%%%%%%%%%%%

%%%%%%% The Closing %%%%%%%%%%%%%%%%%%%%%%%%%%%%%%%%%%%%%%%%%%%%%%%%%%%%%%%
\closing{Best regards, }

\end{letter}

\end{document}

