\documentclass[11pt]{letter}

\usepackage{color}
\newcommand{\tcr}[1]{\textcolor{red}{#1}}
\newcommand{\tcb}[1]{\textcolor{blue}{#1}}
\newcommand{\tcg}[1]{\textcolor{green}{#1}}

%%%%%% Letter Size Setup %%%%%%%%%%%%%%%%%%%%%%%%%%%%%%%%%%%%%%%%%%%%%%%%%%
       \addtolength{\textwidth}{2.cm}     %% For longer or shorter text width
       \addtolength{\topmargin}{-3.0cm}    %% For more or less top margin
       \addtolength{\textheight}{4cm}    %% For longer or shorter textheight
       \addtolength{\oddsidemargin}{-1.cm} %% For odd side margin (twoside)
                                            %% or margin (oneside)

\address{Dr. Jean Ragusa \\
Department of Nuclear Engineering,\\
Texas A\&M University, College Station, TX}
% \vspace{0.5cm}}

%%%%%% The Signature  and Date %%%%%%%%%%%%%%%%%%%%%%%%%%%%%%%%%%%%%%%%%%%%

\signature{\vspace{-1cm} Marc Delchini, Jean Ragusa, and Ray Berry}


\begin{document}

\begin{letter}{Professor Chi-Wang Shu, \\  Editor-in-Chief, Springer Journal of Scientific Computing,\\
\textbf{Re: JOMP-D-15-00204}}


\date{\today}
%%%%%% More vertical space can be added here %%%%%%%%%%%%%%%%%%%%%%%%%%%%%%
%         \vspace{3.0cm}

\opening{Dear Professor Shu,}
         \vspace{0.25cm}
%%%%%% More vertical space can be added here %%%%%%%%%%%%%%%%%%%%%%%%%%%%%%

Please find attached  a revised version of our manuscript entitled
``Viscous Regularization for the Non-equilibrium Seven-Equation Two-Phase Flow Model'',
by M. Delchini, J. Ragusa, and R. Berry for publication in the {\it  Springer Journal of Scientific Computing}. 

We have greatly appreciated the Reviews as we believe they have help us improve significantly the paper. 

Comments from Reviewer \#1 (received as file \tcr{{\tt jsc\_2015\_2.pdf}}) were extremely insightful and allowed
us to fill in some gaps in the original version of the manuscript. For instance, we have added several sections
dealing with the theory of hyperbolic non-conservative systems of equations (HNCSE) as they apply to the 
Seven-Equation two-phase flow Model (SEM), notably (i) one full additional page in the Introduction and 
(ii) a new section, Section 3.1, on how the HNCSE theory applies to the proposed viscous regularization of the SEM.
The additional bibliography includes articles by R. Abgrall, Del Maso, Le Floch, Murat, Bianchini, Bressan. You can find 
our full reply to this review in the file \tcr{{\tt reviewer-1.pdf}}.


We firmly disagree with comments from Reviewer \#2 (received as file \tcr{{\tt delchini\_ragusa\_berry.pdf}})
where they state that our work is {\it ``not good and is a poor extension of the work by Guermond and Popov''}.
%
Our work is {\bf not} a simple extension to the viscous regularization by Guermond and Popov for Euler equations. 
The seven-equation model is a two-pressure and two-velocity
two-phase flow model of the Baer and Nunziato type. These models are strictly hyperbolic 
non-conservative system of equations (HNCSE); hyperbolicity being an important property with respect to
causality principle. However, this type of system involves many difficulties 
for the derivation of numerical methods, often rendering the Riemann problem determination complex due 
to the 7 waves presents (in 1D, more in multi-D).  {\bf We have provided an alternative 
numerical method, based on an artificial viscosity principles,} and we believe this is a valuable alternative for the
solution of such hyperbolic systems.
As aptly pointed out by Reviewer \#1, the seven-equation two-phase flow model is an hyperbolic non-conservative 
system of equations: classical results from the theory of conservation laws cannot be directly
applied to the present system and one needs to invoke the Del Maso-Le Floch-Murat extensions of the 
classical definitions of a weak solution and the entropy solution. Existence of a solution to
a problem in non-conservative form also need additional tools. Another obvious demonstration that our
work is not simple extension of the work by Guermond and Popov is the fact that the volume fraction equation
also needs a  viscous stabilization term and we propose one such definition. 
%
The improvement we make here also allows a correct Mach number scaling for applications to all-speed flows and
the novel formulation allows for an easier extension to arbitrary equations of state.
In addition, their remark \#4 regarding the entropy residual is wrong. 
We have not, however, completely dismissed their review as some of their observations were relevant and have revised
our manuscript accordingly. You can find our full reply to this review in the file \tcr{{\tt reviewer-2.pdf}}.

We believe that we have adequately addressed all of the Reviewer's questions and hope that our paper will be found 
suitable for publication.

%In our revision, we have addressed the issues raised by the
%reviewers per their suggestions. 
%
%%%
\bigskip

We remain available for any further questions, should there be any.
%\vspace{0.25cm}


%%%%%% More vertical space can be added here %%%%%%%%%%%%%%%%%%%%%%%%%%%%%%

%%%%%%% The Closing %%%%%%%%%%%%%%%%%%%%%%%%%%%%%%%%%%%%%%%%%%%%%%%%%%%%%%%
\closing{Best regards, }

\end{letter}

\end{document}

