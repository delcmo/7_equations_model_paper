%%%%%%%%%%%%%%%%%%%%%%%%%%%%%%%%%%%%%%%%%%%%%%%%%%%
%
%  New template code for TAMU Theses and Dissertations starting Fall 2012.  
%  For more info about this template or the 
%  TAMU LaTeX User's Group, see http://www.howdy.me/.
%
%  Author: Wendy Lynn Turner 
%	 Version 1.0 
%  Last updated 8/5/2012
%
%%%%%%%%%%%%%%%%%%%%%%%%%%%%%%%%%%%%%%%%%%%%%%%%%%%

%%%%%%%%%%%%%%%%%%%%%%%%%%%%%%%%%%%%%%%%%%%%%%%%%%%%%%%%%%%%%%%%%%%%%%
%%                           APPENDIX A 
%%%%%%%%%%%%%%%%%%%%%%%%%%%%%%%%%%%%%%%%%%%%%%%%%%%%%%%%%%%%%%%%%%%%%

\phantomsection

\section{Entropy equation for the multi-dimensional seven equation model without viscous regularization}\label{app:sev-equ-model-entropy}
%
This appendix provides the steps that lead to the derivation of the phasic entropy equation of the Seven-Equation two-phase flow Model \cite{SEM}. For the purpose of this appendix, two phases are considered with no interphase mass or heat transfer and denoted by the indexes $j$ and $k$. In the Seven-Equation two-phase flow Model, each phase obeys to the following set of equations (\eqt{eq:sev_equ-app}):
\begin{subequations}
\label{eq:sev_equ-app}
%
\begin{align}
\partial_t \left( \alpha_k  A\right) + A \mbold u_{int} \cdot \grad \alpha_k = A \mu_P \left( P_k - P_j \right) 
\end{align}
%
\begin{align}
\partial_t \left( \alpha_k \rho_k A \right) + \div \left( \alpha_k \rho_k \mbold u_k A \right) = 0 
\end{align}
%
\begin{multline}
\partial_t \left( \alpha_k \rho_k \mbold u_k A \right) + \div \left[ \alpha_k A \left( \rho_k \mbold u_k \otimes \mbold u_k + P_k \mathbb{I} \right) \right] =\\
\alpha_k P_k \grad A + P_{int} A \grad \alpha_k + A \lambda_u \left( \mbold u_j - \mbold u_k \right) 
\end{multline}
%
\begin{multline}
\partial_t \left( \alpha_k \rho_k E_k A \right) + \div \left[ \alpha_k A \mbold u_k \left( \rho_k E_k + P_k \right) \right] = \\
P_{int} A \mbold u_{int} \cdot \grad \alpha_k - A\mu_P \bar{P}_{int} \left( P_k-P_j \right) + \bar{\mbold u}_{int} A \lambda_u \left( \mbold u_j - \mbold u_k \right)
\end{multline}
\end{subequations}
where $\rho_k$, $\mbold u_k$, $E_k$ and $P_k$ denote the density, velocity, specific total energy, and pressure of  phase $k$, respectively. $\mu_P$ and $\lambda_u$ and the pressure and velocity relaxation parameters, respectively. We recall that we assume that the cross section $A$ is only function of space: $\partial_t A = 0$ (a value of $A \neq 1$ is mostly of practical important for 1D nozzle problems). 
Variables with subscript ${int}$ correspond to the interfacial variables; their definitions are given in \eqt{eq:sev_equ2-app}. 
\begin{equation}
\label{eq:sev_equ2-app}
\left\{
\begin{array}{lll}
P_{int} = \bar{P}_{int} - \frac{\grad \alpha_k}{|| \grad \alpha_k ||} \frac{Z_k Z_j}{Z_k + Z_j} \left( \mbold u_k-\mbold u_j \right) \\
\bar{P}_{int} = \frac{Z_k P_j + Z_j P_k}{Z_k + Z_j} \\
\mbold u_{int} = \bar{\mbold u}_{int} - \frac{\grad \alpha_k}{|| \grad \alpha_k ||} \frac{P_k - P_j}{Z_k + Z_j} \\
\bar{\mbold u}_{int} = \frac{Z_k \mbold u _k + Z_j \mbold u_j}{Z_k + Z_j}
\end{array}
\right.
\end{equation}
where $Z_k = \rho_k c_k$ and $Z_j = \rho_j c_j$ are the impedances of phases $k$ and $j$, respectively. The speed of sound is denoted by the symbol $c$. The sign function $sgn(x)$ returns $\pm 1$ according to the sign of variable $x$.

The first step in proving the entropy minimum principle for \eqt{eq:sev_equ-app} 
consists of recasting these equations using the primitive variables $(\alpha_k, \rho_k, \mbold u_k, e_k)$, where $e_k$ is the specific internal energy of phase $k$. We introduce the material derivative $\frac{D (\cdot)}{Dt} = \partial_t (\cdot) + \mbold u_k \cdot \grad (\cdot)$ for simplicity. 

The continuity equation can be expressed as follows:
\begin{equation}
\label{eq:cont1-app}
\alpha_k A \frac{D \rho_k}{Dt} + \rho_k A \mu_P \left( P_k-P_j \right) + \rho_k A \left( \mbold u_k-\mbold u_{int} \right) \cdot \grad \alpha_k + \rho_k \alpha_k \div \left( A \mbold u_k \right) = 0 \,.
\end{equation}
The momentum and continuity equations are combined to yield an equation for the velocity:
\begin{equation}
\label{eq:vel1-app}
\alpha_k \rho_k A \frac{D\mbold u_k}{Dt} + \grad \left( \alpha_k A P_k \right) = \alpha_k P_k \grad A + P_{int} A \grad \alpha_k + A \lambda_u \left( \mbold u_j-\mbold u_k \right) \,.
\end{equation}
A kinetic energy equation is obtained by taking the vector scalar product of the previous result with $\mbold u_k$ to yield:
\begin{multline}
\label{eq:kin1-app}
\alpha_k \rho_k A \frac{D\left(\mbold u_k^2/2\right)}{Dt} + \mbold u_k \grad \left( \alpha_k A P_k \right) = \\ \mbold u_k  \Big( \alpha_k P_k \grad A + P_{int} A \grad \alpha_k + A \lambda_u \left( \mbold u_j-\mbold u_k \right) \Big) \,.
\end{multline}
%
The internal energy equation is obtained by subtracting the above kinetic energy equation from the total energy equation:
\begin{multline}\label{eq:internal1}
\alpha_k \rho_k A \frac{D e_k}{Dt} + \alpha_k P_k \div \left(A \mbold u_k \right) = 
 P_{int} A \left(\mbold u_{int}-\mbold u_k \right) \cdot \grad \alpha_k \\
 - \bar{P}_{int} A \mu_P \left(P_k-P_j \right) + A \lambda_u \left(\mbold u_j-\mbold u_k  \right) \cdot \left(\bar{\mbold u}_{int}-\mbold u_k \right) \,.
\end{multline}

In the next step, we assume the existence of a phasic entropy $s_k$ that is function of the density $\rho_k$ and the  internal energy $e_k$. Using the chain rule, 
\begin{equation}
\frac{Ds_k}{Dt} = (s_\rho)_k \frac{D \rho_k}{Dt} + (s_e)_k \frac{De_k}{Dt},
\end{equation}
we combine the density and internal energy equations ($\rho_k (s_\rho)_k \times \eqt{eq:cont1-app}  + (s_e)_k \times \eqt{eq:internal1}$) to obtain  the following entropy equation :
\begin{multline}
\label{eq:ent1}
\alpha_k \rho_k A \frac{Ds_k}{Dt} + 
\underbrace{\alpha_k \left( P_k (s_e)_k + \rho_k^2 (s_\rho)_k \right)  \div \left( A \mbold u_k \right) }_\textrm{(a)} = \\
(s_e)_k A \left[ P_{int}(\mbold u_{int}-\mbold u_k)\cdot \grad \alpha_k - \bar{P}_{int} A \mu_P (P_k-P_j) + A \lambda_u (\bar{\mbold u}_{int}-\mbold u_k) \cdot (\mbold u_j-\mbold u_k)\right] \\
- \rho_k^2 (s_\rho)_k \left[ \mu_P A (P_k-P_j) + A(\mbold u_k-\mbold u_{int}) \cdot \grad \alpha_k\right] 
\end{multline}
where $(s_e)_k$ and $(s_\rho)_k$ denote the partial derivatives of entropy $s_k$ with respect to the internal energy $e_k$ and the density $\rho_k$, respectively.
The term denoted by (a) on the left-hand side of \eqt{eq:ent1} can be set to zero by invoking the Gibbs relation from the second law of thermodynamics:
\begin{equation}
T_k ds_k = de_k - \frac{P_k}{\rho_k^2} d \rho_k \text{ with } (s_e)_k = \frac{1}{T_k} \text{ and } (s_\rho)_k = - \frac{P_k}{\rho_k^2} (s_e)_k
\end{equation}
which yields
\begin{equation}
\label{eq:ent2}
 P_k (s_e)_k + \rho_k^2 (s_\rho)_k = 0 .
\end{equation} 
% The above equation is equivalent to the application of the second law of thermodynamic law when assuming reversibility:

Finally, \eqt{eq:ent1} is as follows:
\begin{eqnarray}
\label{eq:ent3}
((s_e)_k)^{-1} \alpha_k \rho_k \frac{Ds_k}{Dt} = \underbrace{\left[ P_{int} (\mbold u_{int}-\mbold u_k) + P_k (\mbold u_k-\mbold u_{int}) \right] \cdot \grad \alpha_k}_\textrm{(b)} + \nonumber\\ 
\underbrace{\mu_P (P_k-P_j)(P_k-\bar{P}_{int})}_\textrm{(c)} + \underbrace{\lambda_u(\mbold u_j-\mbold u_k)\cdot(\bar{\mbold u}_{int}-\mbold u_k)}_\textrm{(d)}
\end{eqnarray}
The right-hand side of \eqt{eq:ent3} has been split into three terms, (b), (c), and (d); next we analyze each of these terms separately. The terms (c) and (d) can be easily recast by using the definitions of $\bar{\mbold u}_{int}$ and $\bar{P}_{int}$ given in \eqt{eq:sev_equ2-app}:
\begin{eqnarray}
\label{eq:ent4}
\mu_P (P_k-P_j)(P_k-\bar{P}_{int}) = \mu_P \frac{Z_k}{Z_k+Z_j} (P_j - P_k)^2\nonumber\\
\lambda_u(\mbold u_j-\mbold u_k)\cdot(\bar{\mbold u}_{int}-\mbold u_k) = \lambda_u \frac{Z_j}{Z_k+Z_j} (\mbold u_j - \mbold u_k)^2 
\end{eqnarray}
By definition, $\mu_P$, $\lambda_u$ and $Z_k$ are all positive. Thus, the above terms (c) and (d) are unconditionally positive. 

We now inspect term (b). Once again, we use the definitions of $P_{int}$ and $\mbold u_{int}$ and the following relations:
\begin{eqnarray}
\label{eq:ent4bis}
\mbold u_{int}-\mbold u_k &=& \frac{Z_j}{Z_k+Z_j}(\mbold u_j-\mbold u_k) -  \frac{\grad \alpha_k}{\| \grad \alpha_k \|} \frac{Pk-P_j}{Z_k+Z_j} \nonumber\\
P_{int}-P_k &=& \frac{Z_k}{Z_k+Z_j} (P_j-P_k) - \frac{\grad \alpha_k}{\| \grad \alpha_k \|} \frac{Z_k Z_j}{Z_k+Z_j} (\mbold u_k-\mbold u_j), \nonumber 
\end{eqnarray}
Then, term (b) becomes:
\begin{multline}
\label{eq:ent5}
\left[ P_{int} (\mbold u_{int}-\mbold u_k) + P_k (\mbold u_k-\mbold u_{int}) \right] \cdot \grad \alpha_k = (P_{int}-P_k)(\mbold u_{int}-\mbold u_k)\cdot \grad \alpha_k=   \\ 
\frac{Z_k}{\left( Z_k+Z_j \right)^2} \grad \alpha_k \cdot \left[ Z_j (\mbold u_j-\mbold u_k)(P_j-P_k)+\frac{\grad \alpha_k}{\| \grad \alpha_k \|} Z_j^2 (\mbold u_j-\mbold u_k)^2 \right. + \\ 
\left. \frac{\grad \alpha_k}{\| \grad \alpha_k \|}(P_k-P_j)^2 +  \frac{\grad \alpha_k \cdot \grad \alpha_k}{\| \grad \alpha_k \|^2}(P_k-P_j)Z_j (\mbold u_k-\mbold u_j) \right] 
\end{multline}
The above equation is factorized by $\|  \grad \alpha_k \|$ and then recast under a quadratic form using $\frac{\grad \alpha_k \cdot \grad \alpha_k}{\| \grad \alpha_k \|^2} = 1$. This yields:
\begin{align}
\label{eq:ent6}
\left[ (\mbold u_{int}-\mbold u_k)P_{int} + (\mbold u_k-\mbold u_{int})P_k \right] \grad \alpha_k &=  \nonumber \\
\| \grad \alpha_k \| \frac{Z_k }{\left( Z_k+Z_j \right)^2} \left[ Z_j (\mbold u_j-\mbold u_k) + \frac{\grad \alpha_k}{\| \grad \alpha_k \|}(P_k-P_j)\right]^2
\end{align}
Thus, using \eqt{eq:ent3}, \eqt{eq:ent4}, \eqt{eq:ent5} and \eqt{eq:ent6}, the entropy equation obtained in \cite{SEM} holds and is recalled here for convenience:
\begin{align}
(s_{e})_k^{-1} \alpha_k \rho_k A \frac{Ds_k}{Dt} &= \mu_P \frac{Z_k}{Z_k+Z_j} (P_j - P_k)^2 + \lambda_u \frac{Z_j}{Z_k+Z_j} (\mbold u_j -\mbold  u_k)^2 \nonumber
\\
& \| \grad \alpha_k \| \frac{Z_k}{\left( Z_k+Z_j \right)^2} \left[ Z_j (\mbold u_j-\mbold u_k)+\frac{\grad \alpha_k}{\| \grad \alpha_k \|}(P_k-P_j)\right]^2. \nonumber
\end{align}
%
\pagebreak{}