%%
%% This is file `elsarticle-template-num.tex',
%% generated with the docstrip utility.
%%
%% The original source files were:
%%
%% elsarticle.dtx  (with options: `numtemplate')
%% 
%% Copyright 2007, 2008 Elsevier Ltd.
%% 
%% This file is part of the 'Elsarticle Bundle'.
%% -------------------------------------------
%% 
%% It may be distributed under the conditions of the LaTeX Project Public
%% License, either version 1.2 of this license or (at your option) any
%% later version.  The latest version of this license is in
%%    http://www.latex-project.org/lppl.txt
%% and version 1.2 or later is part of all distributions of LaTeX
%% version 1999/12/01 or later.
%% 
%% The list of all files belonging to the 'Elsarticle Bundle' is
%% given in the file `manifest.txt'.
%% 

%% Template article for Elsevier's document class `elsarticle'
%% with numbered style bibliographic references
%% SP 2008/03/01

%\documentclass[preprint,12pt]{elsarticle}
\documentclass[preprint,10pt]{elsarticle}
%\documentclass[final,3p,times]{elsarticle} 

%% Use the option review to obtain double line spacing
%% \documentclass[authoryear,preprint,review,12pt]{elsarticle}

%% Use the options 1p,twocolumn; 3p; 3p,twocolumn; 5p; or 5p,twocolumn
%% for a journal layout:
%% \documentclass[final,1p,times]{elsarticle}
%% \documentclass[final,1p,times,twocolumn]{elsarticle}
%% \documentclass[final,3p,times]{elsarticle}
%% \documentclass[final,3p,times,twocolumn]{elsarticle}
%% \documentclass[final,5p,times]{elsarticle}
%% \documentclass[final,5p,times,twocolumn]{elsarticle}

%% if you use PostScript figures in your article
%% use the graphics package for simple commands
\usepackage{float}
\usepackage{color}
\usepackage{caption}
\usepackage{subcaption}
\usepackage{appendix}
%% or use the graphicx package for more complicated commands
\usepackage{graphicx}
%% or use the epsfig package if you prefer to use the old commands
%% \usepackage{epsfig}

%% The amssymb package provides various useful mathematical symbols 
%% The amsthm package provides extended theorem environments
\usepackage{amssymb}
\usepackage{amsmath}
% more math
\usepackage{amsfonts}
\usepackage{amstext}
\usepackage{amsbsy}
\usepackage{mathbbol} 
%% The lineno packages adds line numbers. Start line numbering with
%% \begin{linenumbers}, end it with \end{linenumbers}. Or switch it on
%% for the whole article with \linenumbers.
\usepackage{lineno}

\journal{Journal of Comp. Phys.}
%%%%%%%%%%%%%%%%%%%%%%%%%%%%%%%%%%%%%%%%%%%%%%%%%%%%%%%%%%%%%%%%%%%%
% operators
\renewcommand{\div}{\mbold{\nabla}\! \cdot \!}
\newcommand{\grad}{\mbold{\nabla}}
\newcommand{\divv}[1]{\boldsymbol{\nabla}^{#1}\! \cdot \!}
\newcommand{\gradd}[1]{\vec{\nabla}^{#1}}
\newcommand{\mbold}[1]{\boldsymbol#1}
% latex shortcuts
\newcommand{\bea}{\begin{eqnarray}}
\newcommand{\eea}{\end{eqnarray}}
\newcommand{\be}{\begin{equation}}
\newcommand{\ee}{\end{equation}}
\newcommand{\bal}{\begin{align}}
\newcommand{\eali}{\end{align}}
\newcommand{\bi}{\begin{itemize}}
\newcommand{\ei}{\end{itemize}}
\newcommand{\ben}{\begin{enumerate}}
\newcommand{\een}{\end{enumerate}}
% DGFEM commands
\newcommand{\jmp}[1]{[\![#1]\!]}                     % jump
\newcommand{\mvl}[1]{\{\!\!\{#1\}\!\!\}}             % mean value
\newcommand{\keff}{\ensuremath{k_{\textit{eff}}}\xspace}
% shortcut for domain notation
\newcommand{\D}{\mathcal{D}}
% vector shortcuts
\newcommand{\vo}{\vec{\Omega}}
\newcommand{\vr}{\vec{r}}
\newcommand{\vn}{\vec{n}}
\newcommand{\vnk}{\vec{\mathbf{n}}}
\newcommand{\vj}{\vec{J}}
\newcommand{\eig}[1]{\| #1 \|_2}
%
\newcommand{\EI}{\mathcal{E}_h^i}
\newcommand{\ED}{\mathcal{E}_h^{\partial \D^d}}
\newcommand{\EN}{\mathcal{E}_h^{\partial \D^n}}
\newcommand{\ER}{\mathcal{E}_h^{\partial \D^r}}
\newcommand{\reg}{\textit{reg}}
%
\newcommand{\norm}{\textrm{norm}}
\renewcommand{\Re}{\textrm{Re}}
\newcommand{\Pe}{\textrm{P\'e}}
\renewcommand{\Pr}{\textrm{Pr}}
%
\newcommand{\resi}{R}
%\newcommand{\resinew}{\tilde{D}_e}
\newcommand{\resinew}{\widetilde{\resi}}
\newcommand{\resisource}{\widetilde{\resi}^{source}}
\newcommand{\matder}[1]{\frac{\textrm{D} #1}{\textrm{D} t}}
%
% extra space
\newcommand{\qq}{\quad\quad}
% common reference commands
\newcommand{\eqt}[1]{Eq.~(\ref{#1})}                     % equation
\newcommand{\fig}[1]{Fig.~\ref{#1}}                      % figure
\newcommand{\tbl}[1]{Table~\ref{#1}}                     % table
\newcommand{\sct}[1]{Section~\ref{#1}}                   % section
\newcommand{\app}[1]{Appendix~\ref{#1}}                   % appendix
%
\newcommand{\ie}{i.e.,\@\xspace}
\newcommand{\eg}{e.g.,\@\xspace}
\newcommand{\psc}[1]{{\sc {#1}}}
\newcommand{\rs}{\psc{R7}\xspace}
%
\newcommand\br{\mathbf{r}}
%\newcommand{\tf}{\varphi}
\newcommand{\tf}{b}
%
\newcommand{\tcr}[1]{\textcolor{red}{#1}}
\newcommand{\tcb}[1]{\textcolor{blue}{#1}}
\newcommand{\mt}[1]{\marginpar{ {\tiny #1}}}
%
\bibliographystyle{elsarticle-num}
%%%%%%%%%%%%%%%%%%%%%%%%%%%%%%%%%%%%%%%%%%%%%%%%%%%%%%%%%%%%%%%%%%%%%
%
%   BEGIN DOCUMENT
%
%%%%%%%%%%%%%%%%%%%%%%%%%%%%%%%%%%%%%%%%%%%%%%%%%%%%%%%%%%%%%%%%%%%%%
\begin{document}

%%%%%%%%%%%%%%%%%%%%%%%%%%%%%%%%%%%%%%%%%%%%%%%%%%%%%%%%%%%%%%%%%%%%
\begin{frontmatter}

%% Title, authors and addresses

%% use the tnoteref command within \title for footnotes;
%% use the tnotetext command for theassociated footnote;
%% use the fnref command within \author or \address for footnotes;
%% use the fntext command for theassociated footnote;
%% use the corref command within \author for corresponding author footnotes;
%% use the cortext command for theassociated footnote;
%% use the ead command for the email address,
%% and the form \ead[url] for the home page:
%\title{Title\tnoteref{label1}}
%% \tnotetext[label1]{}
%% \author{Name\corref{cor1}\fnref{label2}}
%% \ead{email address}
%% \ead[url]{home page}
%% \fntext[label2]{}
%% \cortext[cor1]{}
%% \address{Address\fnref{label3}}
%% \fntext[label3]{}
%-------------------------
%-------------------------
\title{Extension of the entropy viscosity method to the multi-D 7-equation two-phase flow model.\\
\tcb{I do not know if we should have 'multi-D' in the title since we will only present $1$-D results}}
%-------------------------
%-------------------------
\author{Marc O. Delchini\fnref{label1}}
\ead{delchmo@tamu.edu}

\author{Jean C. Ragusa\corref{cor1}\fnref{label1}}
\ead{jean.ragusa@tamu.edu}

\author{Ray A. Berry\fnref{label2}}
\ead{ray.berry@inl.gov}

\address[label1]{Department of Nuclear Engineering, Texas A\&M University, College Station, TX 77843, USA \fnref{label1}}

\address[label2]{Idaho National Laboratory, Idaho Falls, ID 83415, USA \fnref{label2}}

\cortext[cor1]{Corresponding author}
%-------------------------
%-------------------------
%-------------------------
\begin{abstract}
blabla
\end{abstract}
%-------------------------
%-------------------------
\begin{keyword}
  two-phase flow model \sep with variable area \sep entropy viscosity method \sep stabilization method \sep low Mach regime \sep shocks.
\end{keyword}
%-------------------------
\end{frontmatter}
\linenumbers
%%%%%%%%%%%%%%%%%%%%%%%%%%%%%%%%%%%%%%%%%%%%%%%%%%%%%%%%%%%%%%%%%%%%%%%%%%%%%
\section{Introduction}\label{sec:intro}
%%%%%%%%%%%%%%%%%%%%%%%%%%%%%%%%%%%%%%%%%%%%%%%%%%%%%%%%%%%%%%%%%%%%%%%%%%%%%
\begin{itemize}
\item a few lines about the need for accurately resolving two-phase flows
\item background on the different two-phase flow models: 5, 6 and 7-equation two-phase flow models
\item then, focus on the different types of 7-equation two-phase flow models: they mostly differ because of the closure relaxations used
\item discuss the different numerical solvers developed for the 7-equation two-phase flow model: HLL, HLLC, and approximated Riemann solvers accounting for the source terms
\item emphasize the fact that the above numerical solvers only works on discontinuous schemes
\item then, introduce the entropy viscosity method and details the organization of the paper 
\end{itemize}
%
Compressible two-phase flows are found in numerous industrial applications and are an ongoing area of research in modeling and simulation over many years. A variety of models with different levels of complexity has been developed such as: five-equation model \cite{Kapila_2001}, six-equation model \cite{Toumi_1996}, and more recently the seven-equation model \cite{SEM}. These models are all obtained by integrating the single-phase flow balance equations weighed by a characteristic or indicator function for each phase. The resulting system of equations contains non-conservative terms that describe the interaction between phases but also an equation for the volume fraction. Once a system of equations describing the physics is derived, the next challenging step is to develop a robust and accurate discretization to obtain a numerical solution. Assuming that the system of equations is hyperbolic under some conditions, a Riemann solver could be used but is often ruled out because of the complexity due to the number of equations involved. Furthermore, careless approximation for the treatment of the non-conservative terms can lead to failure in computing the numerical solution \cite{Abgrall_2002}. An alternative is to use an approximate Riemann solver, a well-established approach for single-phase flows, while deriving a consistent discretization scheme for the non-conservative terms. 

This methodology was applied to the seven-equation model (SEM) introduced by Berry et al. in \cite{SEM}. This model is known to be unconditionally hyperbolic which is highly desirable when working with approximate Riemann solvers and can treat a wide range of applications. Its particularity comes from the pressure and velocity relaxation terms in the volume fraction, momentum and energy equations that can bring the two phases in equilibrium when using large values of the relaxation parameters. In other words, the seven-equation model can degenerate into the six- and five-equation models. Alike for the other two-phase flow models, solving for the seven-equation model requires a numerical solver and significant effort was dedicated to this task for spatially discontinuous schemes. Because each phase is assumed to obey the Euler equations, most of the numerical solvers are adapted from the single-phase approximate Riemann solvers. For example, Saurel et al. \cite{Saurel_2001a, Saurel_2001b} employed a HLL-type scheme to solve for the SEM but noted that excessive dissipation was added to the contact discontinuity. A more advanced HLLC-type scheme was developed in \cite{Li_2004} but only for the subsonic case and then extended to supersonic flows in \cite{Zein_2010}. More recently, Ambroso et al. \cite{Ambroso_2012} proposed an approximate Riemann solver accounting for source terms such as gravity and drag forces, but with no interphase mass transfer.
%%%%%%%%%%%%%%%%%%%%%%%%%%%%%%%%%%%%%%%%%%%%%%%%%%%%%%%%%%%%%%%%%%%%%%%%%%%%%
%%%%%%%%%%%%%%%%%%%%%%%%%%%%%%%%%%%%%%%%%%%%%%%%%%%%%%%%%%%%%%%%%%%%%%%%%%%%%
\section{The multi-D 7-equation two-phase flow model}\label{sec:7-equ-model}
%%%%%%%%%%%%%%%%%%%%%%%%%%%%%%%%%%%%%%%%%%%%%%%%%%%%%%%%%%%%%%%%%%%%%%%%%%%%%
%%%%%%%%%%%%%%%%%%%%%%%%%%%%%%%%%%%%%%%%%%%%%%%%%%%%%%%%%%%%%%%%%%%%%%%%%%%%%
%\begin{itemize}
%\item give the equations and detail the different terms
%\item include the relaxation terms, the mass and heat exchange terms
%\item eigenvalues
%\item entropy equation WITHOUT the dissipative terms and five the details of the derivation in the appendix
%\end{itemize}
%
The multi-D seven-equation two-phase model presented in this paper is obtained by assuming that each phase obeys the single-phase Euler equations (with phase-exchange terms) and by integrating over a control volume after multiplying by a characteristic function. The detailed derivation can be found in \cite{SEM}. In this section, the governing multi-dimensional equations are recalled for a phase $k$ in interaction with a phase $j$. Each phase obeys the following mass, momentum and energy balance equations, supplemented by a non-conservative volume-fraction equation:
%
\begin{subequations}\label{eq:liq-7-eqn-sect5}
\begin{align}
  % liquid mass conservation
  \label{multi-d-7-equ-liq}
  \frac{\partial \left( \alpha \rho \right)_{k} A}{\partial t}
  + \div \left( \alpha \rho \mbold u A\right)_{k}
  &= - \Gamma A_{int} A
\end{align}
\begin{align}
  % liquid momentum
  \frac{\partial \left( \alpha \rho \mbold u \right)_{k} A}{\partial t}
  + \div \left[ \alpha_{k} A \left( \rho \mbold u \otimes \mbold u + P \mathbb{I} \right)_{k} \right]
  &= P_{int} A \grad \alpha_{k} + P_{k} \alpha_{k} \grad A
    \nonumber
  \\
  &+ A \lambda_u (\mbold u_{j} - \mbold u_{k})
  - \Gamma A_{int} \mbold u_{int} A
\end{align}
\begin{align}
  % liquid total energy
  \frac{\partial \left( \alpha \rho E \right)_{k} A}{\partial t}
  + \div \left[ \alpha_{k} \mbold u_{k} A \left( \rho E + P \right)_{k} \right]
  &= P_{int} \mbold u_{int} A \grad \alpha_{k} - \bar{P}_{int} A \mu_P (P_{k} - P_{j})
        \nonumber
  \\
  + \bar{\mbold u}_{int} A \lambda_u (\mbold u_{j} - \mbold u_{k})
&  + \Gamma A_{int} \left( \frac{P_{int}}{\rho_{int}} - H_{k, int} \right) A
%\nonumber 
%\\
%& + Q_{wall,k} + Q_{int,k}
\end{align}
\begin{align}
  % liquid volume fraction
  \label{eqn:multi-d-7-eqn-liq-vol}
  \frac{\partial \alpha_{k} A}{\partial t} + A\mbold u_{int} \cdot \grad \alpha_{k}
  &= A \mu_P (P_{k} - P_{j}) - \frac{\Gamma A_{int} A}{\rho_{int}}
\end{align}
\end{subequations}
%
%On the same model, the equations for the vapor phase are:
%%
%\begin{subequations}\label{eq:vap-7-eqn-sect5}
%\begin{align}
%  \label{multi-d-7-equ-vap}
%  % vapor mass conservation
%  \frac{\partial \left( \alpha \rho A\right)_{vap}}{\partial t}
%  + \div \left( \alpha \rho \mbold u \right)_{vap} A
%  =  \Gamma A_{int} A
%\end{align}
%\begin{align}
%  % vapor momentum
%  \frac{\partial \left( \alpha \rho u \right)_{vap} A}{\partial t}
%  + \div \left[ \alpha_{vap} A \left( \rho \mbold u \otimes \mbold u + P\mathbb{I} \right)_{vap} \right]
%  &= P_{int} A \grad \alpha_{vap} + P_{vap} \alpha_{vap} \grad A
%  \\
%  \nonumber
%  &+ A \lambda_u (\mbold u_{liq} - \mbold u_{vap})
%  + \Gamma A_{int} u_{int} A
%\end{align}
%\begin{align}
%  % vapor total energy
%  \frac{\partial \left( \alpha \rho E \right)_{vap} A}{\partial t}
%  + \div \left[ \alpha_{vap} \mbold u_{vap} A \left( \rho E + P \right)_{vap} \right]
%  &= P_{int} \mbold u_{int} A \grad \alpha_{vap} - \bar{P}_{int} A \mu_P (P_{vap} - P_{liq})
%    \nonumber
%  \\
%  + \bar{\mbold u}_{int} A \lambda_u (\mbold u_{liq} - \mbold u_{vap})
%&- \Gamma A_{int} \left( \frac{P_{int}}{\rho_{int}} - H_{vap, int} \right) A
%\nonumber 
%\\
%& + Q_{wall,vap} + Q_{int,vap}
%\end{align}
%\begin{align}
%  % vapor phase volume fraction
%  \label{eqn:multi-d-7-eqn-vap-vol}
%  \frac{\partial \alpha_{vap} A}{\partial t} + A \mbold u_{int} \cdot \grad \alpha_{vap}
%  &= A \mu_P (P_{vap} - P_{liq}) + \frac{\Gamma A_{int} A}{\rho_{int}}
%\end{align}
%\end{subequations}
%
where $\alpha_k$, $\rho_k$, $\mbold u_k$ and $E_k$ denote the volume fraction, the density, the velocity vector and the total specific energy of phase $k$, respectively. The phase pressure $P_k$ is computed from an equation of state. The interfacial pressure and velocity and their corresponding average values are denoted by $P_{int}$, $\mbold u_{int}$, $\bar{P}_{int}$ and $\bar{\mbold u}_{int}$, respectively, and are defined in \eqt{eq:int_variables_def}. 
%
\begin{subequations}
\label{eq:int_variables_def}
\begin{align}
  \label{E-R:83}
  P_{int} &= \bar{P}_{int} + \frac{Z_{k}Z_{j}}{Z_{k}+Z_{j}} \frac{\grad \alpha_{k}}{|| \grad \alpha_{k} ||} \cdot (\mbold u_{j}-\mbold u_{k})
  \\
  \bar{P}_{int} &= \frac{Z_{j}P_{k}+Z_{k}P_{j}}{Z_{k}+Z_{j}}
 \\
  \label{E-R:84}
  \mbold u_{int} &= \bar{\mbold u}_{int} +  \frac{\grad \alpha_{k}}{|| \grad \alpha_{k} ||} \frac{P_{j}-P_{k}}{Z_{k}+Z_{j}}
  \\
  \bar{\mbold u}_{int} &= \frac{Z_{k} \mbold u_{k}+Z_{j}\mbold u_{j}}{Z_{k}+Z_{j}}.
\end{align}
\end{subequations}
%
The interfacial specific total enthalpy of phase $k$, $H_{k}$, is defined as follows: $H_k = h_k + 0.5 || \mbold u ||^2$. Following \cite{SEM}, the pressure and velocity relaxation coefficients, $\mu_P$  and $\lambda_u$ respectively, are function of the acoustic impedance $Z_k = \rho_k c_k$ and the specific interfacial area $A_{int}$ as shown in \eqt{eq:relaxation_coeff}.
%
\begin{subequations}
\label{eq:relaxation_coeff}
\begin{align}
  \label{E-R:85}
  \lambda_u &= \frac{1}{2} \mu_P Z_{k} Z_{j}
  \\
  \label{E-R:86}
  \mu_P &= \frac{A_{int}}{Z_{k}+Z_{j}}
\end{align}
\end{subequations}
%
The specific interfacial area (i.e., the interfacial surface area per unit
volume of two-phase mixture), $A_{int}$, must be specified from some type of
flow regime map or function under the form of a correlation. In \cite{SEM}, $A_{int}$ is chosen to be a function of the liquid volume fraction:
%
\begin{equation}\label{eq:Aint-sect4}
A_{int} = A_{int}^{max} \left[ 6.75 \left(1-\alpha_{k} \right)^2 \alpha_{k} \right],
\end{equation}
% 
where $A_{int}^{max} = 5100$ $m^2 / m^3$. With such definition, the interfacial area is zero in the limits $\alpha_{k} = 0$ and $\alpha_{k} = 1$. Lastly, $\Gamma$ is the net mass transfer rate per unit interfacial area from phase $j$ to phase $k$. Its expression, given in \eqt{eq:mass_transfer}, is obtained by considering a vaporization/condensation process that is dominated by heat diffusion at the interface \cite{SEM, BerryMarco_2014}:
%
\begin{align} \label{eq:mass_transfer}
  \nonumber
  \Gamma = \Gamma_{j}
  &= \frac{h_{T,  k} \left( T_{k} - T_{int} \right) + h_{T,  j} \left( T_{j} - T_{int} \right)}{h_{j,  int} - h_{k,  int}}
  \\
  &= \frac{h_{T,  k} \left( T_{k} - T_{int} \right) + h_{T,  j} \left( T_{j} - T_{int} \right)}{L_v \left( T_{int} \right)} ,
\end{align}
%
where $L_v \left( T_{int} \right) = h_{j,  int} - h_{k,  int}$
represents the latent heat of vaporization.  The interface
temperature is determined by the saturation constraint
$T_{int}=T_{sat}(P)$ with the appropriate pressure $P=\bar{P}_{int}$
determined above. The interfacial heat transfer coefficients for phases $k$ and $j$ are denoted by $h_{T,  k}$ and $h_{T,  j}$, respectively, and computed from correlations \cite{SEM}. 

The set of equations obeyed by phase $j$ are simply obtained by substituting $k$ by $j$ and $j$ by $k$ in \eqt{eq:liq-7-eqn-sect5}, keeping the same definition of the interfacial variables and remembering that $\Gamma_j = - \Gamma_k$. In the case of two-phase flows, the equation for the volume fraction of phase $j$ is simply replaced by the algebraic relation
%
\begin{align}
 \alpha_{j}= 1 - \alpha_{k}, \nonumber
\end{align}
%
which reduces the number of equations from eight to seven and yields the seven-equation two-phase flow model. 

The seven-equation model has interesting properties that are discussed next. A set of seven waves is present in such a model: two acoustic waves and a contact wave for each phase supplanted by a volume fraction wave propagating at the interfacial velocity $\mbold u_{int}$. Considering a domain of dimension $\mathbb{D}$, the corresponding eigenvalues are the following for each phase $k$:
% 
\begin{align}\label{eq:eigenvalues}
&\lambda_1 = \mbold u_{int} \cdot \bar{\mbold n} \nonumber\\
&\lambda_{2,k} = \mbold u_k \cdot \bar{\mbold n} - c_k \nonumber\\
&\lambda_{3,k} = \mbold u_k \cdot \bar{\mbold n} + c_k \\
&\lambda_{d+3,k} = \mbold u_k \cdot \bar{\mbold n} \text{ for } d = 1 \dots \mathbb{D},\nonumber
\end{align}
%
where $\bar{\mbold n}$ is an unit vector pointing to a given direction. The eigenvalues given in \eqt{eq:eigenvalues} are unconditionally real which presents an interesting property for the development of numerical methods since the system is hyperbolic and well-posed. To relax the seven-equation model to
the ill-posed classical six-equation model, only the pressures should be
relaxed toward a single pressure for both phases.  This is
accomplished by specifying the pressure relaxation coefficient to be
very large, i.e., letting it approach infinity.  But if the pressure
relaxation coefficient goes to infinity, so does the velocity
relaxation rate also approach infinity.  This then relaxes the
seven-equation model not to the classical six-equation model but to the
mechanical equilibrium five-equation model of Kapila \cite{Kapila_2001}.  This reduced
five-equation model is also hyperbolic and well-posed. The five-equation
model provides a very useful starting point for constructing
multi-dimensional interface resolving methods which dynamically
captures evolving and spontaneously generated
interfaces~\cite{Saurel_2009}. Thus the seven-equation model
can be relaxed locally to couple seamlessly with such a
multi-dimensional, interface resolving code. Numerically, the mechanical relaxation coefficients $\mu_P$
(pressure) and $\lambda_u$ (velocity) can be relaxed independently to
yield solutions to useful, reduced models.  It
is noted, however, that relaxation of pressure only by making $\mu_P$
large without relaxing velocity will indeed give ill-posed and
unstable numerical solutions, just as the classical six-equation
two-phase model does, with sufficiently fine spatial resolution, as
confirmed in~\cite{SEM,Herrard_2005}. For each phase $k$, an entropy equation can be derived when accounting only for the pressure and velocity relaxation terms (all of the terms proportional to the net mass transfer term $\Gamma$ are removed). The entropy function for a phase $k$ is denoted by $s_k$ and function of the density $\rho_k$ and the internal energy $e_k$. The derivation is detailed in APPENDIX and only the final result is recalled here:
%
\begin{align}\label{eq:ent-eqn-7-eqn-model}
(s_{e})_k^{-1} \alpha_k \rho_k A \frac{Ds_k}{Dt} &= \mu_P \frac{Z_k}{Z_k+Z_j} (P_j - P_k)^2 + \lambda_u \frac{Z_j}{Z_k+Z_j} (\mbold u_j -\mbold  u_k)^2 \nonumber
\\
& \frac{Z_k}{\left( Z_k+Z_j \right)^2} \left[ Z_j (\mbold u_j-\mbold u_k)+\frac{\grad \alpha_k}{|| \grad \alpha_k ||}(P_k-P_j)\right]^2,
\end{align}
The partial derivative of the entropy function $s_k$ with respect to the internal energy $e_k$, $(s_e)_k$, is defined proportional to the inverse of the temperature of phase $k$ as for the single phase Euler equations \cite{jlg}. The right hand-side of \eqt{eq:ent-eqn-7-eqn-model} is unconditionally positive since all terms are squared and thus, can be used to demonstrate the entropy minimum principle and derive the dissipative terms. Furthermore, \eqt{eq:ent-eqn-7-eqn-model} is valid for each both phases $\left\{k, j\right\}$ and ensures positivity of the total entropy equation that is obtained by summing over the phases:
%
\begin{equation}\label{eq:tot-ent-res-sct4}
\sum_k (s_{e})_k^{-1} \alpha_k \rho_k A \frac{Ds_k}{Dt} = \sum_k (s_{e})_k^{-1} \alpha_k \rho_k A \left( \partial_t s_k + \mbold u_k \cdot \grad s_k \right) \geq 0 \nonumber .
\end{equation}
%
Note that when one phase disappears, \eqt{eq:tot-ent-res-sct4} degenerates into the single phase entropy equation.
%%%%%%%%%%%%%%%%%%%%%%%%%%%%%%%%%%%%%%%%%%%%%%%%%%%%%%%%%%%%%%%%%%%%%%%%%%%%%
\section{A viscous regularization for the multi-D 7-equation two-phase flow model}\label{sec:visc-regu}
%%%%%%%%%%%%%%%%%%%%%%%%%%%%%%%%%%%%%%%%%%%%%%%%%%%%%%%%%%%%%%%%%%%%%%%%%%%%%
\begin{itemize}
\item explain why we work with the phase entropy equation instead of considering the total entropy residual by summing over the two phases
\item viscous regularization must be consistent with single-phase flow equation
\item recall the notion of entropy condition and entropy inequality $\to$ require dissipative terms in order to get a sign
\item give the system of equations with the dissipative terms
\item guide the reader through the derivation of the dissipative terms
\item give the entropy residual with all terms in the right hand-side
\item make the link with the single-phase flow equations
\item explain how to derive the dissipative term for the volume fraction equation
\item emphasizes the fact that the regularization is valid for any EOS with convex entropy
\item a few words about the parabolic regularization
\end{itemize}
%
We now propose to derive a viscous regularization for the seven-equation model given in \eqt{eq:liq-7-eqn-sect5} by using the same method as for the multi-D Euler equation with/without variable area \cite{jlg, Marco_dissertation}. The method consists in adding perturbation terms to the system of equation under consideration, and re-derive the entropy equation whose sign is known to be positive to ensure uniqueness of the numerical solution \cite{Leveque}. Because of the addition of perturbation terms, the entropy equation is modified and contains extra terms of unknown sign. By carefully choosing a definition for each of the perturbation term, the sign of the entropy equation can be determined and proved positive. 
%
In this section, the dissipative terms for the multi-D seven-equation model \emph{with pressure and velocity relaxation source terms} are derived (the mass and energy transfer terms are omitted). The methodology proposed in  SECTION is followed. For clarity purpose, the seven-equation model with pressure and velocity relaxation terms is recalled when considering a phase $k$ in interaction with a second phase $j$:
%
\begin{subequations}\label{eq:sev_equ}
\begin{align}
\partial_t \left( \alpha_k  A\right) + A \mbold u_{int} \cdot \grad \alpha_k = A \mu_P \left( P_k - P_j \right)
\end{align}
\begin{align}
\partial_t \left( \alpha_k \rho_k A \right) + \div \left( \alpha_k \rho_k \mbold u_k A \right) = 0
\end{align}
\begin{align}
\partial_t \left( \alpha_k \rho_k u_k A \right) + \div \left[ \alpha_k A \left( \rho_k \mbold u_k \otimes \mbold u_k + P_k \mathbb{I} \right) \right] &=\nonumber\\
\alpha_k P_k \grad A + P_{int} A \grad \alpha_k &+ A \lambda_u \left( \mbold u_j - \mbold u_k \right)
\end{align}
\begin{align}
\partial_t \left( \alpha_k \rho_k E_k A \right) + \div \left[ \alpha_k A \mbold u_k \left( \rho_k E_k + P_k \right) \right] &=\nonumber\\
A P_{int} \mbold u_{int} \cdot \grad \alpha_k - \mu_P \bar{P}_{int} \left( P_k-P_j \right) &+ A \lambda_u \bar{\mbold u}_{int} \cdot \left( \mbold u_j - \mbold u_k \right)
\end{align}
\end{subequations}
%
%where $\rho_k$, $u_k$, $E_k$ and $P_k$ are the density, the velocity, the specific total energy and the pressure of $k^{th}$ phase, respectively. The pressure and velocity relaxation parameters are denoted by $\mu$ and $\lambda$, respectively. The variables with index $_I$ correspond to the interfacial variables and a definition for those can be found in \cite{SEM}. The cross-section $A$ is only function of space: $\partial_t A = 0$.
In order to apply the EVM, dissipative terms are added to each equation of the system given in \eqt{eq:sev_equ}, which yields:
%
\begin{subequations}\label{eq:sev_equ-with-diss-terms}
\begin{align}\label{eq:sev_equ-with-diss-terms-vf}
\partial_t \left( \alpha_k  A\right) + \mbold u_{int} A \grad \alpha_k = A \mu_P \left( P_k - P_j \right) + \div \mbold l_k
\end{align}
\begin{align}\label{eq:sev_equ-with-diss-terms-cont}
\partial_t \left( \alpha_k \rho_k A \right) + \div \left( \alpha_k \rho_k \mbold u_k A \right) = \div \mbold f_k
\end{align}
\begin{align}\label{eq:sev_equ-with-diss-terms-mom}
\partial_t \left( \alpha_k \rho_k \mbold u_k A \right) + \div \left[ \alpha_k A \left( \rho_k \mbold u_k \otimes \mbold u_k + P_k \mathbb{I} \right) \right] &=\nonumber\\
\alpha_k P_k \grad A + P_{int} A \grad \alpha_k &+ A \lambda_u \left( \mbold u_j - \mbold u_k \right) + \div \mbold g_k
\end{align}
\begin{align}\label{eq:sev_equ-with-diss-terms-ener}
\partial_t \left( \alpha_k \rho_k E_k A \right) + \div \left[ \alpha_k A \mbold u_k \left( \rho_k E_k + P_k \right) \right] &=\nonumber\\
P_{int} A \mbold u_{int} \cdot \grad \alpha_k - \mu_P \bar{P}_{int} \left( P_k-P_j \right) &+ A \lambda_u \bar{\mbold u}_{int} \cdot \left( \mbold u_j - \mbold u_k \right) + \div \left( \mbold h_k + \mbold u \cdot \mbold g_k \right)
\end{align}
\end{subequations}
%
where $\mbold f_k$, $\mbold g_k$, $\mbold h_k$ and $\mbold l_k$ are the dissipative terms. The next step consists of deriving the entropy equation for the phase $k$, on the same model as what is done in APPENDIX. Extra terms will appear in the right-hand-side of the entropy equation due to the dissipative terms. By choosing properly the definition of the dissipative terms, the sign of these extra terms can be controlled in order to ensure positivity of the entropy residual:
%
\begin{enumerate}
\item recast the system of equation given in \eqt{eq:sev_equ-with-diss-terms} in terms of the primitive variables $(\alpha_k, \rho_k, \mbold u_k, e_k)$.
\item derive the entropy equation by using the chain rule:
\begin{equation}
\label{eq:chain_rule-sct4}
\frac{Ds_k}{Dt} = \left( s_{\rho} \right)_k \frac{D \rho_k}{Dt} + \left( s_{e} \right)_k \frac{D e_k}{Dt} 
\end{equation}
where $\frac{D \cdot}{Dt}$ is the material derivative. The terms $(s_e)_k$ and $(s_{\rho})_k$ denote the partial derivative of the entropy $s_k$ with respect to $e_k$ and $\rho_k$, respectively.
\item isolate the terms of interest and choose an appropriate expression for each of the dissipative terms in order to ensure positivity of the right-hand side.
\end{enumerate}
%
We first recast \eqt{eq:sev_equ-with-diss-terms} in terms of the primitive variables: the volume fraction equation remains unchanged. The equation for the primitive variable $\rho_k$ is derived by combining \eqt{eq:sev_equ-with-diss-terms-vf} and \eqt{eq:sev_equ-with-diss-terms-cont}:
%
\begin{equation}\label{eq:rho-7-eqn-model-sect4}
\alpha_k A \left[ \partial_t \rho_k + \left( \mbold u_k - \textcolor{blue}{\mbold u_{int}} \right) \cdot \grad \rho_k \right] = \textcolor{blue}{A \rho_k \mu_P \left( P_k - P_j \right)} + \div \mbold f_k - \rho_k \div \mbold l_k
\end{equation}
%
The velocity equation is obtained by subtracting the density equation from the momentum equation:
%
\begin{align}\label{eq:vel-7-eqn-model-sect4}
\alpha_k \rho_k  A \left[ \partial_t \mbold u_k + \mbold u_k \cdot \div \mbold u_k \right]  + \div \left( \alpha_k \rho_k A P_k \mathbb{I} \right) &=\nonumber\\
\textcolor{blue}{\alpha_k P_k \grad A + P_{int} A \grad \alpha_k + A \lambda \left( \mbold u_j - \mbold u_k \right)} &+ \div \mbold g_k - \mbold u_k \otimes \mbold f_k
\end{align}
%
After multiplying \eqt{eq:vel-7-eqn-model-sect4} by the velocity vector $\mbold u_k$, the resulting kinetic energy equation is subtracted from the total energy equation to obtain the internal energy equation for phase $k$:
%
\begin{align}\label{eq:int-ener-7-eqn-model-sect4}
\alpha_k \rho_k  A \left[ \partial_t \mbold e_k + \mbold u_k \cdot \div \mbold e_k \right]  + \alpha_k \rho_k A P_k \grad \mbold u_k &=\nonumber\\
\textcolor{blue}{P_{int} A \left(\mbold u_{int}-\mbold u_k \right) \cdot \grad \alpha_k} &-  \textcolor{blue}{\alpha_k P_k \mbold u_k \grad A} \nonumber \\ 
\textcolor{blue}{-\bar{P}_{int} A \mu_P \left(P_k-P_j \right)} &+ \textcolor{blue}{A \lambda_u \left(\mbold u_j-\mbold u_k  \right) \cdot \left(\bar{\mbold u}_{int}- \mbold u_k \right)}\nonumber \\
&+ \div \mbold h_k + \mbold g_k : \grad \mbold u_k + || \mbold u ||^2_k \mbold f_k
\end{align}
%
The blue terms in \eqt{eq:rho-7-eqn-model-sect4} and \eqt{eq:int-ener-7-eqn-model-sect4} yield the positive terms in the right-hand-side of \eqt{eq:ent-eqn-7-eqn-model} and thus are ignored in the remaining of the derivation. The entropy equation is now obtained by combining the density equation (\eqt{eq:rho-7-eqn-model-sect4}) and the internal energy equation (\eqt{eq:int-ener-7-eqn-model-sect4}) through the chain rule given in \eqt{eq:chain_rule-sct4} to yield:
%
\begin{equation}\label{eq:ent-res-7-eqn-diss-terms}
\alpha_k \rho_k A \frac{Ds_k}{Dt} = \left(s_e\right)_k \left[ \div \mbold h_k + \mbold g_k : \grad \mbold u_k +  \left( || \mbold u ||^2_k - e_k\right) \div \mbold f_k  \right] + (\rho s_\rho)_k \left[ \div \mbold f_k - \rho_k \div \mbold l_k \right].
\end{equation}
%
where it was assumed that the entropy of phase $k$ satisfies the second thermodynamic law: 
%
\begin{align}\label{eq:2nd-therm-laws-sect4}
&T_k \text{d} s_k = \text{d}e_k - P_k\frac{\text{d}\rho_k}{\rho_k^2} \nonumber \\
& \text{which implies } P_k (s_e)_k + \rho_k (s_\rho)_k = 0, \\
& (s_e)_k = T_k^{-1} \text{ and } (s_\rho)_k = - (s_e)_k P_k \frac{\text{d}\rho_k}{\rho_k^2}. \nonumber
\end{align}
% 
From this point, two options are available in order to derive the dissipative terms: either we consider the total entropy residual of the system by summing \eqt{eq:ent-res-7-eqn-diss-terms} over each phase, or we can consider each phase independently. This dilemma can be answered by remembering that the seven-equation model degenerates into the single phase flow equations in the limits $\alpha_k = 0,1$. Thus, the dissipative terms also have to be consistent with the single-phase flow limits. As a result, it is chosen to derive the dissipative terms by considering each phase independently which will automatically ensure positivity of the total entropy residual as well.

The right-hand side of \eqt{eq:ent-res-7-eqn-diss-terms} can be further simplified by using the following expression
for the dissipative terms $\mbold f_k$,  $\mbold g_k$ and $\mbold h_k$:
\begin{align}\label{eq:def-diss-terms-sect4}
  \mbold f_k &= \tilde{\mbold f}_k + \rho_k \mbold  l_k 
  \\
  \mbold g_k &= \alpha_k \rho_k A \mu_k \mathbb{F}(\mbold u_k) + \mbold f_k \otimes \mbold u_k
  \\
  \mbold h_k &= \tilde{\mbold h}_k - \frac{|| \mbold u||^2 }{2} \mbold f_k + (\rho e)_k \mbold l_k.
\end{align}
Note the area function $A$ in the definition of $\mbold g$. It yields:
%
\begin{align}\label{eq:ent-res-7-eqn-diss-terms2}
&\alpha_k \rho_k A \frac{Ds_k}{Dt} = \nonumber \\
&\underbrace{\left(s_e\right)_k \alpha_k \rho_k A \mu_k \mathbb{F}(\mbold u_k) : \grad \mbold u_k}_{\mathcal{R}_1} +
\underbrace{\left[ \div \tilde{\mbold h}_k  - e_k \div \tilde{\mbold f}_k  \right] + (\rho s_\rho)_k \div \tilde{\mbold f}_k}_{\mathcal{R}_2} + \nonumber \\
&\underbrace{(s_e)_k \div \left( \rho_k e_k \mbold l_k \right) -  (s_e)_k e_k \div \left( \rho_k \mbold l_k \right) + \rho_k (s_\rho)_k \div \left( \rho_k \mbold l_k \right) 
  - \rho_k^2 (s_\rho)_k \div \mbold l_k}_{\mathcal{R}_3},
\end{align}
%
where $\mu_k$ is a positive viscosity coefficient for phase $k$. For simplicity, the right-hand-side of \eqt{eq:ent-res-7-eqn-diss-terms2} is split into three terms denoted by $\mathcal{R}_1$, $\mathcal{R}_2$ and $\mathcal{R}_3$. Since $(s_e)_k$ is defined as the inverse of the temperature and thus positive, the sign of the first term, $\mathcal{R}_1$, is conditioned by the choice of the function $\mathbb{F}(\mbold u_k)$ so that the product with the tensor $\grad \mbold u_k$ is positive. As in \cite{jlg}, $\mathbb{F}(\mbold u_k)$ is chosen proportional to the symmetric gradient of the velocity vector $\grad^s \mbold u_k$, whom entries are given by $(\grad^s \mbold u)_{i,j} = \frac{1}{2} \left( \partial_{x_i} u_i + \partial_{x_j} u_j \right)$. Such a choice ensures the associated dissipative terms to be rotationally invariant and also positivity of $\mathcal{R}_1$. An other option would be to simply set $\mathbb{F}(\mbold u_k)$ proportional to $\grad \mbold u_k$ which allows to recover the parabolic regularization. 

After a few lines of algebra, the third term ${\mathcal{R}_3}$ can be recast as a function of the gradient of the entropy as follows:
\begin{align}
 \label{eq:ent-R3-sct4}
  \mathcal{R}_2  =  \rho_k A \mbold l_k \cdot \grad s_k.
\end{align} 
One of the assumptions made in the entropy minimum principle is to that the entropy 
is at a minimum which implies that its gradient is null. Because of this, it follows that
the term $\mathcal{R}_3$ is zero at the minimum and thus, the entropy minimum principle is verified
independently of the definition of the dissipative term $\mbold l_k$ used in the volume fraction
equation. It will be explained later in this section how to derive a definition for $\mbold l_k$.

We now focus on the term denoted by $\mathcal{R}_2$, that is found identical to the right-hand-side of the single phase entropy equation obtained from the multi-D Euler equations (see \eqt{eq:rhs-euler-equ-app1} in APPENDIX). Thus, following \cite{jlg} and also APPENDIX, the term $\mathcal{R}_2$ is known to be positive when (i) assuming concavity of the entropy function $s_k$ with respect to the internal energy $e_k$ and the specific volume $1 / \rho_k$ (or convexity of $-s_k$) and (ii) choosing the following definitions for the dissipative terms $\tilde{h}_k$ and $\tilde{f}_k$:
%
\begin{align}
&\tilde{\mbold f}_k = \alpha_k A \kappa_k \grad \rho_k \\
&\tilde{\mbold h}_k = \alpha_k A \kappa_k \grad \left( \rho e \right)_k,
\end{align}
%  
where $\kappa_k$ is another positive viscosity coefficient. The entropy equation can now be written in its final form:
%
\begin{align}\label{eq:ent-res-7-eqn-diss-terms3}
&\alpha_k \rho_k A \frac{Ds_k}{Dt} - \mbold f_k \cdot \grad s_k - \div \left( \alpha_k \rho_k A \grad s_k \right) = \nonumber\\
&- \alpha_k A \kappa_k \mathbf{Q}_k + (s_e)_k \alpha_k A \rho_k \mu_k \grad^s \mbold u_k : \grad \mbold u_k,
\end{align}
%
where $\mathbf{Q}_k$ is a negative semi-definite quadratic form defined as:
%
\begin{eqnarray}
\mathbf{Q}_k &=& X^t_k \Sigma_k X_k \nonumber \\
\text{with } X_k &=& \begin{bmatrix}
\grad \rho_k \\
\grad e_k 
\end{bmatrix}
\text{and } \Sigma_k = \begin{bmatrix}
       \partial_{\rho_k} (\rho^2_k \partial_{\rho_k} s_k) & \partial_{\rho_k,e_k} s_k  \\[0.3em]
       \partial_{\rho_k,e_k} s_k & \partial_{e_k,e_k} s_k           \\[0.3em]
     \end{bmatrix}. \nonumber 
\end{eqnarray}
%
\eqt{eq:ent-res-7-eqn-diss-terms3} is used to prove the entropy minimum principle: assuming that $s_k$ reaches its minimum value in $\mbold r_{min}(t)$ at each time $t$, the gradient, $\grad s_k$, and Laplacian, $\Delta s_k$,  of the entropy are null and positive at this particular point, respectively. Furthermore, it is recalled that the viscosity coefficients $\mu_k$ and $\kappa_k$ are positive by definition. Then, because the right-hand-side of \eqt{eq:ent-res-7-eqn-diss-terms3} is proven positive, the entropy minimum principle holds for each phase $k$, \textbf{independently of the definition of the dissipative term} $\mbold l_k$, such as:
%
\begin{equation}\label{eq:ent-res-7-eqn-diss-terms4}
\alpha_k \rho_k A \partial_t s_k(\mbold r_{min},t)) \geq 0 \Rightarrow \partial_t s_k(\mbold r_{min},t)) \geq 0 \nonumber
\end{equation}
%

It remains to obtain a definition for the
dissipative term $\mbold l_k$ used in the volume fraction equation. A way to achieve this is to
consider the volume fraction equation, \eqt{eq:sev_equ-with-diss-terms-vf}, by itself and notice that it is an hyperbolic equation
with eigenvalue $\mbold u_{int}$. An entropy equation can be derived and used to prove the
entropy minimum principle by properly choosing the dissipative term. The objective is to
ensure positivity of the volume fraction and also uniqueness of the weak solution. Following
the work of Guermond et al. in \cite{jlg1, jlg2} and by analogy
with Burger's equation described in SECTION, it can be shown that a dissipative term ensuring positivity and
uniqueness of the weak solution for the volume fraction equation, is of the form $\mbold l_k = \beta_k A \grad \alpha_k $ where $\beta_k$
is a positive viscosity coefficient.

All of the dissipative terms are now defined and recalled here:
%
\begin{subequations}\label{eq:visc-reg-7-equ-sect4}
\begin{align}
  \mbold l_k &= \beta_k A \grad \alpha_k 
\end{align}
\begin{align}
  \mbold f_k &= \alpha_k A \kappa_k \grad \rho_k + \rho_k A \mbold l_k 
\end{align}
\begin{align}
  \mbold g_k &= \alpha_k A \mu_k \rho \grad^s \mbold u_k 
\end{align}
\begin{align}
  \mbold h_k &=  \alpha_k A \kappa_k \grad \left( \rho e \right)_k + \mbold u_k : \mbold g_k - \frac{|| \mbold u_k||^2}{2} \mbold f_k + (\rho e)_k \mbold l_k 
\end{align}
\end{subequations}
%
At this point, some remarks are in order:
\begin{enumerate}
\item {The viscous regularization given in \eqt{eq:visc-reg-7-equ-sect4} for the multi-D seven-equation model, is equivalent to the parabolic regularization \cite{Parabolic} when assuming $\beta_k = \kappa_k$ and $\mathbb{F}(\mbold u_k) = \alpha_k \rho_k \kappa_k \grad \mbold u_k$. However, decoupling between the regularization on the velocity and on the density in the momentum equation is important to make the regularization rotation invariant but also to ensure well-scaled dissipative terms for a wide range of Mach number as was shown in SECTION for the multi-D Euler equations.}
\item {The dissipative term $\mbold l_k$ requires the definition of a new viscosity
    coefficient $\beta_k$. It was shown that this viscosity coefficient is independent of
    the other viscosity coefficients $\mu_k$ and $\kappa_k$. Its definition should
    account for the eigenvalue associated with the void fraction equation $\mbold u_{int}$.
    In addition, an entropy residual can be determined by analogy to Burger's
    equation. }
%    It is noted, however, that the eigenvalue $\mbold u_{int}$ can be discontinuous
%    since its definition involves the sign of the void fraction gradient, which
%    makes the theory more challenging. For simplicity, we ignore this aspect of the
%    theory in this report.\tcr{maybe for a report, but here you should say a bit more}}

\item {The dissipative term $\mbold f_k$ is a function of $\mbold l_k$. Thus, all of the other
    dissipative terms are also functions of $\mbold l_k$.}

\item {The partial derivatives $(s_e)_k$ and $(s_{\rho_k})_k$ can be computed using the
    definition provided in \eqt{eq:2nd-therm-laws-sect4} and are functions of the thermodynamic
    variables: pressure, temperature and density.}

\item {All of the dissipative terms are chosen to be proportional to the the void
    fraction $\alpha_k$ and the cross-sectional area $A$, but the one in the volume fraction equation that is only proportional to $A$. For instance, $\alpha_k A \grad \rho_k$ is the
    flux of the dissipative term in the continuity equation through the area seen
    by the phase $\alpha_k A$. When one of the phases disappears, the dissipative terms
    must to go to zero for consistency. On the other hand, when $\alpha_k$ goes to one,
    the single-phase equation must be recovered. }
    
\item{Compatibility of the viscous regularization proposed in \eqt{eq:visc-reg-7-equ-sect4} with the generalized entropies identified in Harten et al. \cite{Harten} has not been investigated yet. However, it is believed that the entropy inequalities still holds because of the similarities of the entropy residual for the multi-D seven-equation model with the entropy residual derived in the single phase flow case \cite{jlg}.} 
%A rigorous proof is ongoing work and will be included in the final version.
%\tcr{don't you want to include this in the final version?} \tcb{I do}}
\end{enumerate}
%
Through the derivations of the viscous regularization, it was noted that another set of dissipative terms $\mbold f_k$ and $\mbold l_k$ would also ensures positivity of the entropy residual:
%
\begin{subequations}
\begin{align}\label{eq:def-l-k-wrong-sect4}
\mbold l_k =\beta_k T_k \left[ \frac{\rho_k}{P_k+\rho_k e_k} \grad \left( \frac{P_k}{\rho_k e_k} \right) - \frac{1}{P_k} \grad \rho_k \right]
\end{align}
\begin{align}
\mbold f_k = \kappa_k \grad \rho_k +  \frac{\rho^2_k (s_{\rho})_k}{\left( \rho s_{\rho} - e s_e \right)_k} \mbold l_k
\end{align}
\end{subequations}
%
However, the definition of $\mbold l_k$ proposed in \eqt{eq:def-l-k-wrong-sect4} was not considered as valid for the following reasons: positivity of the volume fraction cannot be achieved and the parabolic regularization is not retrieved.\\ 

A rotation invariant viscous regularization for the multi-D seven-equation model is now available involving three viscosity coefficients $\beta_k$, $\mu_k$ and $\kappa_k$, for each phase $k$. Definition of these viscosity coefficients is the purpose of the next section (SECTION).
%
%%%%%%%%%%%%%%%%%%%%%%%%%%%%%%%%%%%%%%%%%%%%%%%%%%%%%%%%%%%%%%%%%%%%%%%%%%%%%
\section{A definition of the viscosity coefficients for all Mach flows}\label{sec:low-Mach}
%%%%%%%%%%%%%%%%%%%%%%%%%%%%%%%%%%%%%%%%%%%%%%%%%%%%%%%%%%%%%%%%%%%%%%%%%%%%%
\begin{itemize}
\item non-dimensionalize the equations but use $P_\infty$ for the pressure instead of $(\rho c^2)_\infty$
\item introduce a new Pechlet number for $\beta$: its behavior should be the same as the Pechlet number for $\kappa$
\item two cases: zero and infinite relaxation coefficients
\item derive the normalization parameters for the isentropic and non-isentropic flows
\item discussion about the 
\end{itemize}
%
This section aims at deriving a definition of the viscosity coefficients involved in the viscous regularization for the multi-D seven-equation model. We propose to follow the same methodology as in SECTION for the multi-D Euler equations: after obtaining the non-dimensional equations, a definition for the viscosity coefficients is derived based on the entropy residual and consistent with the low-Mach asymptotic limit. Particular attention is paid to the definition of the viscosity coefficient $\beta_k$ used in the volume fraction equation.

Using the EVM to define the viscosity coefficients is not the unique option here. Other numerical methods initially developed for single-phase flows, such as pressure-based and Lapidus viscosity methods, could be used as a starting point and adapted to the seven-equation model. Such a reasoning is motivated by one of the initial assumptions of the seven-equation model that assumes each phase verifies the Euler equations.
%------------------------------------------------------------------------------------------------------
\subsection{Definition of the viscosity coefficients}\label{sec:visc-coeff-sem}
%------------------------------------------------------------------------------------------------------
The viscous regularization derived in SECTION for the multi-D SEM requires three viscosity coefficients for each phase $k$ denoted by $\beta_k$, $\mu_k$ and $\kappa_k$. Following the methodology detailed in SECTION, for each viscosity coefficient an upper bound, denoted by the subscript $max$, is defined and referred to as the first-order viscosity coefficient, along with a entropy viscosity coefficient that is set proportional to an entropy residual and denoted by the subscript $e$:
%
\begin{align}\label{eq:def-visc-sem-sct4}
\beta_k( \mbold r, t) = \min \left( \beta_{e,k}( \mbold r, t), \beta_{max,k} ( \mbold r, t) \right), \nonumber \\
\mu_k( \mbold r, t) = \min \left( \mu_{e,k}( \mbold r, t), \mu_{max,k} ( \mbold r, t) \right), \nonumber \\
\kappa_k( \mbold r, t) = \min \left( \kappa_{e,k}( \mbold r, t), \kappa_{max,k} ( \mbold r, t) \right) \,. \nonumber
\end{align}
% 
where all of the variables are locally defined. As for the multi-D single-phase Euler equations and for the same reasons, the entropy residual for each phase $k$ is recast as a function of the pressure, the velocity, the density and the speed of sound as follows:
%
\begin{equation}\label{eq:ent_res-sem-sct4}
\resi_k(\mbold r,t) := \partial_t s_k + \mbold u_k \cdot \grad s_k = \matder{s_k} = \frac{(s_e)_k}{(P_e)_k} \left( \underbrace{\matder{P_k} - c_k^2 \matder{\rho_k} }_{\resinew_k(\mbold r,t)} \right) ,
\end{equation} 
%
where $\resinew_k(\mbold r,t)$ is the new entropy residual of phase $k$ and will experience the same variations as $\resi_k(\mbold r,t)$. 

We first choose to investigate the definitions of the high and first-order viscosity coefficients for $\mu_k$ and $\kappa_k$. It is noted that the dissipative terms function of $\mu_k$ and $\kappa_k$ are the same as the ones for the single-phase Euler equation when considering $\tilde{A} = \alpha_k A$ as a pseudo cross section. Furthermore, we need to ensure consistency with the single-phase Euler equation in the limits $\alpha_k \to 1$. Thus, based on the work done in SECTION, the first order viscosity coefficients are set proportional to the local maximum eigenvalue $\lambda_k$,
%
\begin{equation}\label{eq:def-visc-max-sem-sct4}
\kappa_{max,k}( \mbold r, t) = \mu_{max,k}( \mbold r, t) = \frac{h}{2} \left( || \mbold u_k|| + c_k \right)
\end{equation}
%
and the entropy viscosity viscosity coefficients are defined as
%
\begin{subequations}
\label{eq:visc_definition-sct4}
\begin{equation}
\mu_{e,k}(\mbold r,t)    = h^2 \frac{\max\left( | \resinew_k(\mbold r_q,t) |\,, || \mbold u_k(\mbold r_q,t) || J[P_k](t) \,, || \mbold u_k(\mbold r_q,t) || c_k^2(\mbold r_q,t) J[\rho_k](t) \right)}{\norm_{P,k}^\mu}    \, ,
\end{equation} 
\text{and} 
\begin{equation}
\kappa_{e,k}(\mbold r,t) = h^2 \frac{\max\left( | \resinew_k(\mbold r_q,t) |\,, || \mbold u_k(\mbold r_q,t) || J[P_k](t) \,, || \mbold u_k(\mbold r_q,t) || c_k^2(\mbold r_q,t) J[\rho_k](t) \right)}{\norm_{P,k}^\kappa} \, .
\end{equation}
\end{subequations}
%
where $h$ is the grid size and $J[x](t)$ denotes the jump of the quantity $x$ and was defined in SECTION. The normalization parameters $\norm_{P,k}^\mu$ and $\norm_{P,k}^\kappa$ will be determined later in this section by inspecting the non-dimensional version of the seven-equation model.

It remains to specify the viscosity coefficients $\beta_e$ and $\beta_{max}$. For the purpose of this paragraph, let us consider the scalar volume fraction equation and assume that the interface velocity $\mbold u_{int}$ is given. Because it is a scalar hyperbolic equation, it is proposed to define the high and first-order viscosity coefficients on the same model as Burger's equation. Thus, $\beta_{max}$ is set proportional to the eigenvalue that is the interface velocity $\mbold u_{int}$,
%
\begin{equation}\label{eq:def-beta-max-sen-sect4}
\beta_{max,k}( \mbold r, t) = \frac{h}{2} || \mbold u_{int} ||,
\end{equation}
%
whereas the entropy viscosity viscosity coefficient $\beta_e$ is function of an entropy residual, $R_{\alpha,k}$, derived from the volume fraction equation for phase $k$ as follows:
%
\begin{align}\label{eq:def-beta-sen-sect4}
\beta_{e,k}( \mbold r, t) = h^2 \frac{\max\left( | R_{\alpha,k}(\mbold r_q,t) |\,, || \mbold u_{int}(\mbold r_q,t) || J[\alpha_k](t) \right)}{\norm_{k}^\beta} \,
\end{align}
%
where $\norm_{k}^\beta$ denotes a normalization parameters whom definition will be further investigated. To derive the entropy residual $R_{\alpha,k}$, we consider the volume fraction equation for phase $k$ with its viscous regularization and assume the existence of a mathematical entropy denoted by $\eta(\alpha_k)$:
%
\begin{equation}\label{eq:vf-sem-sct4}
\partial_t \left(A \alpha_k \right) + A \mbold u_{int} \cdot \grad \alpha_k = \div \left( \beta_k A \grad \alpha_k \right)
\end{equation}
% 
After multiplying by $\frac{\text{d} \eta (\alpha_k)}{\text{d} \alpha_k}$ and using the chain rule, an expression for the entropy residual $R_{\alpha,k}$ is obtained:
%
\begin{equation}\label{eq:vf-sem2-sct4}
R_{\alpha,k} = \partial_t \left(A \eta(\alpha_k) \right) + A \mbold u_{int} \cdot \grad \eta(\alpha_k) = \frac{\text{d} \eta (\alpha_k)}{\text{d} \alpha_k} \div \left( \beta_k A \grad \alpha_k \right)
\end{equation}
% 
Because \eqt{eq:vf-sem2-sct4} is identical to \eqt{eq:weak_sol8_sct1b}, it is concluded that $R_{\alpha,k} \geq 0$ when assuming $\eta$ convex with respect to $\alpha_k$, which justifies the definition of the entropy viscosity viscosity coefficient $\beta_{e,k}$ given in \eqt{eq:def-beta-sen-sect4} based on \eqt{sec:evm_hyp_sc_sct1b}. The entropy function is taken equal to $\eta(\alpha_k) = \frac{\alpha_k^2}{2}$ which is convex.
%
%------------------------------------------------------------------------------------------------------
\subsection{Low-Mach asymptotic limit of the seven-equation model}\label{sec:low-Mach-sem}
%------------------------------------------------------------------------------------------------------
%
In order to have a complete definition for the viscosity coefficients $\beta_k$, $\mu_k$ and $\kappa_k$, the normalization parameters introduced in the definition of the entropy viscosity coefficients $\beta_{e,k}$, $\mu_{e,k}$ and $\kappa_{e,k}$ have to be determined. In SECTION, the normalization parameters were derived from the non-dimensionalized multi-D Euler equations in order to obtain well-scaled dissipative terms. Thus, it is proposed to follow the same method to derive the three normalization parameters $\norm_{P,k}^\mu$, $\norm_{P,k}^\kappa$ and $\norm_{k}^\beta$ used in the definition of the viscosity coefficients involved in the viscous regularization of the seven-equation model. For simplicity, the Ideal Gas equation of state is considered through the derivations.

For now, the definition of the viscosity coefficients is simply derived by analogy to SECTION. First, we define the far-field or stagnation coefficients for each phase as it is done in \eqt{eq:norm_param} by adding the subscript $k$ to $\infty$. Then, the scaled equations are derived for each phase which leads to the definition of a phasic P\'eclet and Reynolds numbers referred to as $\Pe_k$ and $\Re_k$, respectively, that are tied to the far-field or stagnation quantities of the viscosity coefficients $\mu_{k,\infty}$ and $\kappa_{k,\infty}$ as shown in \eqt{eq:ref_numb_7eq}:
%  
\begin{equation}
\label{eq:ref_numb_7eq}
\Re_{k,\infty} = \frac{u_{k,\infty} L_\infty}{\mu_{k,\infty}} \text{ and }
\Pe_{k,\infty} = \frac{u_{k,\infty} L_\infty}{\kappa_{k,\infty}} \, .
\end{equation}
%
Because the viscous regularization derived previously in SECTION requires an extra viscosity coefficient $\beta_k$ for the volume fraction equation, a new P\'eclet number, $\Pe_{k,\infty}^\beta$ is also defined as follows,
%
\begin{equation}
\label{eq:ref_numb_7eq_beta}
\Pe_{k,\infty}^\beta = \frac{u_{int,\infty} L_\infty}{\beta_{k,\infty}} \,
\end{equation}
%
that will allow us to derive the proper scaling for $\beta_{k,\infty}$. Once the scaled equations are obtained, the scaling of the numerical numbers can be chosen in order to meet the different criteria already listed in SECTION. The scaling of the new P\'eclet number we defined, $\Pe_{k,\infty}^\beta$, is derived from the scaled volume fraction equation that does not contain any term weighted by the reference Mach number $M_\infty$, which yields $\Pe_{k,\infty}^\beta=1$ to have a well-scaled dissipative term. This scaling is the same as for $\Pe_{k,\infty}$ from the continuity equation: the volume fraction and continuity equations have similar behavior since they are both advection-type equations. Thus, based on the reasoning used in SECTION, the following definitions for the viscosity coefficients is proposed in \eqt{eq:final_def_visc_coeff-sem}: 
%
\begin{subequations}
\label{eq:final_def_visc_coeff-sem}
%
\begin{equation}
\mu_k(\mbold r,t)    = \min \Big (\mu_{\max,k}(\mbold r,t), \mu_{e,k} (\mbold r,t)    \Big) \text{  and  }
\kappa_k(\mbold r,t) = \min \Big (\mu_{\max,k}(\mbold r,t), \kappa_{e,k} (\mbold r,t) \Big ) 
\end{equation}
%
where the first-order viscosity is given by
\begin{equation}\label{eq:first-order-visc-sct4-sem}
  \kappa_{\max,k}(\mbold r,t)  = \mu_{\max,k} (\mbold r,t) = \frac{h}{2} \Big ( ||\mbold u_k|| + c_k \Big ) 
\end{equation}
%
and the entropy viscosity coefficients by 
%
\begin{equation}
\kappa_{e,k}(\mbold r,t) = \frac{h^2 \max(\resinew_k, J_k)}{ \rho_k c_k^2 }  \text{  and  }
\mu_{e,k}(\mbold r,t)    = \frac{h^2 \max(\resinew_k, J_k)}{ \norm_{P,k}^\mu} 
\end{equation}
% 
with the jumps given by
%
\begin{equation}
J_k =  \max \Big ( || \mbold u_k || [[ \grad P_k \cdot \mbold n ]], || \mbold u_k || c_k^2 [[\grad \rho_k \cdot \mbold n]] \Big) 
\end{equation}
\end{subequations}
%
where $\norm_{P,k}^\kappa$ is computed from \eqt{eq:norm_ent3-7eq}.
%
\begin{equation}
\label{eq:norm_ent3-7eq}
\norm_P^\mu = (1-\sigma(M)) \rho c^2  + \sigma(M)  \rho ||\mbold{u} ||^2  
\end{equation}
%
%\begin{equation}
%\label{eq:norm_ent2-sem}
%\norm_{P,k}^\mu =  \left\{
%\begin{array}{ll}
% \rho_k ||\mbold u_k ||^2       & \text{ if } \left| \resinew_k^* \right| \geq M_k \text{ (i.e., non-isentropic flow)} \\
% \rho_k c_k^2 = \norm_{P,k}^\kappa & \text{ otherwise}
%\end{array}
%\right. \,.
%\end{equation}
%
where $M_k$ is the local Mach number for phase $k$. The function $\sigma(M)$ is taken from \eqt{eq:sigma_fct} with the same parameters as for the single-phase flow equations: $a=3$ and $M^{thres} = 0.05$. The jump $J_k$ is a function of the jump of pressure and density gradients across the face with respect to its normal vector $\mbold n$. Then, the largest value over all faces is determined and used in the definition of the viscosity coefficients. Lastly, the viscosity coefficient for the volume fraction equation is given by:
%
\begin{equation}\label{eq:first-order-beta-sct4-sem}
\beta_k(\mbold r,t) = \min \Big (\beta_{\max,k}(\mbold r,t), \beta_{e,k} (\mbold r,t) \Big ) 
\end{equation}
%
where the first-order viscosity is given by
\begin{equation}\label{eq:first-order-beta-max-sct4-sem}
\beta_{\max,k} (\mbold r,t) = \frac{h}{2} ||\mbold u_{int}||
\end{equation}
%
and the corresponding entropy viscosity coefficient, $\beta_{e,k}$, by 
%
\begin{equation}
\beta_{e,k}(\mbold r,t) = \frac{h^2 \max(R_{\alpha,k}, J_{\alpha,k})}{|| \alpha_k - \bar{\alpha}_k ||_\infty},
\end{equation}
where $\bar{\alpha}_k$ is the average value of the volume fraction over the entire computational domain, and $|| \cdot ||_\infty$ denotes the infinite norm. The definition of the $\beta_{e,k}$ is consistent with the scaling of $\Pe^\beta_{k,\infty} = 1$. The jump is given by:
%
\begin{equation}
J_{\alpha,k} = || \mbold u_{int} || \cdot [[ \grad \alpha_k \cdot \mbold n ]]. 
\end{equation}
With the definition of the viscosity coefficients $\mu_k$ and $\kappa_k$ proposed in \eqt{eq:final_def_visc_coeff}, the low-Mach asymptotic limit is ensured for isentropic flow, and transonic flows with shocks will be correctly resolved for each phase $k$. Furthermore, the definition of the viscosity coefficient $\beta_k$ is consistent with the EVM used for the scalar hyperbolic equations and thus should efficiently stabilize shocks forming the in the volume fraction profile. Plus, it is noted that the viscous regularization and the definition of the viscosity coefficients proposed for the seven-equation two-phase flow model degenerates into the EVM used for the single-phase Euler equations. In order to validate the proposed definition of the viscosity coefficients, 1-D numerical simulations are performed in SECTION.
%
%%%%%%%%%%%%%%%%%%%%%%%%%%%%%%%%%%%%%%%%%%%%%%%%%%%%%%%%%%%%%%%%%%%%%%%%%%%%%
\section{$1$-D numerical results}\label{sec:results}
%%%%%%%%%%%%%%%%%%%%%%%%%%%%%%%%%%%%%%%%%%%%%%%%%%%%%%%%%%%%%%%%%%%%%%%%%%%%%
\begin{itemize}
\item simple advection problem
\item shock tube with two independent fluids: exact solution and could do convergence test for this particular test
\item shock tube with infinite relaxation coefficients
\item $1$-D nozzle with two independent fluids
\item $1$-D nozzle with infinite relaxation coefficients
\item $1$-D nozzle with infinite relaxation coefficients,  mass and heat transfer
\end{itemize}
%%%%%%%%%%%%%%%%%%%%%%%%%%%%%%%%%%%%%%%%%%%%%%%%%%%%%%%%%%%%%%%%%%%%%%%%%%%%%
\bibliography{mybibfile}
%%%%%%%%%%%%%%%%%%%%%%%%%%%%%%%%%%%%%%%%%%%%%%%%%%%%%%%%%%%%%%%%%%%%%%%%%%%%%
\appendix
\end{document}