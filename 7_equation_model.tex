%%
%% This is file `elsarticle-template-num.tex',
%% generated with the docstrip utility.
%%
%% The original source files were:
%%
%% elsarticle.dtx  (with options: `numtemplate')
%% 
%% Copyright 2007, 2008 Elsevier Ltd.
%% 
%% This file is part of the 'Elsarticle Bundle'.
%% -------------------------------------------
%% 
%% It may be distributed under the conditions of the LaTeX Project Public
%% License, either version 1.2 of this license or (at your option) any
%% later version.  The latest version of this license is in
%%    http://www.latex-project.org/lppl.txt
%% and version 1.2 or later is part of all distributions of LaTeX
%% version 1999/12/01 or later.
%% 
%% The list of all files belonging to the 'Elsarticle Bundle' is
%% given in the file `manifest.txt'.
%% 

%% Template article for Elsevier's document class `elsarticle'
%% with numbered style bibliographic references
%% SP 2008/03/01

%\documentclass[preprint,12pt]{elsarticle}
\documentclass[preprint,10pt]{elsarticle}
%\documentclass[final,3p,times]{elsarticle} 

%% Use the option review to obtain double line spacing
%% \documentclass[authoryear,preprint,review,12pt]{elsarticle}

%% Use the options 1p,twocolumn; 3p; 3p,twocolumn; 5p; or 5p,twocolumn
%% for a journal layout:
%% \documentclass[final,1p,times]{elsarticle}
%% \documentclass[final,1p,times,twocolumn]{elsarticle}
%% \documentclass[final,3p,times]{elsarticle}
%% \documentclass[final,3p,times,twocolumn]{elsarticle}
%% \documentclass[final,5p,times]{elsarticle}
%% \documentclass[final,5p,times,twocolumn]{elsarticle}

%% if you use PostScript figures in your article
%% use the graphics package for simple commands
\usepackage{float}
\usepackage{color}
\usepackage{caption}
\usepackage{subcaption}
\usepackage{appendix}
%% or use the graphicx package for more complicated commands
\usepackage{graphicx}
%% or use the epsfig package if you prefer to use the old commands
%% \usepackage{epsfig}

%% The amssymb package provides various useful mathematical symbols 
%% The amsthm package provides extended theorem environments
\usepackage{amssymb}
\usepackage{amsmath}
% more math
\usepackage{amsfonts}
\usepackage{amstext}
\usepackage{amsbsy}
\usepackage{mathbbol} 
%% The lineno packages adds line numbers. Start line numbering with
%% \begin{linenumbers}, end it with \end{linenumbers}. Or switch it on
%% for the whole article with \linenumbers.
\usepackage{lineno}

\journal{Journal of Comp. Phys.}
%%%%%%%%%%%%%%%%%%%%%%%%%%%%%%%%%%%%%%%%%%%%%%%%%%%%%%%%%%%%%%%%%%%%
% operators
\renewcommand{\div}{\vec{\nabla}\! \cdot \!}
\newcommand{\grad}{\vec{\nabla}}
\newcommand{\divv}[1]{\boldsymbol{\nabla}^{#1}\! \cdot \!}
\newcommand{\gradd}[1]{\vec{\nabla}^{#1}}
\newcommand{\mbold}[1]{\boldsymbol#1}
% latex shortcuts
\newcommand{\bea}{\begin{eqnarray}}
\newcommand{\eea}{\end{eqnarray}}
\newcommand{\be}{\begin{equation}}
\newcommand{\ee}{\end{equation}}
\newcommand{\bal}{\begin{align}}
\newcommand{\eali}{\end{align}}
\newcommand{\bi}{\begin{itemize}}
\newcommand{\ei}{\end{itemize}}
\newcommand{\ben}{\begin{enumerate}}
\newcommand{\een}{\end{enumerate}}
% DGFEM commands
\newcommand{\jmp}[1]{[\![#1]\!]}                     % jump
\newcommand{\mvl}[1]{\{\!\!\{#1\}\!\!\}}             % mean value
\newcommand{\keff}{\ensuremath{k_{\textit{eff}}}\xspace}
% shortcut for domain notation
\newcommand{\D}{\mathcal{D}}
% vector shortcuts
\newcommand{\vo}{\vec{\Omega}}
\newcommand{\vr}{\vec{r}}
\newcommand{\vn}{\vec{n}}
\newcommand{\vnk}{\vec{\mathbf{n}}}
\newcommand{\vj}{\vec{J}}
\newcommand{\eig}[1]{\| #1 \|_2}
%
\newcommand{\EI}{\mathcal{E}_h^i}
\newcommand{\ED}{\mathcal{E}_h^{\partial \D^d}}
\newcommand{\EN}{\mathcal{E}_h^{\partial \D^n}}
\newcommand{\ER}{\mathcal{E}_h^{\partial \D^r}}
\newcommand{\reg}{\textit{reg}}
%
\newcommand{\norm}{\textrm{norm}}
\renewcommand{\Re}{\textrm{Re}}
\newcommand{\Pe}{\textrm{P\'e}}
\renewcommand{\Pr}{\textrm{Pr}}
%
% extra space
\newcommand{\qq}{\quad\quad}
% common reference commands
\newcommand{\eqt}[1]{Eq.~(\ref{#1})}                     % equation
\newcommand{\fig}[1]{Fig.~\ref{#1}}                      % figure
\newcommand{\tbl}[1]{Table~\ref{#1}}                     % table
\newcommand{\sct}[1]{Section~\ref{#1}}                   % section
\newcommand{\app}[1]{Appendix~\ref{#1}}                   % appendix
%
\newcommand{\ie}{i.e.,\@\xspace}
\newcommand{\eg}{e.g.,\@\xspace}
\newcommand{\psc}[1]{{\sc {#1}}}
\newcommand{\rs}{\psc{R7}\xspace}
%
\newcommand\br{\mathbf{r}}
%\newcommand{\tf}{\varphi}
\newcommand{\tf}{b}
%
\newcommand{\tcr}[1]{\textcolor{red}{#1}}
\newcommand{\tcb}[1]{\textcolor{blue}{#1}}
\newcommand{\mt}[1]{\marginpar{ {\tiny #1}}}
%
\bibliographystyle{elsarticle-num}
%%%%%%%%%%%%%%%%%%%%%%%%%%%%%%%%%%%%%%%%%%%%%%%%%%%%%%%%%%%%%%%%%%%%%
%
%   BEGIN DOCUMENT
%
%%%%%%%%%%%%%%%%%%%%%%%%%%%%%%%%%%%%%%%%%%%%%%%%%%%%%%%%%%%%%%%%%%%%%
\begin{document}

%%%%%%%%%%%%%%%%%%%%%%%%%%%%%%%%%%%%%%%%%%%%%%%%%%%%%%%%%%%%%%%%%%%%
\begin{frontmatter}

%% Title, authors and addresses

%% use the tnoteref command within \title for footnotes;
%% use the tnotetext command for theassociated footnote;
%% use the fnref command within \author or \address for footnotes;
%% use the fntext command for theassociated footnote;
%% use the corref command within \author for corresponding author footnotes;
%% use the cortext command for theassociated footnote;
%% use the ead command for the email address,
%% and the form \ead[url] for the home page:
%\title{Title\tnoteref{label1}}
%% \tnotetext[label1]{}
%% \author{Name\corref{cor1}\fnref{label2}}
%% \ead{email address}
%% \ead[url]{home page}
%% \fntext[label2]{}
%% \cortext[cor1]{}
%% \address{Address\fnref{label3}}
%% \fntext[label3]{}
%-------------------------
%-------------------------
\title{Extension of the entropy viscosity method to the multi-D 7-equation two-phase flow model.\\
\tcb{I do not know if we should have 'multi-D' in the title since we will only present $1$-D results}}
%-------------------------
%-------------------------
\author{Marc O. Delchini\fnref{label1}}
\ead{delchmo@tamu.edu}

\author{Jean C. Ragusa\corref{cor1}\fnref{label1}}
\ead{jean.ragusa@tamu.edu}

\author{Ray A. Berry\fnref{label2}}
\ead{ray.berry@inl.gov}

\address[label1]{Department of Nuclear Engineering, Texas A\&M University, College Station, TX 77843, USA \fnref{label1}}

\address[label2]{Idaho National Laboratory, Idaho Falls, ID 83415, USA \fnref{label2}}

\cortext[cor1]{Corresponding author}
%-------------------------
%-------------------------
%-------------------------
\begin{abstract}
blabla
\end{abstract}
%-------------------------
%-------------------------
\begin{keyword}
  two-phase flow model \sep with variable area \sep entropy viscosity method \sep stabilization method \sep low Mach regime \sep shocks.
\end{keyword}
%-------------------------
\end{frontmatter}
\linenumbers
%%%%%%%%%%%%%%%%%%%%%%%%%%%%%%%%%%%%%%%%%%%%%%%%%%%%%%%%%%%%%%%%%%%%%%%%%%%%%
\section{Introduction}\label{sec:intro}
%%%%%%%%%%%%%%%%%%%%%%%%%%%%%%%%%%%%%%%%%%%%%%%%%%%%%%%%%%%%%%%%%%%%%%%%%%%%%
\begin{itemize}
\item a few lines about the need for accurately resolving two-phase flows
\item background on the different two-phase flow models: 5, 6 and 7-equation two-phase flow models
\item then, focus on the different types of 7-equation two-phase flow models: they mostly differ because of the closure relaxations used
\item discuss the different numerical solvers developed for the 7-equation two-phase flow model: HLL, HLLC, and approximated Riemann solvers accounting for the source terms
\item emphasize the fact that the above numerical solvers only works on discontinuous schemes
\item then, introduce the entropy viscosity method and details the organization of the paper 
\end{itemize}
%
Compressible two-phase flows are found in numerous industrial applications and are an ongoing area of research in modeling and simulation over many years. A variety of models with different levels of complexity has been developed such as: five-equation model \cite{Kapila_2001}, six-equation model \cite{Toumi_1996}, and more recently the seven-equation model \cite{SEM}. These models are all obtained by integrating the single-phase flow balance equations weighed by a characteristic or indicator function for each phase. The resulting system of equations contains non-conservative terms that describe the interaction between phases but also an equation for the volume fraction. Once a system of equations describing the physics is derived, the next challenging step is to develop a robust and accurate discretization to obtain a numerical solution. Assuming that the system of equations is hyperbolic under some conditions, a Riemann solver could be used but is often ruled out because of the complexity due to the number of equations involved. Furthermore, careless approximation for the treatment of the non-conservative terms can lead to failure in computing the numerical solution \cite{Abgrall_2002}. An alternative is to use an approximate Riemann solver, a well-established approach for single-phase flows, while deriving a consistent discretization scheme for the non-conservative terms. 

This methodology was applied to the seven-equation model (SEM) introduced by Berry et al. in \cite{SEM}. This model is known to be unconditionally hyperbolic which is highly desirable when working with approximate Riemann solvers and can treat a wide range of applications. Its particularity comes from the pressure and velocity relaxation terms in the volume fraction, momentum and energy equations that can bring the two phases in equilibrium when using large values of the relaxation parameters. In other words, the seven-equation model can degenerate into the six- and five-equation models. Alike for the other two-phase flow models, solving for the seven-equation model requires a numerical solver and significant effort was dedicated to this task for spatially discontinuous schemes. Because each phase is assumed to obey the Euler equations, most of the numerical solvers are adapted from the single-phase approximate Riemann solvers. For example, Saurel et al. \cite{Saurel_2001a, Saurel_2001b} employed a HLL-type scheme to solve for the SEM but noted that excessive dissipation was added to the contact discontinuity. A more advanced HLLC-type scheme was developed in \cite{Li_2004} but only for the subsonic case and then extended to supersonic flows in \cite{Zein_2010}. More recently, Ambroso et al. \cite{Ambroso_2012} proposed an approximate Riemann solver accounting for source terms such as gravity and drag forces, but with no interphase mass transfer.
%%%%%%%%%%%%%%%%%%%%%%%%%%%%%%%%%%%%%%%%%%%%%%%%%%%%%%%%%%%%%%%%%%%%%%%%%%%%%
\section{The multi-D 7-equation two-phase flow model}\label{sec:7-equ-model}
%%%%%%%%%%%%%%%%%%%%%%%%%%%%%%%%%%%%%%%%%%%%%%%%%%%%%%%%%%%%%%%%%%%%%%%%%%%%%
\begin{itemize}
\item give the equations and detail the different terms
\item include the relaxation terms, the mass and heat exchange terms
\item eigenvalues
\item entropy equation WITHOUT the dissipative terms and five the details of the derivation in the appendix
\end{itemize}
%
The multi-D seven-equation model is obtained by assuming that each phase obeys the single-phase Euler equations (with phase-exchange terms) and by integrating over a control volume after multiplying by a characteristic function. The detailed derivation can be found in \cite{SEM}. In this section, the governing equations are recalled for each phase (liquid and vapor) and the source terms are described. 

The liquid phase obeys the following mass, momentum and energy balance equations, supplemented by a non-conservative volume-fraction equation:
%
\begin{subequations}\label{eq:liq-7-eqn-sect5}
\begin{align}
  % liquid mass conservation
  \label{multi-d-7-equ-liq}
  \frac{\partial \left( \alpha \rho \right)_{liq} A}{\partial t}
  + \div \left( \alpha \rho \mbold u A\right)_{liq}
  &= - \Gamma A_{int} A
\end{align}
\begin{align}
  % liquid momentum
  \frac{\partial \left( \alpha \rho \mbold u \right)_{liq} A}{\partial t}
  + \div \left[ \alpha_{liq} A \left( \rho \mbold u \otimes \mbold u + P \mathbb{I} \right)_{liq} \right]
  &= P_{int} A \grad \alpha_{liq} + P_{liq} \alpha_{liq} \grad A
    \nonumber
  \\
  &+ A \lambda_u (\mbold u_{vap} - \mbold u_{liq})
  - \Gamma A_{int} \mbold u_{int} A
\end{align}
\begin{align}
  % liquid total energy
  \frac{\partial \left( \alpha \rho E \right)_{liq} A}{\partial t}
  + \div \left[ \alpha_{liq} \mbold u_{liq} A \left( \rho E + P \right)_{liq} \right]
  &= P_{int} \mbold u_{int} A \grad \alpha_{liq} - \bar{P}_{int} A \mu_P (P_{liq} - P_{vap})
        \nonumber
  \\
  + \bar{\mbold u}_{int} A \lambda_u (\mbold u_{vap} - \mbold u_{liq})
&  + \Gamma A_{int} \left( \frac{P_{int}}{\rho_{int}} - H_{liq, int} \right) A
\nonumber 
\\
& + Q_{wall,liq} + Q_{int,liq}
\end{align}
\begin{align}
  % liquid volume fraction
  \label{eqn:multi-d-7-eqn-liq-vol}
  \frac{\partial \alpha_{liq} A}{\partial t} + A\mbold u_{int} \cdot \grad \alpha_{liq}
  &= A \mu_P (P_{liq} - P_{vap}) - \frac{\Gamma A_{int} A}{\rho_{int}}
\end{align}
\end{subequations}
%
On the same model, the equations for the vapor phase are:
%
\begin{subequations}\label{eq:vap-7-eqn-sect5}
\begin{align}
  \label{multi-d-7-equ-vap}
  % vapor mass conservation
  \frac{\partial \left( \alpha \rho A\right)_{vap}}{\partial t}
  + \div \left( \alpha \rho \mbold u \right)_{vap} A
  =  \Gamma A_{int} A
\end{align}
\begin{align}
  % vapor momentum
  \frac{\partial \left( \alpha \rho u \right)_{vap} A}{\partial t}
  + \div \left[ \alpha_{vap} A \left( \rho \mbold u \otimes \mbold u + P\mathbb{I} \right)_{vap} \right]
  &= P_{int} A \grad \alpha_{vap} + P_{vap} \alpha_{vap} \grad A
  \\
  \nonumber
  &+ A \lambda_u (\mbold u_{liq} - \mbold u_{vap})
  + \Gamma A_{int} u_{int} A
\end{align}
\begin{align}
  % vapor total energy
  \frac{\partial \left( \alpha \rho E \right)_{vap} A}{\partial t}
  + \div \left[ \alpha_{vap} \mbold u_{vap} A \left( \rho E + P \right)_{vap} \right]
  &= P_{int} \mbold u_{int} A \grad \alpha_{vap} - \bar{P}_{int} A \mu_P (P_{vap} - P_{liq})
    \nonumber
  \\
  + \bar{\mbold u}_{int} A \lambda_u (\mbold u_{liq} - \mbold u_{vap})
&- \Gamma A_{int} \left( \frac{P_{int}}{\rho_{int}} - H_{vap, int} \right) A
\nonumber 
\\
& + Q_{wall,vap} + Q_{int,vap}
\end{align}
\begin{align}
  % vapor phase volume fraction
  \label{eqn:multi-d-7-eqn-vap-vol}
  \frac{\partial \alpha_{vap} A}{\partial t} + A \mbold u_{int} \cdot \grad \alpha_{vap}
  &= A \mu_P (P_{vap} - P_{liq}) + \frac{\Gamma A_{int} A}{\rho_{int}}
\end{align}
\end{subequations}
%
where $\alpha_k$, $\rho_k$, $\mbold u_k$ and $E_k$ denote the volume fraction, the density, the velocity vector and the total specific energy of phase $k=\left\{ liq, vap \right\}$, respectively. The phase pressure $P_k$ is computed from an equation of state. The interfacial variables are denoted by the subscript $int$ and their definition will be given in \sct{sec:source-terms-7-eqt-sect5}. The interfacial pressure and velocity and their corresponding average values are denoted by $P_{int}$, $\mbold u_{int}$, $\bar{P}_{int}$ and $\bar{\mbold u}_{int}$, respectively. $\Gamma$ is the net mass transfer rate per unit interfacial area from the liquid to the vapor phase and $A_{int}$ is the interfacial area per unit volume of mixture.  Also, $H_{liq, int}$ and $H_{vap, int}$ are the liquid and gas total specific enthalpies at the interface, respectively, with the following definition: $H_k = h_k + 0.5 || \mbold u ||^2$. $\mu_P$ is the pressure relaxation coefficient and $\lambda_u$ denotes the velocity relaxation coefficient. The wall and interfacial heat sources are denoted by $Q_{wall,k}$ and $Q_{int,k}$, respectively, and are detailed in \sct{sec:source-terms-7-eqt-sect5}. Lastly, the cross section $A$ is assumed spatially dependent. In the case of two-phase flows, the equation for the vapor volume fraction, \eqt{eqn:multi-d-7-eqn-vap-vol}, is simply replaced by the algebraic relation
%
\begin{align}
 \alpha_{vap}= 1 - \alpha_{liq}
\end{align}
%
\begin{align}
  \label{E-R:83}
  P_{int} &= \bar{P}_{int} + \frac{Z_{liq}Z_{vap}}{Z_{liq}+Z_{vap}} \frac{\grad \alpha_{liq}}{|| \grad \alpha_{liq} ||} \cdot (\mbold u_{vap}-\mbold u_{liq})
  \\
  \bar{P}_{int} &= \frac{Z_{vap}P_{liq}+Z_{liq}P_{vap}}{Z_{liq}+Z_{vap}}
\end{align}
%
The interfacial velocities $\mbold u_{int}$ and its average value $\bar{\mbold u}_{int}$ are computed from:
%
\begin{align}
  \label{E-R:84}
  \mbold u_{int} &= \bar{\mbold u}_{int} +  \frac{\grad \alpha_{liq}}{|| \grad \alpha_{liq} ||} \frac{P_{vap}-P_{liq}}{Z_{liq}+Z_{vap}}
  \\
  \bar{\mbold u}_{int} &= \frac{Z_{liq} \mbold u_{liq}+Z_{vap}\mbold u_{vap}}{Z_{liq}+Z_{vap}}.
\end{align}
%
The pressure, $\mu_P$, and velocity, $\lambda_u$, relaxation coefficients are proportional to each other and function of the interfacial area $A_{int}$:
%
\begin{align}
  \label{E-R:85}
  \lambda_u &= \frac{1}{2} \mu_P Z_{liq} Z_{vap}
  \\
  \label{E-R:86}
  \mu_P &= \frac{A_{int}}{Z_{liq}+Z_{vap}}
\end{align}
%
The specific interfacial area (i.e., the interfacial surface area per unit
volume of two-phase mixture), $A_{int}$, must be specified from some type of
flow regime map or function under the form of a correlation. In \cite{SEM}, $A_{int}$ is chosen to be a function of the liquid volume fraction:
%
\begin{equation}\label{eq:Aint-sect4}
A_{int} = A_{int}^{max} \left[ 6.75 \left(1-\alpha_{liq} \right)^2 \alpha_{liq} \right],
\end{equation}
% 
where $A_{int}^{max} = 5100$ $m^2 / m^3$. With such definition, the interfacial area is zero in the limits $\alpha_{liq} = 0$ and $\alpha_{liq} = 1$.
To relax the seven-equation model to
the ill-posed classical six-equation model, only the pressures should be
relaxed toward a single pressure for both phases.  This is
accomplished by specifying the pressure relaxation coefficient to be
very large, i.e., letting it approach infinity.  But if the pressure
relaxation coefficient goes to infinity, so does the velocity
relaxation rate also approach infinity.  This then relaxes the
seven-equation model not to the classical six-equation model but to the
mechanical equilibrium five-equation model of Kapila \cite{Kapila_2001}.  This reduced
five-equation model is also hyperbolic and well-posed. The five-equation
model provides a very useful starting point for constructing
multi-dimensional interface resolving methods which dynamically
captures evolving and spontaneously generated
interfaces~\cite{Saurel_2009}. Thus the seven-equation model
can be relaxed locally to couple seamlessly with such a
multi-dimensional, interface resolving code.

Numerically, the mechanical relaxation coefficients $\mu_P$
(pressure) and $\lambda_u$ (velocity) can be relaxed independently to
yield solutions to useful, reduced models (as explained previously).  It
is noted, however, that relaxation of pressure only by making $\mu_P$
large without relaxing velocity will indeed give ill-posed and
unstable numerical solutions, just as the classical six-equation
two-phase model does, with sufficiently fine spatial resolution, as
confirmed in~\cite{SEM,Herrard_2005}.

Even though the implementation of the seven-equation two-phase
model does not use
the generalized approach of DEM \cite{SEM}, the interfacial pressure and velocity
closures as well as the pressure and velocity relaxation coefficients
of Equations~\eqref{E-R:83} to~\eqref{E-R:86} are utilized. \\

A simple expression for the interphase
mass flow rate is obtained from \cite{SEM}:
\begin{align}
  \nonumber
  \Gamma = \Gamma_{vap}
  &= \frac{h_{T,  liq} \left( T_{liq} - T_{int} \right) + h_{T,  vap} \left( T_{vap} - T_{int} \right)}{h_{vap,  int} - h_{liq,  int}}
  \\
  &= \frac{h_{T,  liq} \left( T_{liq} - T_{int} \right) + h_{T,  vap} \left( T_{vap} - T_{int} \right)}{L_v \left( T_{int} \right)}
\end{align}
where $L_v \left( T_{int} \right) = h_{vap,  int} - h_{liq,  int}$
represents the latent heat of vaporization.  The interface
temperature is determined by the saturation constraint
$T_{int}=T_{sat}(P)$ with the appropriate pressure $P=\bar{P}_{int}$
determined above, the interphase mass flow rate is thus determined.\\

The set of eight equations given in \eqt{eq:liq-7-eqn-sect5} and in \eqt{eq:vap-7-eqn-sect5} is now reduced to seven which yields the multi-D seven-equation model. A set of seven waves is present in such a model: two acoustic waves and a contact wave for each phase supplanted by a volume fraction wave propagating at the interfacial velocity $\mbold u_{int}$. Considering a domain of dimension $\mathbb{D}$, the corresponding eigenvalues are the following for each phase $k$:
% 
\begin{align}
&\lambda_1 = \mbold u_{int} \cdot \bar{\mbold n} \nonumber\\
&\lambda_{2,k} = \mbold u_k \cdot \bar{\mbold n} - c_k \nonumber\\
&\lambda_{3,k} = \mbold u_k \cdot \bar{\mbold n} + c_k \nonumber\\
&\lambda_{d+3,k} = \mbold u_k \cdot \bar{\mbold n} \text{ for } d = 1 \dots \mathbb{D},\nonumber
\end{align}
%
where $\bar{\mbold n}$ is a unit vector pointing to a given direction.
For each phase $k$, an entropy equation can be derived when accounting only for the pressure and velocity relaxation terms (all of the terms proportional to the net mass transfer term $\Gamma$ and the interfacial heat transfer $Q_{int,k}$ are removed). The entropy function for a phase $k$ is denoted by $s_k$ and function of the density $\rho_k$ and the internal energy $e_k$. The derivation is detailed in \app{app:sev-equ-model-entropy} and only the final result is recalled here when assuming that the phase $k$ is in interaction with a phase $j$:
%
\begin{align}\label{eq:ent-eqn-7-eqn-model}
(s_{e})_k^{-1} \alpha_k \rho_k A \frac{Ds_k}{Dt} &= \mu_P \frac{Z_k}{Z_k+Z_j} (P_j - P_k)^2 + \lambda_u \frac{Z_j}{Z_k+Z_j} (\mbold u_j -\mbold  u_k)^2 \nonumber
\\
& \frac{Z_k}{\left( Z_k+Z_j \right)^2} \left[ Z_j (\mbold u_j-\mbold u_k)+\frac{\grad \alpha_k}{|| \grad \alpha_k ||}(P_k-P_j)\right]^2,
\end{align}
where $Z_{k}$ denotes the phasic acoustic impedance and is defined as the product of the density and the speed of sound: $Z_k = \rho_k c_k$. The partial derivative of the entropy function $s_k$ with respect to the internal energy $e_k$, $(s_e)_k$, is defined proportional to the inverse of the temperature of phase $k$ as for the single phase Euler equations. The right hand-side of \eqt{eq:ent-eqn-7-eqn-model} is unconditionally positive since all terms are squared. Furthermore, \eqt{eq:ent-eqn-7-eqn-model} is valid for each phase $k=\left\{liq, vap \right\}$ and ensures positivity of the total entropy equation that is obtained by summing over the phases:
%
\begin{equation}\label{eq:tot-ent-res-sct4}
\sum_k (s_{e})_k^{-1} \alpha_k \rho_k A \frac{Ds_k}{Dt} = \sum_k (s_{e})_k^{-1} \alpha_k \rho_k A \left( \partial_t s_k + \mbold u_k \cdot \grad s_k \right) \geq 0.
\end{equation}
%
Note that when one phase disappears, \eqt{eq:tot-ent-res-sct4} degenerates into the single phase entropy equation.
%%%%%%%%%%%%%%%%%%%%%%%%%%%%%%%%%%%%%%%%%%%%%%%%%%%%%%%%%%%%%%%%%%%%%%%%%%%%%
\section{A viscous regularization for the multi-D 7-equation two-phase flow model}\label{sec:visc-regu}
%%%%%%%%%%%%%%%%%%%%%%%%%%%%%%%%%%%%%%%%%%%%%%%%%%%%%%%%%%%%%%%%%%%%%%%%%%%%%
\begin{itemize}
\item explain why we work with the phase entropy equation instead of considering the total entropy residual by summing over the two phases
\item viscous regularization must be consistent with single-phase flow equation
\item recall the notion of entropy condition and entropy inequality $\to$ require dissipative terms in order to get a sign
\item give the system of equations with the dissipative terms
\item guide the reader through the derivation of the dissipative terms
\item give the entropy residual with all terms in the right hand-side
\item make the link with the single-phase flow equations
\item explain how to derive the dissipative term for the volume fraction equation
\item emphasizes the fact that the regularization is valid for any EOS with convex entropy
\item a few words about the parabolic regularization
\end{itemize}
%%%%%%%%%%%%%%%%%%%%%%%%%%%%%%%%%%%%%%%%%%%%%%%%%%%%%%%%%%%%%%%%%%%%%%%%%%%%%
\section{A definition of the viscosity coefficients for all Mach flows}\label{sec:low-Mach}
%%%%%%%%%%%%%%%%%%%%%%%%%%%%%%%%%%%%%%%%%%%%%%%%%%%%%%%%%%%%%%%%%%%%%%%%%%%%%
\begin{itemize}
\item non-dimensionalize the equations but use $P_\infty$ for the pressure instead of $(\rho c^2)_\infty$
\item introduce a new Pechlet number for $\beta$: its behavior should be the same as the Pechlet number for $\kappa$
\item two cases: zero and infinite relaxation coefficients
\item derive the normalization parameters for the isentropic and non-isentropic flows
\item discussion about the 
\end{itemize}
%%%%%%%%%%%%%%%%%%%%%%%%%%%%%%%%%%%%%%%%%%%%%%%%%%%%%%%%%%%%%%%%%%%%%%%%%%%%%
\section{$1$-D numerical results}\label{sec:results}
%%%%%%%%%%%%%%%%%%%%%%%%%%%%%%%%%%%%%%%%%%%%%%%%%%%%%%%%%%%%%%%%%%%%%%%%%%%%%
\begin{itemize}
\item simple advection problem
\item shock tube with two independent fluids: exact solution and could do convergence test for this particular test
\item shock tube with infinite relaxation coefficients
\item $1$-D nozzle with two independent fluids
\item $1$-D nozzle with infinite relaxation coefficients
\item $1$-D nozzle with infinite relaxation coefficients,  mass and heat transfer
\end{itemize}
%%%%%%%%%%%%%%%%%%%%%%%%%%%%%%%%%%%%%%%%%%%%%%%%%%%%%%%%%%%%%%%%%%%%%%%%%%%%%
\bibliography{mybibfile}
%%%%%%%%%%%%%%%%%%%%%%%%%%%%%%%%%%%%%%%%%%%%%%%%%%%%%%%%%%%%%%%%%%%%%%%%%%%%%
\appendix
\end{document}