%%
%% This is file `elsarticle-template-num.tex',
%% generated with the docstrip utility.
%%
%% The original source files were:
%%
%% elsarticle.dtx  (with options: `numtemplate')
%% 
%% Copyright 2007, 2008 Elsevier Ltd.
%% 
%% This file is part of the 'Elsarticle Bundle'.
%% -------------------------------------------
%% 
%% It may be distributed under the conditions of the LaTeX Project Public
%% License, either version 1.2 of this license or (at your option) any
%% later version.  The latest version of this license is in
%%    http://www.latex-project.org/lppl.txt
%% and version 1.2 or later is part of all distributions of LaTeX
%% version 1999/12/01 or later.
%% 
%% The list of all files belonging to the 'Elsarticle Bundle' is
%% given in the file `manifest.txt'.
%% 

%% Template article for Elsevier's document class `elsarticle'
%% with numbered style bibliographic references
%% SP 2008/03/01

%\documentclass[preprint,12pt]{elsarticle}
\documentclass[preprint,10pt]{elsarticle}
%\documentclass[final,3p,times]{elsarticle} 

%% Use the option review to obtain double line spacing
%% \documentclass[authoryear,preprint,review,12pt]{elsarticle}

%% Use the options 1p,twocolumn; 3p; 3p,twocolumn; 5p; or 5p,twocolumn
%% for a journal layout:
%% \documentclass[final,1p,times]{elsarticle}
%% \documentclass[final,1p,times,twocolumn]{elsarticle}
%% \documentclass[final,3p,times]{elsarticle}
%% \documentclass[final,3p,times,twocolumn]{elsarticle}
%% \documentclass[final,5p,times]{elsarticle}
%% \documentclass[final,5p,times,twocolumn]{elsarticle}

%% if you use PostScript figures in your article
%% use the graphics package for simple commands
\usepackage{float}
\usepackage{color}
\usepackage{caption}
\usepackage{subcaption}
\usepackage{appendix}
%% or use the graphicx package for more complicated commands
\usepackage{graphicx}
%% or use the epsfig package if you prefer to use the old commands
%% \usepackage{epsfig}

%% The amssymb package provides various useful mathematical symbols 
%% The amsthm package provides extended theorem environments
\usepackage{amssymb}
\usepackage{amsmath}
% more math
\usepackage{amsfonts}
\usepackage{amstext}
\usepackage{amsbsy}
\usepackage{mathbbol} 
%% The lineno packages adds line numbers. Start line numbering with
%% \begin{linenumbers}, end it with \end{linenumbers}. Or switch it on
%% for the whole article with \linenumbers.
\usepackage{lineno}

\journal{Journal of Comp. Phys.}
%%%%%%%%%%%%%%%%%%%%%%%%%%%%%%%%%%%%%%%%%%%%%%%%%%%%%%%%%%%%%%%%%%%%
% operators
\renewcommand{\div}{\vec{\nabla}\! \cdot \!}
\newcommand{\grad}{\vec{\nabla}}
% latex shortcuts
\newcommand{\bea}{\begin{eqnarray}}
\newcommand{\eea}{\end{eqnarray}}
\newcommand{\be}{\begin{equation}}
\newcommand{\ee}{\end{equation}}
\newcommand{\bal}{\begin{align}}
\newcommand{\eali}{\end{align}}
\newcommand{\bi}{\begin{itemize}}
\newcommand{\ei}{\end{itemize}}
\newcommand{\ben}{\begin{enumerate}}
\newcommand{\een}{\end{enumerate}}
% DGFEM commands
\newcommand{\jmp}[1]{[\![#1]\!]}                     % jump
\newcommand{\mvl}[1]{\{\!\!\{#1\}\!\!\}}             % mean value
\newcommand{\keff}{\ensuremath{k_{\textit{eff}}}\xspace}
% shortcut for domain notation
\newcommand{\D}{\mathcal{D}}
% vector shortcuts
\newcommand{\vo}{\vec{\Omega}}
\newcommand{\vr}{\vec{r}}
\newcommand{\vn}{\vec{n}}
\newcommand{\vnk}{\vec{\mathbf{n}}}
\newcommand{\vj}{\vec{J}}
\newcommand{\eig}[1]{\| #1 \|_2}

\newcommand{\EI}{\mathcal{E}_h^i}
\newcommand{\ED}{\mathcal{E}_h^{\partial \D^d}}
\newcommand{\EN}{\mathcal{E}_h^{\partial \D^n}}
\newcommand{\ER}{\mathcal{E}_h^{\partial \D^r}}
\newcommand{\reg}{\textit{reg}}

% extra space
\newcommand{\qq}{\quad\quad}
% common reference commands
\newcommand{\eqt}[1]{Eq.~(\ref{#1})}                     % equation
\newcommand{\fig}[1]{Fig.~\ref{#1}}                      % figure
\newcommand{\tbl}[1]{Table~\ref{#1}}                     % table
\newcommand{\sct}[1]{Section~\ref{#1}}                   % section
\newcommand{\app}[1]{Appendix~\ref{#1}}                   % appendix

\newcommand\br{\mathbf{r}}
%\newcommand{\tf}{\varphi}
\newcommand{\tf}{b}

\newcommand{\tcr}[1]{\textcolor{red}{#1}}
\newcommand{\mt}[1]{\marginpar{ {\tiny #1}}}

\bibliographystyle{elsarticle-num}
%%%%%%%%%%%%%%%%%%%%%%%%%%%%%%%%%%%%%%%%%%%%%%%%%%%%%%%%%%%%%%%%%%%%%
%
%   BEGIN DOCUMENT
%
%%%%%%%%%%%%%%%%%%%%%%%%%%%%%%%%%%%%%%%%%%%%%%%%%%%%%%%%%%%%%%%%%%%%%
\begin{document}

%%%%%%%%%%%%%%%%%%%%%%%%%%%%%%%%%%%%%%%%%%%%%%%%%%%%%%%%%%%%%%%%%%%%
\begin{frontmatter}

%% Title, authors and addresses

%% use the tnoteref command within \title for footnotes;
%% use the tnotetext command for theassociated footnote;
%% use the fnref command within \author or \address for footnotes;
%% use the fntext command for theassociated footnote;
%% use the corref command within \author for corresponding author footnotes;
%% use the cortext command for theassociated footnote;
%% use the ead command for the email address,
%% and the form \ead[url] for the home page:
%\title{Title\tnoteref{label1}}
%% \tnotetext[label1]{}
%% \author{Name\corref{cor1}\fnref{label2}}
%% \ead{email address}
%% \ead[url]{home page}
%% \fntext[label2]{}
%% \cortext[cor1]{}
%% \address{Address\fnref{label3}}
%% \fntext[label3]{}
%-------------------------
%-------------------------
\title{Extension of the entropy viscosity method to the low Mach regime for the multi-D Euler equations (with variable area).}
%-------------------------
%-------------------------
\author{Marc O. Delchini\fnref{label1}}
\ead{delchmo@tamu.edu}

\author{Jean C. Ragusa\corref{cor1}\fnref{label1}}
\ead{jean.ragusa@tamu.edu}

\author{Ray A. Berry\fnref{label2}}
\ead{ray.berry@inl.gov}

\address[label1]{Department of Nuclear Engineering, Texas A\&M University, College Station, TX 77843, USA \fnref{label1}}

\address[label2]{Idaho National Laboratory, Idaho Falls, ID 83415, USA \fnref{label2}}

\cortext[cor1]{Corresponding author}
%-------------------------
%-------------------------
%-------------------------
\begin{abstract}
blabla
\end{abstract}
%-------------------------
%-------------------------
\begin{keyword}
  Euler equations with variable area \sep entropy viscosity method \sep stabilization method \sep low Mach regime \sep shocks.
\end{keyword}
%-------------------------
\end{frontmatter}
%%%%%%%%%%%%%%%%%%%%%%%%%%%%%%%%%%%%%%%%%%%%%%%%%%%%%%%%%%%%%%%%%%%%
\linenumbers
%%%%%%%%%%%%%%%%%%%%%%%%%%%%%%%%%%%%%%%%%%%%%%%%%%%%%%%%%%%%%%%%%%%%%%%%%%%%%%%%%%%%%%%%%%%%%%%%%%%%
%%%%%%%%%%%%%%%%%%%%%%%%%%%%%%%%%%%%%%%%%%%%%%%%%%%%%%%%%%%%%%%%%%%%%%%%%%%%%%%%%%%%%%%%%%%%%%%%%%%%
\section{Introduction} \label{sec:intro}
%%%%%%%%%%%%%%%%%%%%%%%%%%%%%%%%%%%%%%%%%%%%%%%%%%%%%%%%%%%%%%%%%%%%%%%%%%%%%%%%%%%%%%%%%%%%%%%%%%%%
%%%%%%%%%%%%%%%%%%%%%%%%%%%%%%%%%%%%%%%%%%%%%%%%%%%%%%%%%%%%%%%%%%%%%%%%%%%%%%%%%%%%%%%%%%%%%%%%%%%%
\end{document}
%%%%%%%%%%%%%%%%%%%%%%%%%%%%%%%%%%%%%%%%%%%%%%%%%%%%%%%%%%%%%%%%%%%%%%%%%%%%%%%%%%%%%%%%%%%%%%%%%%%%
%%%%%%%%%%%%%%%%%%%%%%%%%%%%%%%%%%%%%%%%%%%%%%%%%%%%%%%%%%%%%%%%%%%%%%%%%%%%%%%%%%%%%%%%%%%%%%%%%%%%
