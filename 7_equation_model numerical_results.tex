%%
%% This is file `elsarticle-template-num.tex',
%% generated with the docstrip utility.
%%
%% The original source files were:
%%
%% elsarticle.dtx  (with options: `numtemplate')
%% 
%% Copyright 2007, 2008 Elsevier Ltd.
%% 
%% This file is part of the 'Elsarticle Bundle'.
%% -------------------------------------------
%% 
%% It may be distributed under the conditions of the LaTeX Project Public
%% License, either version 1.2 of this license or (at your option) any
%% later version.  The latest version of this license is in
%%    http://www.latex-project.org/lppl.txt
%% and version 1.2 or later is part of all distributions of LaTeX
%% version 1999/12/01 or later.
%% 
%% The list of all files belonging to the 'Elsarticle Bundle' is
%% given in the file `manifest.txt'.
%% 

%% Template article for Elsevier's document class `elsarticle'
%% with numbered style bibliographic references
%% SP 2008/03/01

%\documentclass[preprint,12pt]{elsarticle}
\documentclass[preprint,10pt]{elsarticle}
%\documentclass[final,3p,times]{elsarticle} 

%% Use the option review to obtain double line spacing
%% \documentclass[authoryear,preprint,review,12pt]{elsarticle}

%% Use the options 1p,twocolumn; 3p; 3p,twocolumn; 5p; or 5p,twocolumn
%% for a journal layout:
%% \documentclass[final,1p,times]{elsarticle}
%% \documentclass[final,1p,times,twocolumn]{elsarticle}
%% \documentclass[final,3p,times]{elsarticle}
%% \documentclass[final,3p,times,twocolumn]{elsarticle}
%% \documentclass[final,5p,times]{elsarticle}
%% \documentclass[final,5p,times,twocolumn]{elsarticle}

%% if you use PostScript figures in your article
%% use the graphics package for simple commands
\usepackage{float}
\usepackage{color}
\usepackage{caption}
\usepackage{subcaption}
\usepackage[titletoc,title]{appendix}
%% or use the graphicx package for more complicated commands
\usepackage{graphicx}
%% or use the epsfig package if you prefer to use the old commands
%% \usepackage{epsfig}

%% The amssymb package provides various useful mathematical symbols 
%% The amsthm package provides extended theorem environments
\usepackage{amssymb}
\usepackage{amsmath}
\usepackage{mathrsfs}
% more math
\usepackage{amsfonts}
\usepackage{amstext}
\usepackage{amsbsy}
\usepackage{mathbbol} 
%% The lineno packages adds line numbers. Start line numbering with
%% \begin{linenumbers}, end it with \end{linenumbers}. Or switch it on
%% for the whole article with \linenumbers.
\usepackage{lineno}

\journal{Journal of Comp. Phys.}
%%%%%%%%%%%%%%%%%%%%%%%%%%%%%%%%%%%%%%%%%%%%%%%%%%%%%%%%%%%%%%%%%%%%
% operators
\renewcommand{\div}{\mbold{\nabla}\! \cdot \!}
\newcommand{\grad}{\mbold{\nabla}}
\newcommand{\divv}[1]{\boldsymbol{\nabla}^{#1}\! \cdot \!}
\newcommand{\gradd}[1]{\mbold{\nabla}^{#1}}
\newcommand{\mbold}[1]{\boldsymbol#1}
% latex shortcuts
\newcommand{\bea}{\begin{eqnarray}}
\newcommand{\eea}{\end{eqnarray}}
\newcommand{\be}{\begin{equation}}
\newcommand{\ee}{\end{equation}}
\newcommand{\bal}{\begin{align}}
\newcommand{\eali}{\end{align}}
\newcommand{\bi}{\begin{itemize}}
\newcommand{\ei}{\end{itemize}}
\newcommand{\ben}{\begin{enumerate}}
\newcommand{\een}{\end{enumerate}}
\usepackage{amsthm}
\newtheorem*{remark}{Remark}
% DGFEM commands
\newcommand{\jmp}[1]{[\![#1]\!]}                     % jump
\newcommand{\mvl}[1]{\{\!\!\{#1\}\!\!\}}             % mean value
\newcommand{\keff}{\ensuremath{k_{\textit{eff}}}\xspace}
% shortcut for domain notation
\newcommand{\D}{\mathcal{D}}
% vector shortcuts
\newcommand{\vo}{\mbold{\Omega}}
\newcommand{\vr}{\mbold{r}}
\newcommand{\vn}{\mbold{n}}
\newcommand{\vnk}{\mbold{\mathbf{n}}}
\newcommand{\vj}{\mbold{J}}
\newcommand{\eig}[1]{\| #1 \|_2}
%
\newcommand{\EI}{\mathcal{E}_h^i}
\newcommand{\ED}{\mathcal{E}_h^{\partial \D^d}}
\newcommand{\EN}{\mathcal{E}_h^{\partial \D^n}}
\newcommand{\ER}{\mathcal{E}_h^{\partial \D^r}}
\newcommand{\reg}{\textit{reg}}
%
\newcommand{\norm}{\textrm{norm}}
\renewcommand{\Re}{\textrm{Re}}
\newcommand{\Pe}{\textrm{P\'e}}
\renewcommand{\Pr}{\textrm{Pr}}
%
\newcommand{\resi}{R}
%\newcommand{\resinew}{\tilde{D}_e}
\newcommand{\resinew}{\widetilde{\resi}}
\newcommand{\resisource}{\widetilde{\resi}^{source}}
\newcommand{\matder}[1]{\frac{\textrm{D} #1}{\textrm{D} t}}
%
% extra space
\newcommand{\qq}{\quad\quad}
% common reference commands
\newcommand{\eqt}[1]{Eq.~(\ref{#1})}                     % equation
\newcommand{\fig}[1]{Fig.~\ref{#1}}                      % figure
\newcommand{\tbl}[1]{Table~\ref{#1}}                     % table
\newcommand{\sct}[1]{Section~\ref{#1}}                   % section
\newcommand{\app}[1]{Appendix~\ref{#1}}                   % appendix
%
\newcommand{\ie}{i.e.,\@\xspace}
\newcommand{\eg}{e.g.,\@\xspace}
\newcommand{\psc}[1]{{\sc {#1}}}
\newcommand{\rs}{\psc{R7}\xspace}
%
\newcommand\br{\mathbf{r}}
%\newcommand{\tf}{\varphi}
\newcommand{\tf}{b}
%
\newcommand{\tcr}[1]{\textcolor{red}{#1}}
\newcommand{\tcb}[1]{\textcolor{blue}{#1}}
\newcommand{\mt}[1]{\marginpar{ {\tiny #1}}}
%
\bibliographystyle{elsarticle-num}
%%%%%%%%%%%%%%%%%%%%%%%%%%%%%%%%%%%%%%%%%%%%%%%%%%%%%%%%%%%%%%%%%%%%%
%
%   BEGIN DOCUMENT
%
%%%%%%%%%%%%%%%%%%%%%%%%%%%%%%%%%%%%%%%%%%%%%%%%%%%%%%%%%%%%%%%%%%%%%
\begin{document}

%%%%%%%%%%%%%%%%%%%%%%%%%%%%%%%%%%%%%%%%%%%%%%%%%%%%%%%%%%%%%%%%%%%%
\begin{frontmatter}

%% Title, authors and addresses

%% use the tnoteref command within \title for footnotes;
%% use the tnotetext command for theassociated footnote;
%% use the fnref command within \author or \address for footnotes;
%% use the fntext command for theassociated footnote;
%% use the corref command within \author for corresponding author footnotes;
%% use the cortext command for theassociated footnote;
%% use the ead command for the email address,
%% and the form \ead[url] for the home page:
%\title{Title\tnoteref{label1}}
%% \tnotetext[label1]{}
%% \author{Name\corref{cor1}\fnref{label2}}
%% \ead{email address}
%% \ead[url]{home page}
%% \fntext[label2]{}
%% \cortext[cor1]{}
%% \address{Address\fnref{label3}}
%% \fntext[label3]{}
%-------------------------
%-------------------------
\title{$1$-D numerical results of the seven-equation two-phase flow model with the entropy viscosity method.}
%-------------------------
%-------------------------
\author{Marc O. Delchini\fnref{label1}}
\ead{delchmo@tamu.edu}

\author{Jean C. Ragusa\corref{cor1}\fnref{label1}}
\ead{jean.ragusa@tamu.edu}

\author{Ray A. Berry\fnref{label2}}
\ead{ray.berry@inl.gov}

\address[label1]{Department of Nuclear Engineering, Texas A\&M University, College Station, TX 77843, USA \fnref{label1}}

\address[label2]{Idaho National Laboratory, Idaho Falls, ID 83415, USA \fnref{label2}}

\cortext[cor1]{Corresponding author}
%-------------------------
%-------------------------
%-------------------------
\begin{abstract}
blabla
\end{abstract}
%-------------------------
%-------------------------
\begin{keyword}
  two-phase flow model \sep with variable area \sep entropy viscosity method \sep stabilization method \sep low Mach regime \sep shocks.
\end{keyword}
%-------------------------
\end{frontmatter}
\linenumbers
%%%%%%%%%%%%%%%%%%%%%%%%%%%%%%%%%%%%%%%%%%%%%%%%%%%%%%%%%%%%%%%%%%%%%%%%%%%%%
\section{Introduction}\label{sec:intro}
%%%%%%%%%%%%%%%%%%%%%%%%%%%%%%%%%%%%%%%%%%%%%%%%%%%%%%%%%%%%%%%%%%%%%%%%%%%%%
\begin{itemize}
\item a few lines about the need for accurately resolving two-phase flows
\item background on the different two-phase flow models: 5, 6 and 7-equation two-phase flow models
\item then, focus on the different types of 7-equation two-phase flow models: they mostly differ because of the closure relaxations used
\item discuss the different numerical solvers developed for the 7-equation two-phase flow model: HLL, HLLC, and approximated Riemann solvers accounting for the source terms
\item emphasize the fact that the above numerical solvers only works on discontinuous schemes
\item then, introduce the entropy viscosity method and details the organization of the paper 
\end{itemize}
%%%%%%%%%%%%%%%%%%%%%%%%%%%%%%%%%%%%%%%%%%%%%%%%%%%%%%%%%%%%%%%%%%%%%%%%%%%%%
%%%%%%%%%%%%%%%%%%%%%%%%%%%%%%%%%%%%%%%%%%%%%%%%%%%%%%%%%%%%%%%%%%%%%%%%%%%%%
\section{The seven-equation two-phase flow model with an all-Mach flow viscous regularization}\label{sec:7-equ-model}
%%%%%%%%%%%%%%%%%%%%%%%%%%%%%%%%%%%%%%%%%%%%%%%%%%%%%%%%%%%%%%%%%%%%%%%%%%%%%
%%%%%%%%%%%%%%%%%%%%%%%%%%%%%%%%%%%%%%%%%%%%%%%%%%%%%%%%%%%%%%%%%%%%%%%%%%%%%
\begin{itemize}
\item give the equations and detail the different terms with viscous regularization.
\item include the relaxation terms, the mass and heat exchange terms.
\item eigenvalues.
\item entropy equation with the dissipative terms and five the details of the derivation in the appendix.
\item recall the definition of the viscosity coefficients for all-Mach flows.
\end{itemize}
%%------------------------------------------------------------------------------------------------------
%\subsection{Definition of the viscosity coefficients}\label{sec:visc-coeff-sem}
%%------------------------------------------------------------------------------------------------------
%In the entropy viscosity method, each viscosity coefficient is function of an upper and a lower bound that are referred to as first-order viscosity coefficient and entropy viscosity coefficient (high-order coefficient), respectively, as shown in \eqt{eq:def-visc-sem-sct4}. The first-order viscosity coefficient is denoted by the subscript $max$ and is defined proportional to the largest local eigenvalue so that the stabilization scheme becomes over-dissipative and smooth out all discontinuities when the entropy residual is large. The entropy viscosity coefficient is set proportional to an entropy residual and jumps of quantities to determine, and denoted by the subscript $e$. 
%%
%\begin{align}\label{eq:def-visc-sem-sct4}
%\beta_k( \mbold r, t) = \min \left( \beta_{e,k}( \mbold r, t), \beta_{max,k} ( \mbold r, t) \right), \nonumber \\
%\mu_k( \mbold r, t) = \min \left( \mu_{e,k}( \mbold r, t), \mu_{max,k} ( \mbold r, t) \right),  \\
%\kappa_k( \mbold r, t) = \min \left( \kappa_{e,k}( \mbold r, t), \kappa_{max,k} ( \mbold r, t) \right) ,\nonumber 
%\end{align}
%% 
%where all of the variables are locally defined. We now define the first-order viscosity coefficients and will focus first on the phasic viscosity coefficients $\kappa_k$ and $\mu_k$ that are intimately linked to the mass, momentum and energy equations. These two viscosity coefficients are involved in dissipative terms that identical to the ones obtained for the single-phase Euler equations \cite{jlg, Marco_paper_low_mach} when seeing the term $\alpha_k A$ as a pseudo cross-section and assuming an uniform void fraction profile. Thus, it is chosen to define the corresponding first-order viscosity coefficients proportional to the local largest eigenvalue $|| \mbold u_k || + c_k$ as follows:
%%
%\begin{equation}\label{eq:def-visc-max-sem-sct4}
%\kappa_{max,k}( \mbold r, t) = \mu_{max,k}( \mbold r, t) = \frac{h}{2} \left( || \mbold u_k||( \mbold r, t) + c_k( \mbold r, t) \right),
%\end{equation}
%%
%where $h$ is the grid size (each phase is solved on the same mesh). It remains to define the first-order viscosity coefficient, $\beta_{max,k}$, used in the void fraction equation. Because the void fraction equation can be treated as a hyperbolic scalar equation with an unique eigenvalue $\mbold u_{int}$, the first-order viscosity coefficient is defined by analogy with Burger's equation \cite{jlg1, jlg2} as follows:
%%
%%
%\begin{equation}\label{eq:def-beta-max-sen-sect4}
%\beta_{max,k}( \mbold r, t) = \frac{h}{2} || \mbold u_{int}( \mbold r, t) ||.
%\end{equation}
%%
%After defining the first-order viscosity coefficients for each phase, we focus our attention to the entropy viscosity coefficients denoted by the subscript $e$ in \eqt{eq:def-visc-sem-sct4}. We first choose to investigate the definitions of $\mu_{e,k}$ and $\kappa_{e,k}$. The entropy viscosity coefficients are set proportional to the entropy residual given in \eqt{eq:ent_res}, that is known to be positive and peaked in the shock region. 
%%
%\begin{equation}\label{eq:ent_res}
%\resi_k(\mbold r,t) := \matder{s_k} =  \partial_t s_k + \mbold u_k \cdot \grad s_k
%\end{equation}
%%
%It is also accounted for the jumps of quantities that will be determined further. The objective is to be able to track spatially and temporally any shock and discontinuity forming in the computational domain. In \cite{Marco_paper_low_mach}, it was demonstrated the usefulness of recasting the entropy residual as a function of pressure, velocity, density and speed of sound as shown in \eqt{eq:ent_res-sem-sct4}. The alternative expression of the entropy residual denoted by $\resinew_k(\mbold r,t)$, no longer requires an analytical expression of the entropy $s_k$ and experiences the same variations (in absolute value) as the original definition of the entropy residual (\eqt{eq:ent_res}).
%%
%\begin{equation}\label{eq:ent_res-sem-sct4}
%\resi_k(\mbold r,t)  = \matder{s_k} = \frac{(s_e)_k}{(P_e)_k} \left( \underbrace{\matder{P_k} - c_k^2 \matder{\rho_k} }_{\resinew_k(\mbold r,t)} \right) ,
%\end{equation} 
%%
%Using the new expression of the entropy residual $\resinew_k$, we now propose a definition, given in \eqt{eq:visc_definition-sct4}, for the phasic entropy viscosity coefficients $\mu_{e,k}$ and $\kappa_{e,k}$ that also accounts for jumps, $J_k$, of some function of the pressure and density for generality purpose. The jump helps at tracking contact waves or discontinuities other than shock that are not seen by the entropy residual. Its definition will be detailed in \sct{sec:low-Mach}. A distinct normalization parameter is also introduced for each viscosity coefficient that is used for dimensionality purpose: a quick dimensional study of the dissipative terms shows that the viscosity coefficients are kinematic viscosity ($m^2 \cdot s^{-1}$). Thus, the normalization parameters has units in pressure and its final definition will be determined by a low-Mach asymptotic limit of \eqt{eq:sev_equ-with-diss-terms} in order to ensure well-scaled dissipative terms for all-Mach flows. We see here the advantage of using the new expression for the entropy residual $\resinew_k$ that offers more diversity in the choice of the normalization parameters: the pressure itself and combination of the density, the sound speed and the norm of the velocity.
%%
%\begin{subequations}\label{eq:visc_definition-sct4}
%\begin{equation}
%\mu_{e,k}(\mbold r,t)    = h^2 \frac{\max\left( | \resinew_k(\mbold r_q,t) |\,, || J_k^\mu || \right)}{\norm_{P,k}^\mu},
%\end{equation} 
%\text{and} 
%\begin{equation}
%\kappa_{e,k}(\mbold r,t) = h^2 \frac{\max\left( | \resinew_k(\mbold r_q,t) |\,, || J_k^\kappa || \right)}{\norm_{P,k}^\kappa} \, .
%\end{equation}
%\end{subequations}
%%
%It remains to define the entropy viscosity coefficient $\beta_{e,k}$. For the purpose of this paragraph, let us consider the scalar void fraction equation and assume that the interface velocity $\mbold u_{int}$ is given. Because it is a scalar hyperbolic equation, it is proposed to define the entropy viscosity coefficients on the same model as what is done for Burger's equation \cite{jlg1, jlg2}. Thus, the entropy viscosity viscosity coefficient $\beta_e$ is defined as a function of an entropy residual, $R_{k}^\alpha$, derived from the void fraction equation for phase $k$, and the jump of a function of the void fraction, $J_k^\alpha$, as shown in \eqt{eq:def-beta-sen-sect4}.
%%
%\begin{align}\label{eq:def-beta-sen-sect4}
%\beta_{e,k}( \mbold r, t) = h^2 \frac{\max\left( | R_{k}^\alpha(\mbold r_q,t) |\,, || J_k^\alpha|| \right)}{\norm_{\alpha, k}^\beta} \,
%\end{align}
%%
%We also introduce a normalization parameter, $\norm_{\alpha,k}^\beta$, whose expression will be further investigated in \sct{sec:low-Mach-sem}. To derive the entropy residual, $R_{\alpha,k}$, we consider the void fraction equation for phase $k$ with its viscous regularization and assume the existence of an entropy denoted by $\eta_k(\alpha_k)$ \cite{Leveque}:
%%
%\begin{equation}\label{eq:vf-sem-sct4}
%\partial_t \left(A \alpha_k \right) + A \mbold u_{int} \cdot \grad \alpha_k = \div \left( \beta_k A \grad \alpha_k \right)
%\end{equation}
%% 
%After multiplying by $\frac{\text{d} \eta (\alpha_k)}{\text{d} \alpha_k}$ and using the chain rule, an expression for the entropy equation is obtained:
%%
%\begin{equation}\label{eq:vf-sem2-sct4}
%\underbrace{\partial_t \left(A \eta(\alpha_k) \right) + A \mbold u_{int} \cdot \grad \eta(\alpha_k)}_{R_{k}^\alpha} = \frac{\text{d} \eta (\alpha_k)}{\text{d} \alpha_k} \div \left( \beta_k A \grad \alpha_k \right)
%\end{equation}
%% 
%The entropy residual, $R_{k}^\alpha$, is defined as the left hand side of \eqt{eq:vf-sem2-sct4} and is known to be peaked in the shock region and positive when assuming convexity of the entropy $\eta_k$ with respect to $\alpha_k$ \cite{Leveque}. Such a behavior is identical to the entropy residual $\resinew_k$ defined in \eqt{eq:ent_res-sem-sct4}, and will allow detection of the shock wave in the void fraction profile when used in the definition of the entropy viscosity coefficient $\beta_{e,k}$.
%%Because \eqt{eq:vf-sem2-sct4} is identical to \eqt{eq:weak_sol8_sct1b}, it is concluded that $R_{\alpha,k} \geq 0$ when assuming $\eta$ convex with respect to $\alpha_k$, which justifies the definition of the entropy viscosity viscosity coefficient $\beta_{e,k}$ given in \eqt{eq:def-beta-sen-sect4} based on \eqt{sec:evm_hyp_sc_sct1b}. The entropy function is taken equal to $\eta(\alpha_k) = \frac{\alpha_k^2}{2}$ which is convex.
%%
%
%At this point of the paper, the definition of the viscosity coefficients are not finalized: the jumps and normalization parameters still have to be defined. The normalization parameters are derived from a low-Mach asymptotic limit analysis which is the purpose of the next section.
%%%%%%%%%%%%%%%%%%%%%%%%%%%%%%%%%%%%%%%%%%%%%%%%%%%%%%%%%%%%%%%%%%%%%%%%%%%%%
%%%%%%%%%%%%%%%%%%%%%%%%%%%%%%%%%%%%%%%%%%%%%%%%%%%%%%%%%%%%%%%%%%%%%%%%%%%%%
\section{Spatial and Temporal Discretizations} \label{sec:disc}
%%%%%%%%%%%%%%%%%%%%%%%%%%%%%%%%%%%%%%%%%%%%%%%%%%%%%%%%%%%%%%%%%%%%%%%%%%%%%
%%%%%%%%%%%%%%%%%%%%%%%%%%%%%%%%%%%%%%%%%%%%%%%%%%%%%%%%%%%%%%%%%%%%%%%%%%%%%
In this section, we briefly describe the spatial and temporal discretizations and the solution techniques 
used to solve the system of equations EQUATION. For conciseness, we re-write the system of 
equations in the following form:
\begin{equation}
\label{eq:form}
\partial_t \mathbf{U}_k + \div \mathbf{F}_k \left( \mathbf{U}_k \right) = \mathbf{R}_k \left( \mathbf{U}_k \right) + \mathbf{N}_k \left( \mathbf{U}_k \right) + \div \mathbf{D}_k(\mathbf{U}_k) \grad \mathbf{U}_k
\end{equation}
where $\mathbf{U}_k=[(\alpha A )_k, \, (\alpha \rho A)_k,\, (\alpha \rho \mbold{u} A)_k,\, (\alpha \rho E A)_k]^T$ is the solution vector, $\mathbf{F}_k \left( \mathbf{U}_k \right)$ denotes the inviscid flux, $\div D_k (\mathbf{U}_k) \grad \mathbf{U}_k$ is the dissipative flux and $\mathbf{N}_k \left( \mathbf{U}_k \right)$ and $\mathbf{R}_k \left( \mathbf{U}_k \right)$ contain the non-conservative and relaxation terms, respectively. 
\begin{equation}
  \mathbf{F} \equiv
  \begin{bmatrix}
      0     \\
    ( \alpha \rho u A )_k     \\
    \left[ \alpha \left(\rho u^2 + P \right) A \right ]_k  \\
    \left[ \alpha u (\rho E + P) A \right]_k
  \end{bmatrix},
%\end{equation}
%\begin{equation}
  \mathbf{N} \equiv
  \begin{bmatrix}
      - A \mbold{u}_{int} \cdot \grad \alpha_k     \\
    0     \\
    \alpha_k P_k \grad A + P_{int} A \grad \alpha_k  \\
    P_{int} A \mbold{u}_{int} \cdot \grad \alpha_k
  \end{bmatrix} \nonumber
\end{equation}
\begin{equation}
\text{and }
  \mathbf{R} \equiv
  \begin{bmatrix}
      A \mu_P \left( P_k - P_j \right)     \\
    0     \\
    A \lambda_u \left( \mbold u_j - \mbold u_k \right) \\
    - \bar{P}_{int} A \mu_P \left( P_k - P_j \right) + \bar{u}_{int} A \lambda_u \left( \mbold u_j - \mbold u_k \right)
  \end{bmatrix}. \nonumber
\end{equation}
%
%%%%%%%%%%%%%%%%%%%%%%%%%%%%%%%%%%%%%%%%%%%%%%%
%\subsection{An all-speed formulation of the viscosity coefficients}\label{eq:visc-formulation}
%%%%%%%%%%%%%%%%%%%%%%%%%%%%%%%%%%%%%%%%%%%%%%%
%The study of the above limit cases yields two different possible scalings for the phasic Reynolds number: 
%$\Re_{k,\infty} = 1$ in the low-Mach limit and $\Re_{k,\infty}  = M_{k,\infty}^2$ for non-isentropic flows, 
%whereas the phasic numerical P\'eclet numbers ($\Pe_{k,\infty}^\kappa$ and $\Pe_{k,\infty}^\beta$) always scales as one. In order to have a stabilization method 
%valid for a wide range of Mach numbers, from very low-Mach to supersonic flows, these two scalings should 
%be combined in a unique definition. 
%
%We begin with the  normalization parameter $\norm_{P,k}^\kappa$. Using the 
%definition of the viscosity coefficients given in \eqt{eq:visc_definition-sct4} and the scaling of 
%\eqt{eq:norm_param}, it can be shown that:
%%
%\begin{equation}
%\label{eq:norm_relation}
%\kappa_{k,\infty} = \frac{ \rho_{k,\infty} c_{k,\infty}^2 u_{k,\infty} L_{k,\infty} }{ \norm_{k,P,\infty}^{\kappa} } \, ,
%\end{equation}
%%
%where $\norm_{k,P,\infty}$ is the reference far-field quantity for the normalization parameter $\norm_{k,P}$. 
%Substituting \eqt{eq:norm_relation} into \eqt{eq:ref_numb} and recalling that the phasic numerical P\'eclet 
%number scales as unity, we obtain:
%%
%\begin{equation}
%\label{eq:norm_relation_bis}
%\norm_{k,P,\infty}^{\kappa} = \Pe_{k,\infty} \rho_{k,\infty} c_{k,\infty}^2 = \rho_{k,\infty} c_{k,\infty}^2 \, .
%\end{equation}
%%
%\eqt{eq:norm_relation_bis} provides a proper normalization factor to define the $\kappa_k$ viscosity coefficient.
%%
%The derivation for $\norm_{k,P}^\mu$ is similar and yields
%\begin{align}
%\label{eq:norm_ent2}
%&\norm_{k,P,\infty}^\mu = \Re_{k,\infty} \rho_{k,\infty} c_{k,\infty}^2 =  \nonumber \\ 
%& \ \ \ \ \ \ \ \ \ \left\{
%\begin{array}{ll}
% \rho_{k,\infty} ||u_{k,\infty} ||^2       & \text{ for non-isentropic flows} \\
% \rho_{k,\infty} c_{k,\infty}^2 & \text{ for low-Mach flows}
%\end{array}
%\right. \,.
%\end{align}
%A smooth function to transition between these two states is as follows: % the shifted sigmoid function
%\begin{equation}
%% \sigma(M) = \frac{1}{1+\exp\big(-(M-M^\text{thresh})\big)} \,,
% \sigma(M_k) = \frac{\tanh\big(a_k(M_k-M_k^\text{thresh})\big) + | \tanh\big(a_k(M_k-M_k^\text{thresh})\big) | }{2} \,,
%\end{equation}
%where $M_k^\text{thresh}$ is a phasic threshold Mach number value beyond which the flow is no longer 
%considered  to be low-Mach (we use $M_k^\text{thresh}=0.05$), $M_k$ is the local Mach number, 
%and the scalar $a_k$ determines how rapidly the transition from  
%$\norm_{k,P,\infty}^\mu= \rho_k c_k^2$ to $\norm_{k,P}^\mu=\rho_k \| \mbold u_k \|^2$ occurs in the vicinity of $M_k^\text{thresh}$ (we use $a_k = 3$). 
%It is easy to verify that
%\begin{equation}
%\label{eq:norm_ent3}
%\norm_{k,P}^\mu = (1-\sigma(M_k)) \rho_k c_k^2  + \sigma(M_k)  \rho_k ||\mbold u_k ||^2  
%\end{equation}
%satisfies \eqt{eq:norm_ent2}.
%
%It remains to determine the normalization parameter, $\norm_{\alpha,k}^\beta$, for the viscosity coefficient $\beta_k$, by using the scaling of the P\'eclet number $\Pe_{k,\infty}^\beta$ derived from the low-Mach asymptotic limit. Following the same reasoning as above, it yields:
%%
%\begin{equation}
%\label{eq:norm_relation_beta}
%\norm_{k,\alpha,\infty}^\beta = 1 \ ,
%\end{equation}
%%
%where $\norm_{k,\alpha,\infty}$ is the reference far-field quantity for the normalization parameter $\norm_{\alpha,k}$ used in the definition of the viscosity coefficient $\beta_k$ (\eqt{eq:def-beta-sen-sect4}). The normalization parameter scales as one. Then, it is chosen to use the same scaling as for Burger's equation \cite{jlg2} e.g.
%%
%\begin{equation}
%\label{eq:norm_relation_beta2}
%\norm_{k,\alpha}^\beta =  || \eta (\alpha_k) - \bar{\eta}(\alpha_k) ||_\infty\ ,
%\end{equation}
%%
%where $\bar{\eta}$ is the average value of the entropy $\eta$ over the entire computational domain.
%
%At this point of the paper, we have derived a viscous regularization for the seven-equation model consistent with the entropy minimum principle, and defined viscosity coefficients for all-Mach flows. 
%===================================================================================================
\subsection{Spatial and Temporal Discretizations} \label{sec:spatial-disc}
%===================================================================================================
The system of equations given in \eqt{eq:form} is discretized using a continuous Galerkin finite element 
method and temporal integrators available through the MOOSE multiphysics framework \cite{Moose}.
%---------------------------------------------------------------------------------------------------
\subsubsection{Continuous Finite Elements} 
%---------------------------------------------------------------------------------------------------
In order to apply the continuous finite element method, \eqt{eq:form} is multiplied by a test function 
$\mathbf W(\mbold{r})$, integrated by parts and each integral is decomposed into a sum of integrals over 
each element $K$ of the discrete mesh $\Omega$. The following weak form is obtained:
\begin{multline}\label{eq:cfem}
\sum_K \int_{K} \partial_t \mathbf U \, \mathbf W - \sum_K \int_{K} \mathbf F (\mathbf U) \cdot \grad \mathbf W + \int_{\partial \Omega} \mathbf F (\mathbf U) \cdot \mbold{n} \mathbf W - \sum_K \int_{K} \left( \mathbf N (\mathbf U) + \mathbf R (\mathbf U) \right) \mathbf W  \\
+ \sum_K \int_{K} D(\mathbf U) \grad \mathbf U \cdot \grad \mathbf W 
- \int_{\partial \Omega} D(\mathbf U) \grad \mathbf U \cdot \mbold{n} \, \mathbf W = 0 \,.
\end{multline}
The integrals over the elements $K$ are evaluated using a numerical quadrature. The MOOSE framework 
provides a wide range of test functions and quadrature rules. Linear Lagrange polynomials are employed 
as test functions in the results section. Second-order spatial convergence will be demonstrated for smooth solutions. 
%
%---------------------------------------------------------------------------------------------------
\subsubsection{Temporal integration} 
%---------------------------------------------------------------------------------------------------
The MOOSE framework offers both first- and second-order explicit and implicit temporal integrators. 
In all of the numerical examples presented in \sct{sec:results}, the temporal derivative  will be 
evaluated using the second-order, backward difference temporal integrator BDF2 \cite{bdf2}. By considering three 
consecutive solutions, $\mathbf U^{n-1}$, $\mathbf U^n$ and $\mathbf U^{n+1}$, at times $t^{n-1}$, $t^n$ and $t^{n+1}$, respectively, BDF2 can be expressed as:
\begin{equation}
\label{eq:BDF2}
\int_{K} \partial_t \mathbf U \, \mathbf  W = \int_{K} \left( \omega_0 \mathbf U^{n+1}  + \omega_1 \mathbf U^n + \omega_2 \mathbf U^{n-1} \right) \mathbf W \,,
\end{equation}
%
with
\begin{multline}
\omega_0 =\frac{2\Delta t^{n+1}+\Delta t^n}{\Delta t^{n+1} \left( \Delta t^{n+1}+\Delta t^n \right)} \, , \ 
\omega_1 = -\frac{\Delta t^{n+1}+\Delta t^n}{\Delta t^{n+1} \Delta t^n}  \, , \\
\text{ and } \omega_2 = \frac{\Delta t^{n+1}}{\Delta t^n \left( \Delta t^{n+1} + \Delta t^n \right)} \nonumber
\end{multline}
where $\Delta t^{n} = t^n-t^{n-1}$ and $\Delta t^{n+1} = t^{n+1}-t^{n}$.
%---------------------------------------------------------------------------------------------------
\subsection{Boundary conditions} \label{sec:bc}
%---------------------------------------------------------------------------------------------------
Boundary conditions for the seven-equation model are challenging because of the wave-dominated nature of the equations but also because of the non-conservative form of the void fraction equation. 
%Unlike the continuity, momentum and energy equations of each phase, the flux of the void fraction equation cannot be integrated by part because it is not written in a conservative form. 
The boundary condition for the void fraction equation (EQUATION) is treated independently of the other equations (continuity, momentum and energy for each phase) for two reasons: (i) it is a simple advection equation with the real eigenvalue $\mbold u_{int}$, and (ii), the hyperbolic flux, $\mbold u_{int} \cdot \grad \alpha_k$, is not integrated by part since not under a conservative form. The sign of the dot product between the eigenvalue and the outward normal to the boundary, $\mbold u_{int} \cdot \mbold n$, determines the nature of the boundary: negative for an inlet and positive for an outlet. For the later case, the physical information exits the computational domain and does not require any particular treatment. In the former case, the physical information enters the computational domain which requires to specify a value for the void fraction. Since there is no flux at the boundary coming from the integration by part of the hyperbolic flux, the boundary value is imposed by using a Dirichlet boundary condition in the void fraction equation.   
Our implementation of the boundary conditions for the continuity, momentum and energy equations, is inspired by the method described in \cite{SEM} and was adapted for a time implicit solver \cite{Marco_dissertation}. The boundary type is identified from the study of the sign of the eigenvalues that depends on the Mach number. The numerical results presented in \sct{sec:results} were all obtained by using subsonic stagnation and static pressure boundary conditions for the inlet and outlet, respectively. The boundary flux is computed from the supplied variables at the boundary and also by iterating on a given number of variables (depending on the sign of the eigenvalues) though the implicit solver to transmit information from inside the computational domain toward the boundary. 
%After recasting the system of equation using the primitive variables, a characteristic decomposition is performed to compute the appropriate values of the pressure, density and velocity of each phase, used in the flux at the boundaries. 
%\tcb{I will give more details in the future}
%\tcr{say a bit more, give 1-2 sentences to describe how to impose subsonic/supersonic inlet/outlet BCs}. \tcb{Is it enough?}

%For each numerical solution presented in \sct{sec:results}, the type of boundary conditions used 
%will be specified and taken among the following: supersonic inlet, subsonic inlet (stagnation pressure boundary), 
%subsonic outlet, and supersonic outlet. 
%and subsonic inlet \tcr{???} (static pressure boundary). \\
The artificial diffusion coefficient $\mbold D(\mathbf  U)$ is set to zero at the boundary of the computational 
domain so that the boundary term 
$\int_{\partial \Omega} \mbold D(\mathbf  U) \grad \mathbf  U \cdot \mbold{n} \, \mathbf W$ stemming from the 
integration by parts of the artificial dissipative terms in \eqt{eq:cfem} is ignored.
%---------------------------------------------------------------------------------------------------
\subsection{Solver} \label{sec:solver}
%---------------------------------------------------------------------------------------------------
A Jacobian-free-Newton-Krylov (JFNK) method is used to solve for the solution at the end of each time step. 
An approximate Jacobian matrix of the discretized equations was derived and implemented. Obtaining the 
matrix entries requires that the partial derivatives of pressure with respect to the conservative variables 
be known (this is relatively simple for the stiffened and ideal gas equations of state but may be more 
complex for general equations of state). The contributions of the artificial dissipative terms to the 
Jacobian matrix are approximated by lagging the viscosity coefficients (computing them with the previous solution). 
For instance, this is shown in \eqt{eq:jac_diss_term} for the dissipative terms present in the continuity equation:
\begin{equation}
\label{eq:jac_diss_term}
\frac{\partial}{\partial \mathbf U} \left( \kappa \grad \rho \cdot \grad W \right) \simeq \kappa \div \frac{\partial \rho}{\partial \mathbf U} \grad W ,
\end{equation}  
where $\mathbf  U$ denotes any of the conservative variables and $W$ denotes the component of $\mathbf  W$ associated with the 
continuity equation. In the above, we have neglected $\frac{\partial \kappa}{\partial \mathbf U}$.
%%%%%%%%%%%%%%%%%%%%%%%%%%%%%%%%%%%%%%%%%%%%%%%%%%%%%%%%%%%%%%%%%%%%%%%%%%%%%
%%%%%%%%%%%%%%%%%%%%%%%%%%%%%%%%%%%%%%%%%%%%%%%%%%%%%%%%%%%%%%%%%%%%%%%%%%%%%
\section{$1$-D numerical results}\label{sec:results}
%%%%%%%%%%%%%%%%%%%%%%%%%%%%%%%%%%%%%%%%%%%%%%%%%%%%%%%%%%%%%%%%%%%%%%%%%%%%%
%%%%%%%%%%%%%%%%%%%%%%%%%%%%%%%%%%%%%%%%%%%%%%%%%%%%%%%%%%%%%%%%%%%%%%%%%%%%%
\begin{itemize}
\item simple advection problem
\item shock tube with two independent fluids: exact solution and could do convergence test for this particular test
\item shock tube with infinite relaxation coefficients
\item $1$-D nozzle with two independent fluids
\item $1$-D nozzle with infinite relaxation coefficients
\item $1$-D nozzle with infinite relaxation coefficients,  mass and heat transfer
\end{itemize}
%%%%%%%%%%%%%%%%%%%%%%%%%%%%%%%%%%%%%%%%%%%%%%%%%%%%%%%%%%%%%%%%%%%%%%%%%%%%%
%%%%%%%%%%%%%%%%%%%%%%%%%%%%%%%%%%%%%%%%%%%%%%%%%%%%%%%%%%%%%%%%%%%%%%%%%%%%%
\section{Conclusions and future work}\label{sec:conclusion}
%%%%%%%%%%%%%%%%%%%%%%%%%%%%%%%%%%%%%%%%%%%%%%%%%%%%%%%%%%%%%%%%%%%%%%%%%%%%%
%%%%%%%%%%%%%%%%%%%%%%%%%%%%%%%%%%%%%%%%%%%%%%%%%%%%%%%%%%%%%%%%%%%%%%%%%%%%%
\bibliography{mybibfile}
\end{document}