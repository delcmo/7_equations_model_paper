\documentclass{article}
%%%%%%%%%%%%%%%%%%%%%%%%%%%%%%%%%%%%%%%%%%%%%%%%%%%%%%%%%%%%%%%%%%%%%%%%%%%%%%%%%%%%%%%%%%%%%%%%%%%%%%%%%%%%%%%%%%%%%%%%%%%%%%%%%%%%%%%%%%%%%%%%%%%%%%%%%%%%%%%%%%%%%%%%%%%%%%%%%%%%%%%%%%%%%%%%%%%%%%%%%%%%%%%%%%%%%%%%%%%%%%%%%%%%%%%%%%%%%%%%%%%%%%%%%%%%
\usepackage{amsmath,amssymb}
% more math
\usepackage{amsfonts}
\usepackage{amssymb}
\usepackage{amstext}
\usepackage{amsbsy}

\usepackage{color}
\newcommand{\mt}[1]{\marginpar{\small #1}}
%%%%%%%%%%%%%%%%%%%%%%%%%%%%%%%%%%%%%%%%%%%%%%%%%%%%%%%%%%%%%%%%%%%%
% new commands
\newcommand{\nc}{\newcommand}
% operators
\renewcommand{\div}{\vec{\nabla}\! \cdot \!}
\newcommand{\grad}{\vec{\nabla}}
% latex shortcuts
\newcommand{\bea}{\begin{eqnarray}}
\newcommand{\eea}{\end{eqnarray}}
\newcommand{\be}{\begin{equation}}
\newcommand{\ee}{\end{equation}}
\newcommand{\bal}{\begin{align}}
\newcommand{\eali}{\end{align}}
\newcommand{\bi}{\begin{itemize}}
\newcommand{\ei}{\end{itemize}}
\newcommand{\ben}{\begin{enumerate}}
\newcommand{\een}{\end{enumerate}}
% DGFEM commands
\newcommand{\jmp}[1]{[\![#1]\!]}                     % jump
\newcommand{\mvl}[1]{\{\!\!\{#1\}\!\!\}}             % mean value
\newcommand{\keff}{\ensuremath{k_{\textit{eff}}}\xspace}
% shortcut for domain notation
\newcommand{\D}{\mathcal{D}}
% vector shortcuts
\newcommand{\vo}{\vec{\Omega}}
\newcommand{\vr}{\vec{r}}
\newcommand{\vn}{\vec{n}}
\newcommand{\vnk}{\vec{\mathbf{n}}}
\newcommand{\vj}{\vec{J}}
% extra space
\newcommand{\qq}{\quad\quad}
% common reference commands
\newcommand{\eqt}[1]{Eq.~(\ref{#1})}                     % equation
\newcommand{\fig}[1]{Fig.~\ref{#1}}                      % figure
\newcommand{\tbl}[1]{Table~\ref{#1}}                     % table

\newcommand{\ud}{\,\mathrm{d}}

\newcommand{\tcr}[1]{\textcolor{red}{#1}}
%%%%%%%%%%%%%%%%%%%%%%%%%%%%%%%%%%%%%%%%%%%%%%%%%%%%%%%%%%%%%%%%%%%%

\begin{document}

\begin{center}
{ \Large Answers to Reviewer \#2}
\end{center}

\bigskip

\noindent Ref. No.: MS\# NSE16-46\\
Title: ``Application of the reactor system code RELAP-7 to single- and two-phase flow water-hammer problems', \\
{\it Nuclear Science and Engineering}\\

\bigskip
\bigskip

{\color{blue}
1. The title "Application of the reactor system code RELAP-7 to single- and two-phase flow waterhammer Problems" is very misleading and should be changed since there is no simulation of two-phase waterhammer as it may occur in reactor transient. The fluid properties are not steam-water, there is not interphase heat and mass transfers although it is well known that they play a major role in pressure wave propagation
\\}
We agree and modified the title accordingly. We also added tests that include simulation of liquid water, air, and steam shock tubes in Section 3.2. 
\bigskip

{\color{blue}
2. It is written in the introduction that:
This two-phase flow model is strictly hyperbolic, as opposed to the 6- equation model RELAP-5. It is well established that hyperbolic conservation laws can develop shocks and discontinuities [6] and, therefore, require stabilization of the discretized equations.
It seems that one motivation of this 7-equation model is to better predict waterhammer and pressure wave propagation in reactor transient simulation. But several studies show good capabilities of RELAP 5 code to well predict waterhammer (I. Tiselj et al, 2000, 2003, Sokolowski et al., 2012, Kaliatka et al. 2005). In the same way the CATHARE code is also based on a 6 equation model - with the difference of being hyperbolic- and has shown also good capabilities. (P. Nika , G. Serre , 2005, Serre and Bestion, 2001) \\
}
We agree with you that the six-equation model can accurately simulate water hammers but it is only hyperbolic in some limits thanks to the addition of a pressure correction term in the momentum and the energy equations.
\bigskip

{\color{blue}
3. You should mention previous work (see Ref below) and say what is your expected added value. \\
}

{\color{blue}
4. Why not using interfacial heat and mass transfer which play a major role in waterhammer? \\
}

\bigskip

{\color{blue}
5. Why not using interfacial friction which play a major role in waterhammer? \\
}

\bigskip

{\color{blue}
6. Why not using added mass force which is known to play an important role in pressure propagation? \\
}

\bigskip

{\color{blue}
7. You cannot conclude that: " This work contributes to the assessment of the stabilization techniques for reactor flow problems computed with RELAP-7". No reactor flow problem was computed. \\
}
We agree with you. We added new tests involving air, liquid water and steam, and that are more relevant to nuclear reactors. 
\bigskip

{\color{blue}
The paper does not give adequate credit to earlier work in the field. \\
}
We modified the introduction by adding references of previous work (see lines 26 to 47).
\tcr{make sure the lines numbers are correct once we are done modifying the manuscript}

\bigskip

{\color{blue}
Then I would recommend to accept the paper only if a real waterhammer case (see Ref below) was simulated with RELAP-7. This requires a significant additional work.
Otherwise this work as it is may be submitted to a mathematical or numerical review , not to NSE. \\
}
Thank you for the review of our paper. We hope that the modifications we made in the paper will convince you that this paper deserves to be published in NSE.
\end{document}

