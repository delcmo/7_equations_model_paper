%%
%% This is file `elsarticle-template-num.tex',
%% generated with the docstrip utility.
%%
%% The original source files were:
%%
%% elsarticle.dtx  (with options: `numtemplate')
%% 
%% Copyright 2007, 2008 Elsevier Ltd.
%% 
%% This file is part of the 'Elsarticle Bundle'.
%% -------------------------------------------
%% 
%% It may be distributed under the conditions of the LaTeX Project Public
%% License, either version 1.2 of this license or (at your option) any
%% later version.  The latest version of this license is in
%%    http://www.latex-project.org/lppl.txt
%% and version 1.2 or later is part of all distributions of LaTeX
%% version 1999/12/01 or later.
%% 
%% The list of all files belonging to the 'Elsarticle Bundle' is
%% given in the file `manifest.txt'.
%% 

%% Template article for Elsevier's document class `elsarticle'
%% with numbered style bibliographic references
%% SP 2008/03/01

%\documentclass[preprint,12pt]{elsarticle}
\documentclass[preprint,10pt]{elsarticle}
%\documentclass[final,3p,times]{elsarticle} 

%% Use the option review to obtain double line spacing
%% \documentclass[authoryear,preprint,review,12pt]{elsarticle}

%% Use the options 1p,twocolumn; 3p; 3p,twocolumn; 5p; or 5p,twocolumn
%% for a journal layout:
%% \documentclass[final,1p,times]{elsarticle}
%% \documentclass[final,1p,times,twocolumn]{elsarticle}
%% \documentclass[final,3p,times]{elsarticle}
%% \documentclass[final,3p,times,twocolumn]{elsarticle}
%% \documentclass[final,5p,times]{elsarticle}
%% \documentclass[final,5p,times,twocolumn]{elsarticle}

%% if you use PostScript figures in your article
%% use the graphics package for simple commands
\usepackage{float}
\usepackage{color}
\usepackage{caption}
\usepackage{subcaption}
\usepackage{appendix}
%% or use the graphicx package for more complicated commands
\usepackage{graphicx}
%% or use the epsfig package if you prefer to use the old commands
%% \usepackage{epsfig}

%% The amssymb package provides various useful mathematical symbols 
%% The amsthm package provides extended theorem environments
\usepackage{amssymb}
\usepackage{amsmath}
% more math
\usepackage{amsfonts}
\usepackage{amstext}
\usepackage{amsbsy}
\usepackage{mathbbol} 
%% The lineno packages adds line numbers. Start line numbering with
%% \begin{linenumbers}, end it with \end{linenumbers}. Or switch it on
%% for the whole article with \linenumbers.
\usepackage{lineno}

\journal{Journal of Comp. Phys.}
%%%%%%%%%%%%%%%%%%%%%%%%%%%%%%%%%%%%%%%%%%%%%%%%%%%%%%%%%%%%%%%%%%%%
% operators
\renewcommand{\div}{\vec{\nabla}\! \cdot \!}
\newcommand{\grad}{\vec{\nabla}}
\newcommand{\divv}[1]{\boldsymbol{\nabla}^{#1}\! \cdot \!}
\newcommand{\gradd}[1]{\vec{\nabla}^{#1}}
\newcommand{\mbold}[1]{\boldsymbol#1}
% latex shortcuts
\newcommand{\bea}{\begin{eqnarray}}
\newcommand{\eea}{\end{eqnarray}}
\newcommand{\be}{\begin{equation}}
\newcommand{\ee}{\end{equation}}
\newcommand{\bal}{\begin{align}}
\newcommand{\eali}{\end{align}}
\newcommand{\bi}{\begin{itemize}}
\newcommand{\ei}{\end{itemize}}
\newcommand{\ben}{\begin{enumerate}}
\newcommand{\een}{\end{enumerate}}
% DGFEM commands
\newcommand{\jmp}[1]{[\![#1]\!]}                     % jump
\newcommand{\mvl}[1]{\{\!\!\{#1\}\!\!\}}             % mean value
\newcommand{\keff}{\ensuremath{k_{\textit{eff}}}\xspace}
% shortcut for domain notation
\newcommand{\D}{\mathcal{D}}
% vector shortcuts
\newcommand{\vo}{\vec{\Omega}}
\newcommand{\vr}{\vec{r}}
\newcommand{\vn}{\vec{n}}
\newcommand{\vnk}{\vec{\mathbf{n}}}
\newcommand{\vj}{\vec{J}}
\newcommand{\eig}[1]{\| #1 \|_2}
%
\newcommand{\EI}{\mathcal{E}_h^i}
\newcommand{\ED}{\mathcal{E}_h^{\partial \D^d}}
\newcommand{\EN}{\mathcal{E}_h^{\partial \D^n}}
\newcommand{\ER}{\mathcal{E}_h^{\partial \D^r}}
\newcommand{\reg}{\textit{reg}}
%
\newcommand{\norm}{\textrm{norm}}
\renewcommand{\Re}{\textrm{Re}}
\newcommand{\Pe}{\textrm{P\'e}}
\renewcommand{\Pr}{\textrm{Pr}}
%
\newcommand{\resi}{R}
%\newcommand{\resinew}{\tilde{D}_e}
\newcommand{\resinew}{\widetilde{\resi}}
\newcommand{\resisource}{\widetilde{\resi}^{source}}
\newcommand{\matder}[1]{\frac{\textrm{D} #1}{\textrm{D} t}}
%
% extra space
\newcommand{\qq}{\quad\quad}
% common reference commands
\newcommand{\eqt}[1]{Eq.~(\ref{#1})}                     % equation
\newcommand{\fig}[1]{Fig.~\ref{#1}}                      % figure
\newcommand{\tbl}[1]{Table~\ref{#1}}                     % table
\newcommand{\sct}[1]{Section~\ref{#1}}                   % section
\newcommand{\app}[1]{Appendix~\ref{#1}}                   % appendix
%
\newcommand{\ie}{i.e.,\@\xspace}
\newcommand{\eg}{e.g.,\@\xspace}
\newcommand{\psc}[1]{{\sc {#1}}}
\newcommand{\rs}{\psc{R7}\xspace}
%
\newcommand\br{\mathbf{r}}
%\newcommand{\tf}{\varphi}
\newcommand{\tf}{b}
%
\newcommand{\tcr}[1]{\textcolor{red}{#1}}
\newcommand{\tcb}[1]{\textcolor{blue}{#1}}
\newcommand{\mt}[1]{\marginpar{ {\tiny #1}}}
%
\bibliographystyle{elsarticle-num}
%%%%%%%%%%%%%%%%%%%%%%%%%%%%%%%%%%%%%%%%%%%%%%%%%%%%%%%%%%%%%%%%%%%%%
%
%   BEGIN DOCUMENT
%
%%%%%%%%%%%%%%%%%%%%%%%%%%%%%%%%%%%%%%%%%%%%%%%%%%%%%%%%%%%%%%%%%%%%%
\begin{document}
%%%%%%%%%%%%%%%%%%%%%%%%%%%%%%%%%%%%%%%%%%%%%%%%%%%%%%%%%%%%%%%%%%%%%
%\linenumbers
The liquid phase obeys the following mass, momentum and energy balance equations, supplemented by a non-conservative volume-fraction equation:
%
\begin{subequations}\label{eq:liq-7-eqn-sect5}
\begin{align}
  % liquid mass conservation
  \label{multi-d-7-equ-liq}
  \frac{\partial \left( \alpha \rho \right)_{liq} A}{\partial t}
  + \div \left( \alpha \rho \mbold u A\right)_{liq}
  &= 0
\end{align}
\begin{align}
  % liquid momentum
  &\frac{\partial \left( \alpha \rho \mbold u \right)_{liq} A}{\partial t}
  + \div \left[ \alpha_{liq} A \left( \rho \mbold u \otimes \mbold u + P \mathbb{I} \right)_{liq} \right]
      \nonumber
  \\
  &= P_{int} A \grad \alpha_{liq} + P_{liq} \alpha_{liq} \grad A
  + A \lambda_u (\mbold u_{vap} - \mbold u_{liq})
\end{align}
\begin{align}
  % liquid total energy
  &\frac{\partial \left( \alpha \rho E \right)_{liq} A}{\partial t}
  + \div \left[ \alpha_{liq} \mbold u_{liq} A \left( \rho E + P \right)_{liq} \right]
  = P_{int} \mbold u_{int} A \grad \alpha_{liq} 
            \nonumber
  \\
  &- \bar{P}_{int} A \mu_P (P_{liq} - P_{vap})
  + \bar{\mbold u}_{int} A \lambda_u (\mbold u_{vap} - \mbold u_{liq})
\nonumber 
\end{align}
\begin{align}
  % liquid volume fraction
  \label{eqn:multi-d-7-eqn-liq-vol}
  \frac{\partial \alpha_{liq} A}{\partial t} + A\mbold u_{int} \cdot \grad \alpha_{liq}
  &= A \mu_P (P_{liq} - P_{vap})
\end{align}
\end{subequations}
%
On the same model, the equations for the vapor phase are:
%
\begin{subequations}\label{eq:vap-7-eqn-sect5}
\begin{align}
  \label{multi-d-7-equ-vap}
  % vapor mass conservation
  \frac{\partial \left( \alpha \rho A\right)_{vap}}{\partial t}
  + \div \left( \alpha \rho \mbold u \right)_{vap} A
  =  0
\end{align}
\begin{align}
  % vapor momentum
  &\frac{\partial \left( \alpha \rho u \right)_{vap} A}{\partial t}
  + \div \left[ \alpha_{vap} A \left( \rho \mbold u \otimes \mbold u + P\mathbb{I} \right)_{vap} \right]
    \\
  \nonumber
  &= P_{int} A \grad \alpha_{vap} + P_{vap} \alpha_{vap} \grad A
  + A \lambda_u (\mbold u_{liq} - \mbold u_{vap})
\end{align}
\begin{align}
  % vapor total energy
 & \frac{\partial \left( \alpha \rho E \right)_{vap} A}{\partial t}
  + \div \left[ \alpha_{vap} \mbold u_{vap} A \left( \rho E + P \right)_{vap} \right]
  = P_{int} \mbold u_{int} A \grad \alpha_{vap}
    \nonumber
  \\ 
  &- \bar{P}_{int} A \mu_P (P_{vap} - P_{liq})
  + \bar{\mbold u}_{int} A \lambda_u (\mbold u_{liq} - \mbold u_{vap})
\nonumber 
\end{align}
\begin{align}
  % vapor phase volume fraction
  \label{eqn:multi-d-7-eqn-vap-vol}
  \frac{\partial \alpha_{vap} A}{\partial t} + A \mbold u_{int} \cdot \grad \alpha_{vap}
  &= A \mu_P (P_{vap} - P_{liq})
\end{align}
\end{subequations}
%
Interfacial variables:
%
\begin{subequations}
\label{eq:int_variables_def}
\begin{align}
  \label{E-R:83}
  P_{int} &= \bar{P}_{int} + \frac{Z_{liq}Z_{vap}}{Z_{liq}+Z_{vap}} \frac{\grad \alpha_{liq}}{|| \grad \alpha_{liq} ||} \cdot (\mbold u_{vap}-\mbold u_{liq})
  \\
  \bar{P}_{int} &= \frac{Z_{vap}P_{liq}+Z_{liq}P_{vap}}{Z_{liq}+Z_{vap}}
\end{align}
%
\begin{align}
  \label{E-R:84}
  \mbold u_{int} &= \bar{\mbold u}_{int} +  \frac{\grad \alpha_{liq}}{|| \grad \alpha_{liq} ||} \frac{P_{vap}-P_{liq}}{Z_{liq}+Z_{vap}}
  \\
  \bar{\mbold u}_{int} &= \frac{Z_{liq} \mbold u_{liq}+Z_{vap}\mbold u_{vap}}{Z_{liq}+Z_{vap}}.
\end{align}
\end{subequations}
%
The pressure, $\mu_P$, and velocity, $\lambda_u$, relaxation coefficients are proportional to each other and function of the interfacial area $A_{int}$:
%
\begin{align}
  \label{E-R:85}
  \lambda_u &= \frac{1}{2} \mu_P Z_{liq} Z_{vap}
  \\
  \label{E-R:86}
  \mu_P &= \frac{A_{int}}{Z_{liq}+Z_{vap}}
\end{align}
%
The specific interfacial area (i.e., the interfacial surface area per unit
volume of two-phase mixture), $A_{int}$, must be specified from some type of
flow regime map or function under the form of a correlation:
%
\begin{equation}\label{eq:Aint-sect4}
A_{int} = A_{int}^{max} \left[ 6.75 \left(1-\alpha_{liq} \right)^2 \alpha_{liq} \right],
\end{equation}
% 
where $A_{int}^{max} = 5100$ $m^2 / m^3$. With such definition, the interfacial area is zero in the limits $\alpha_{liq} = 0$ and $\alpha_{liq} = 1$. \\
%
\textbf{Initial conditions} \\
%
Uniform pressure for both phases: $P_{vap} = P_{liq} = 10^6$ $Pa$. \\
The two phases are at rest: $u_{vap} = u_{liq} = 0$ $m \cdot s^{-1}$ \\
Uniform temperature for both phases:  $T_{vap} = T_{liq} = 300$ $K$ \\
Volume fraction linearly varies from $0.9$ to $0.4$. \\
%
\textbf{Boundary conditions} \\
%
Static pressure: $\alpha = 0.9$, $P = 10^6$ and $T=300$. \\
Solid wall.
\end{document}